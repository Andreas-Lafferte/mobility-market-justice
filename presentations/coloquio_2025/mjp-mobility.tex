% Options for packages loaded elsewhere
% Options for packages loaded elsewhere
\PassOptionsToPackage{unicode}{hyperref}
\PassOptionsToPackage{hyphens}{url}
\PassOptionsToPackage{dvipsnames,svgnames,x11names}{xcolor}
%
\documentclass[
  spanish,
  letterpaper,
  DIV=11,
  numbers=noendperiod,
  oneside]{scrartcl}
\usepackage{xcolor}
\usepackage[left=1in,marginparwidth=2.0666666666667in,textwidth=4.1333333333333in,marginparsep=0.3in]{geometry}
\usepackage{amsmath,amssymb}
\setcounter{secnumdepth}{-\maxdimen} % remove section numbering
\usepackage{iftex}
\ifPDFTeX
  \usepackage[T1]{fontenc}
  \usepackage[utf8]{inputenc}
  \usepackage{textcomp} % provide euro and other symbols
\else % if luatex or xetex
  \usepackage{unicode-math} % this also loads fontspec
  \defaultfontfeatures{Scale=MatchLowercase}
  \defaultfontfeatures[\rmfamily]{Ligatures=TeX,Scale=1}
\fi
\usepackage{lmodern}
\ifPDFTeX\else
  % xetex/luatex font selection
\fi
% Use upquote if available, for straight quotes in verbatim environments
\IfFileExists{upquote.sty}{\usepackage{upquote}}{}
\IfFileExists{microtype.sty}{% use microtype if available
  \usepackage[]{microtype}
  \UseMicrotypeSet[protrusion]{basicmath} % disable protrusion for tt fonts
}{}
\makeatletter
\@ifundefined{KOMAClassName}{% if non-KOMA class
  \IfFileExists{parskip.sty}{%
    \usepackage{parskip}
  }{% else
    \setlength{\parindent}{0pt}
    \setlength{\parskip}{6pt plus 2pt minus 1pt}}
}{% if KOMA class
  \KOMAoptions{parskip=half}}
\makeatother
% Make \paragraph and \subparagraph free-standing
\makeatletter
\ifx\paragraph\undefined\else
  \let\oldparagraph\paragraph
  \renewcommand{\paragraph}{
    \@ifstar
      \xxxParagraphStar
      \xxxParagraphNoStar
  }
  \newcommand{\xxxParagraphStar}[1]{\oldparagraph*{#1}\mbox{}}
  \newcommand{\xxxParagraphNoStar}[1]{\oldparagraph{#1}\mbox{}}
\fi
\ifx\subparagraph\undefined\else
  \let\oldsubparagraph\subparagraph
  \renewcommand{\subparagraph}{
    \@ifstar
      \xxxSubParagraphStar
      \xxxSubParagraphNoStar
  }
  \newcommand{\xxxSubParagraphStar}[1]{\oldsubparagraph*{#1}\mbox{}}
  \newcommand{\xxxSubParagraphNoStar}[1]{\oldsubparagraph{#1}\mbox{}}
\fi
\makeatother


\usepackage{longtable,booktabs,array}
\usepackage{calc} % for calculating minipage widths
% Correct order of tables after \paragraph or \subparagraph
\usepackage{etoolbox}
\makeatletter
\patchcmd\longtable{\par}{\if@noskipsec\mbox{}\fi\par}{}{}
\makeatother
% Allow footnotes in longtable head/foot
\IfFileExists{footnotehyper.sty}{\usepackage{footnotehyper}}{\usepackage{footnote}}
\makesavenoteenv{longtable}
\usepackage{graphicx}
\makeatletter
\newsavebox\pandoc@box
\newcommand*\pandocbounded[1]{% scales image to fit in text height/width
  \sbox\pandoc@box{#1}%
  \Gscale@div\@tempa{\textheight}{\dimexpr\ht\pandoc@box+\dp\pandoc@box\relax}%
  \Gscale@div\@tempb{\linewidth}{\wd\pandoc@box}%
  \ifdim\@tempb\p@<\@tempa\p@\let\@tempa\@tempb\fi% select the smaller of both
  \ifdim\@tempa\p@<\p@\scalebox{\@tempa}{\usebox\pandoc@box}%
  \else\usebox{\pandoc@box}%
  \fi%
}
% Set default figure placement to htbp
\def\fps@figure{htbp}
\makeatother


% definitions for citeproc citations
\NewDocumentCommand\citeproctext{}{}
\NewDocumentCommand\citeproc{mm}{%
  \begingroup\def\citeproctext{#2}\cite{#1}\endgroup}
\makeatletter
 % allow citations to break across lines
 \let\@cite@ofmt\@firstofone
 % avoid brackets around text for \cite:
 \def\@biblabel#1{}
 \def\@cite#1#2{{#1\if@tempswa , #2\fi}}
\makeatother
\newlength{\cslhangindent}
\setlength{\cslhangindent}{1.5em}
\newlength{\csllabelwidth}
\setlength{\csllabelwidth}{3em}
\newenvironment{CSLReferences}[2] % #1 hanging-indent, #2 entry-spacing
 {\begin{list}{}{%
  \setlength{\itemindent}{0pt}
  \setlength{\leftmargin}{0pt}
  \setlength{\parsep}{0pt}
  % turn on hanging indent if param 1 is 1
  \ifodd #1
   \setlength{\leftmargin}{\cslhangindent}
   \setlength{\itemindent}{-1\cslhangindent}
  \fi
  % set entry spacing
  \setlength{\itemsep}{#2\baselineskip}}}
 {\end{list}}
\usepackage{calc}
\newcommand{\CSLBlock}[1]{\hfill\break\parbox[t]{\linewidth}{\strut\ignorespaces#1\strut}}
\newcommand{\CSLLeftMargin}[1]{\parbox[t]{\csllabelwidth}{\strut#1\strut}}
\newcommand{\CSLRightInline}[1]{\parbox[t]{\linewidth - \csllabelwidth}{\strut#1\strut}}
\newcommand{\CSLIndent}[1]{\hspace{\cslhangindent}#1}

\ifLuaTeX
\usepackage[bidi=basic]{babel}
\else
\usepackage[bidi=default]{babel}
\fi
% get rid of language-specific shorthands (see #6817):
\let\LanguageShortHands\languageshorthands
\def\languageshorthands#1{}


\setlength{\emergencystretch}{3em} % prevent overfull lines

\providecommand{\tightlist}{%
  \setlength{\itemsep}{0pt}\setlength{\parskip}{0pt}}



 


\usepackage{booktabs}
\usepackage{longtable}
\usepackage{array}
\usepackage{multirow}
\usepackage{wrapfig}
\usepackage{float}
\usepackage{colortbl}
\usepackage{pdflscape}
\usepackage{tabu}
\usepackage{threeparttable}
\usepackage{threeparttablex}
\usepackage[normalem]{ulem}
\usepackage{makecell}
\usepackage{xcolor}
\KOMAoption{captions}{tableheading}
\makeatletter
\@ifpackageloaded{caption}{}{\usepackage{caption}}
\AtBeginDocument{%
\ifdefined\contentsname
  \renewcommand*\contentsname{Tabla de contenidos}
\else
  \newcommand\contentsname{Tabla de contenidos}
\fi
\ifdefined\listfigurename
  \renewcommand*\listfigurename{Listado de Figuras}
\else
  \newcommand\listfigurename{Listado de Figuras}
\fi
\ifdefined\listtablename
  \renewcommand*\listtablename{Listado de Tablas}
\else
  \newcommand\listtablename{Listado de Tablas}
\fi
\ifdefined\figurename
  \renewcommand*\figurename{Figura}
\else
  \newcommand\figurename{Figura}
\fi
\ifdefined\tablename
  \renewcommand*\tablename{Tabla}
\else
  \newcommand\tablename{Tabla}
\fi
}
\@ifpackageloaded{float}{}{\usepackage{float}}
\floatstyle{ruled}
\@ifundefined{c@chapter}{\newfloat{codelisting}{h}{lop}}{\newfloat{codelisting}{h}{lop}[chapter]}
\floatname{codelisting}{Listado}
\newcommand*\listoflistings{\listof{codelisting}{Listado de Listados}}
\makeatother
\makeatletter
\makeatother
\makeatletter
\@ifpackageloaded{caption}{}{\usepackage{caption}}
\@ifpackageloaded{subcaption}{}{\usepackage{subcaption}}
\makeatother
\makeatletter
\@ifpackageloaded{sidenotes}{}{\usepackage{sidenotes}}
\@ifpackageloaded{marginnote}{}{\usepackage{marginnote}}
\makeatother
\usepackage{bookmark}
\IfFileExists{xurl.sty}{\usepackage{xurl}}{} % add URL line breaks if available
\urlstyle{same}
\hypersetup{
  pdflang={es},
  colorlinks=true,
  linkcolor={blue},
  filecolor={Maroon},
  citecolor={Blue},
  urlcolor={Blue},
  pdfcreator={LaTeX via pandoc}}


\author{}
\date{}
\begin{document}


\pandocbounded{\includegraphics[keepaspectratio]{images/logo_isuc.png}}

\pandocbounded{\includegraphics[keepaspectratio]{images/jusmer_trans.png}}

\section{\texorpdfstring{\textbf{Preferencias por la
comodificación}}{Preferencias por la comodificación}}\label{preferencias-por-la-comodificaciuxf3n}

\subsection{\texorpdfstring{\textbf{El rol de la movilidad social
intergeneracional y las creencias meritocráticas en
Chile}}{El rol de la movilidad social intergeneracional y las creencias meritocráticas en Chile}}\label{el-rol-de-la-movilidad-social-intergeneracional-y-las-creencias-meritocruxe1ticas-en-chile}

\begin{center}\rule{0.5\linewidth}{0.5pt}\end{center}

\textbf{Andreas Laffert\textsuperscript{1,2}}

\textbf{\textsuperscript{1}Instituto de Sociología, Pontificia
Universidad Católica de Chile}

\textbf{\textsuperscript{2}Centro de Estudios de Conflicto y Cohesión
Social - COES}

Coloquio de Investigación en Justicia Distributiva y Desigualdad
Socioeconómica

4 Septiembre 2025, Santiago

\section{Contexto y motivación}\label{contexto-y-motivaciuxf3n}

\subsection{Contexto y motivación}\label{contexto-y-motivaciuxf3n-1}

\subsection{Antecedentes}\label{antecedentes}

\begin{enumerate}
\def\labelenumi{\arabic{enumi})}
\tightlist
\item
  Privatización y mercantilización de bienes públicos, políticas de
  bienestar y servicios sociales (Gingrich, 2011; Streeck, 2016)
\end{enumerate}

\begin{enumerate}
\def\labelenumi{\arabic{enumi})}
\setcounter{enumi}{1}
\tightlist
\item
  En AL y Chile, modificaron la arquitectura de las instituciones del
  bienestar expandiendo lógica de mercado (Ferre, 2023; Madariaga, 2020)
\end{enumerate}

\begin{enumerate}
\def\labelenumi{\arabic{enumi})}
\setcounter{enumi}{2}
\tightlist
\item
  Este orden económico se refleja en una economía moral específica (Mau,
  2015; Svallfors, 2006)
\end{enumerate}

Preferencias por justicia de mercado (Busemeyer, 2014; Castillo et~al.,
2025; Koos \& Sachweh, 2019; Lindh, 2015)

\subsection{Preferencias por justicia de
mercado}\label{preferencias-por-justicia-de-mercado}

\begin{itemize}
\item
  Lane (1986): justicia de mercado vs.~justicia política
\item
  Creencias normativas que legitiman la idea de que el acceso a los
  servicios sociales esenciales ---como la salud, la educación o las
  pensiones--- debe determinarse según criterios basados en el mercado
  (Lindh, 2015, p. 895)
\item
  Medición: evaluar si las personas consideran justo que el acceso a
  dichos servicios dependa de los ingresos (Castillo et~al., 2025;
  Kluegel et~al., 1999; Lindh, 2015)
\end{itemize}

Preferencias por la comodificación de servicios

\subsection{Preferencias por la comodificación de
servicios}\label{preferencias-por-la-comodificaciuxf3n-de-servicios}

\subsubsection{Contextual}\label{contextual}

\begin{itemize}
\tightlist
\item
  Gasto social (Busemeyer, 2014; Busemeyer \& Iversen, 2020; Immergut \&
  Schneider, 2020)
\item
  Desigualdad económica (Koos \& Sachweh, 2019)
\item
  Nivel de privatización de servicios y regulación del mercado (Koos \&
  Sachweh, 2019; Lindh, 2015)
\end{itemize}

\subsubsection{Individual}\label{individual}

\begin{itemize}
\tightlist
\item
  Estatus socioeconómico -ingresos, educación y ocupación- (Busemeyer \&
  Iversen, 2020; Koos \& Sachweh, 2019; Lindh, 2015; Svallfors, 2007)
\item
  Percepciones sobre la desigualdad y meritocracia (Castillo et~al.,
  2025)
\item
  Conservadurismo/liberalismo económico (Lee \& Stacey, 2023)
\end{itemize}

\subsection{Preferencias por la comodificación de servicios:
Chile}\label{preferencias-por-la-comodificaciuxf3n-de-servicios-chile}

\begin{enumerate}
\def\labelenumi{\arabic{enumi}.}
\item
  Otero \& Mendoza (2024) proveen evidencia sobre:

  \begin{itemize}
  \tightlist
  \item
    Personas en posiciones sociales bajas/subordinadas muestran menor
    apoyo a la justicia de mercado que aquellas en posiciones
    altas/privilegiadas.
  \end{itemize}
\item
  Castillo et~al. (2025) encuentran que:

  \begin{itemize}
  \tightlist
  \item
    La percepción de la desigualdad económica reduce el apoyo a la
    justicia de mercado.
  \item
    Las percepciones meritocráticas (esfuerzo) aumenta el respaldo a la
    justicia de mercado.
  \end{itemize}
\end{enumerate}

\subsection{Este estudio}\label{este-estudio}

El movimiento entre posiciones sociales expone a los individuos
experiencias y mecanismos que afectan nociones de justicia (Alesina
et~al., 2018; Ares, 2020; Day \& Fiske, 2017; Gugushvili, 2017)

Meritocracia como mecanismo de sesgo de atribución(Gugushvili, 2016;
Molina et~al., 2019; Schmidt, 2011).

\section{}\label{section}

\textbf{\emph{(1) ¿Cómo la movilidad social intergeneracional afecta las
preferencias por comodificación en salud, pensiones y educación en
Chile?}}

\textbf{\emph{(2) ¿Cómo las creencias meritocráticas median esta
relación en el contexto chileno?}}

\subsection{Contexto chileno}\label{contexto-chileno}

\begin{itemize}
\tightlist
\item
  Crecimiento con elevada desigualdad (Flores et~al., 2020; Llorca-Jaña
  \& Miller, 2021)
\item
  Profunda privatización y comodificación de áreas de reproducción
  social con fuerte dependencia Estatal (Boccardo, 2020; Madariaga,
  2020)
\item
  Rigida movilidad relativa y fuerte clausura cúspide (López-Roldán \&
  Fachelli, 2021; Torche, 2014)
\item
  Conflictividad social (Núcleo de Sociología Contingente, 2020; Somma
  et~al., 2021) y legitimidad de la desigualdad (Canales Cerón et~al.,
  2021; Castillo et~al., 2025; Panes, 2020)
\end{itemize}

\subsection{Hipótesis}\label{hipuxf3tesis}

\section{Método}\label{muxe9todo}

\subsection{Estrategia de identificación
causal}\label{estrategia-de-identificaciuxf3n-causal}

\begin{figure}

\caption{\label{fig-dag}DAG movilidad social, meritocracia y
preferencias por comodificación}

\centering{

\includegraphics[width=2\linewidth,height=\textheight,keepaspectratio]{mjp-mobility_files/figure-pdf/fig-dag-1.pdf}

}

\end{figure}%

\subsection{Datos}\label{datos}

\begin{itemize}
\item
  Encuesta Longitudinal Social de Chile
  \href{https://coes.cl/elsoc/}{(ELSOC)} de COES.
\item
  Muestra analítica: olas 2016 (N = 2.927), 2018 (N = 3.748) y 2023 (N =
  2.726)
\item
  Encuesta panel representativa de zonas urbanas, localizadas en 40
  ciudades del país (92 comunas y 13 regiones)
\item
  Diseño muestral complejo: probabilístico, por conglomerados,
  multietápico y estratificado según el tamaño de las ciudades.
\item
  Población objetivo incluye a mujeres y hombres de entre 18 y 75 años
  que residen habitualmente en viviendas privadas
\end{itemize}

\subsection{Outcome}\label{outcome}

\subsection{Tratamiento}\label{tratamiento}

Siguiendo la propuesta de Breen \& Ermisch (2024) para estimar el efecto
causal de la movilidad social:

\begin{enumerate}
\def\labelenumi{\arabic{enumi}.}
\item
  \textbf{Asignación de clase}: En base a la ocupación (ISCO-08) del
  padre/madre y del entrevistado se crean estratos de origen y destino
  (bajo, medio y alto) según terciles de la distribución del índice de
  estatus ocupacional (ISEI)
\item
  \textbf{Propensity-score estimation}: Se estima la probabilidad de
  movilidad social mediante un modelo logístico multinomial en relación
  con las trayectorias inmóviles (categ. ref.) para cada estrato.
  Creación de IPW o ``clones''
\end{enumerate}

\subsection{Tratamiento}\label{tratamiento-1}

\begin{table}

\caption{\label{tbl-ocup}Movilidad ocupacional por grupos ocupacionales}

\centering{

}

\end{table}%

\begin{itemize}
\tightlist
\item
  \emph{N} (id único) = 4.447. Porcentajes en azul corresponde a las
  filas y en verde a las columnas.
\end{itemize}

\subsection{Tratamiento}\label{tratamiento-2}

\subsection{Mediador}\label{mediador}

\begin{itemize}
\item
  \textbf{Meritocracia}: se mide a partir de dos componentes, a saber,
  el esfuerzo y el talento (Young, 1958)
\item
  Medición:

  \begin{itemize}
  \tightlist
  \item
    Ítem esfuerzo: ``En Chile, se recompensa a las personas por su
    esfuerzo''
  \item
    Ítem talento: ``En Chile, se recompensa a las personas por su
    inteligencia y habilidades''
  \item
    Ambos ítems se responden en una escala Likert de 1 (totalmente en
    desacuerdo) a 5 (totalmente de acuerdo)
  \end{itemize}
\end{itemize}

\subsection{Estimand}\label{estimand}

\begin{itemize}
\tightlist
\item
  Efecto causal promedio de pasar de una clase de origen \emph{j} a una
  clase de destino \emph{k} comparando a las personas que realmente
  alcanzan \emph{k} con su resultado contrafactual si se hubieran
  trasladado a un destino alternativo \emph{k′} (Breen \& Ermisch, 2024)
  (Ecuación~\ref{eq-att}).
\end{itemize}

\begin{equation}\phantomsection\label{eq-att}{ 
ATT_{jkk′} =E[Y(D=k|O=j, D=k)]−E[Y(D=k′|O=j, D=k)]
}\end{equation}

\begin{itemize}
\tightlist
\item
  El estimador resultante es el efecto promedio del tratamiento sobre
  los tratados (ATT)
\end{itemize}

\subsection{Estimando}\label{estimando}

\begin{itemize}
\item
  Para estimar el efecto causal de la movilidad social, utilizo modelos
  de regresión lineal con ponderaciones de probabilidad inversa (IPW)
  para la movilidad condicional a la clase de origen (Breen \& Ermisch,
  2024)
\item
  Para mediación ocupo el estimador Average Causal Mediation Effect
  (Imai et~al., 2010)
\end{itemize}

\section{Resultados}\label{resultados}

\subsection{Modelos}\label{modelos}

\begin{figure}

\caption{\label{fig-mh}Estimaciones de efectos de mobilidad en
preferencia por comodificación en salud}

\centering{

\includegraphics[width=1.8\linewidth,height=\textheight,keepaspectratio]{mjp-mobility_files/figure-pdf/fig-mh-1.pdf}

}

\end{figure}%

\subsection{Modelos}\label{modelos-1}

\begin{figure}

\caption{\label{fig-me}Estimaciones de efectos de mobilidad en
preferencia por comodificación en educación}

\centering{

\includegraphics[width=1.8\linewidth,height=\textheight,keepaspectratio]{mjp-mobility_files/figure-pdf/fig-me-1.pdf}

}

\end{figure}%

\subsection{Modelos}\label{modelos-2}

\begin{figure}

\caption{\label{fig-mp}Estimaciones de efectos de mobilidad en
preferencia por comodificación en pensiones}

\centering{

\includegraphics[width=1.8\linewidth,height=\textheight,keepaspectratio]{mjp-mobility_files/figure-pdf/fig-mp-1.pdf}

}

\end{figure}%

\subsection{Mediación}\label{mediaciuxf3n}

\begin{figure}

\caption{\label{fig-acme}ACME para meritocracia y preferencias por
comodificación en pensiones}

\centering{

\includegraphics[width=1.8\linewidth,height=\textheight,keepaspectratio]{mjp-mobility_files/figure-pdf/fig-acme-1.pdf}

}

\end{figure}%

\section{Discusión y proyecciones}\label{discusiuxf3n-y-proyecciones}

\section{Discusión y proyecciones}\label{discusiuxf3n-y-proyecciones-1}

\begin{enumerate}
\def\labelenumi{\arabic{enumi}.}
\tightlist
\item
  \textbf{Movilidad social}: ascendente (Middle→High) \emph{incrementa}
  la preferencia por comodificación en pensiones (dominio), no en otros.
\item
  \textbf{Contexto chileno}: profunda comodificación y privatización de
  servicios escenciales → trabajador-empresario
\item
  \textbf{Mediación}: rol de la meritocracia como mediador → ACME
  supuestos muy fuertes = alternativas (?)
\end{enumerate}

\section{Gracias por su atención!}\label{gracias-por-su-atenciuxf3n}

\begin{itemize}
\tightlist
\item
  \textbf{Github del proyecto:}
  \url{https://github.com/Andreas-Lafferte/mobility-market-justice}
\end{itemize}

\subsection*{Referencias}\label{referencias}
\addcontentsline{toc}{subsection}{Referencias}

\phantomsection\label{refs}
\begin{CSLReferences}{1}{0}
\bibitem[\citeproctext]{ref-alesina_intergenerational_2018}
Alesina, A., Stantcheva, S., \& Teso, E. (2018). Intergenerational
{Mobility} and {Preferences} for {Redistribution}. \emph{American
Economic Review}, \emph{108}(2), 521-554.
\url{https://doi.org/10.1257/aer.20162015}

\bibitem[\citeproctext]{ref-ares_changing_2020}
Ares, M. (2020). Changing Classes, Changing Preferences: How Social
Class Mobility Affects Economic Preferences. \emph{West European
Politics}, \emph{43}(6), 1211-1237.
\url{https://doi.org/10.1080/01402382.2019.1644575}

\bibitem[\citeproctext]{ref-boccardo_30_2020}
Boccardo, G. (2020). \emph{30 A{ñ}os de Privatizaciones En {Chile}: Lo
Que La Pandemia Revel{ó}} (Nodo XXI). Santiago.

\bibitem[\citeproctext]{ref-breen_effects_2024}
Breen, R., \& Ermisch, J. (2024). The {Effects} of {Social Mobility}.
\emph{Sociological Science}, \emph{11}, 467-488.
\url{https://doi.org/10.15195/v11.a17}

\bibitem[\citeproctext]{ref-busemeyer_skills_2014}
Busemeyer, M. (2014). \emph{Skills and {Inequality}: {Partisan Politics}
and the {Political Economy} of {Education Reforms} in {Western Welfare
States}}. Cambridge University Press.

\bibitem[\citeproctext]{ref-busemeyer_welfare_2020}
Busemeyer, M., \& Iversen, T. (2020). The {Welfare State} with {Private
Alternatives}: {The Transformation} of {Popular Support} for {Social
Insurance}. \emph{The Journal of Politics}, \emph{82}(2), 671-686.
\url{https://doi.org/10.1086/706980}

\bibitem[\citeproctext]{ref-canalesceron_sujeto_2021}
Canales Cerón, M., Orellana Calderón, V. S., \& Guajardo Mañán, F.
(2021). Sujeto y Cotidiano En La Era Neoliberal: El Caso de La
Educaci{ó}n Chilena. \emph{Revista Mexicana de Ciencias Pol{í}ticas y
Sociales}, \emph{67}(244).
\url{https://doi.org/10.22201/fcpys.2448492xe.2022.244.70386}

\bibitem[\citeproctext]{ref-castillo_perceptions_2025}
Castillo, J. C., Laffert, A., Carrasco, K., \& Iturra, J. (2025).
Perceptions of {Inequality} and {Meritocracy}: {Their Interplay} in
{Shaping Preferences} for {Market Justice} in {Chile} (2016-2023).
\emph{Under Review at Frontiers in Sociology}.

\bibitem[\citeproctext]{ref-day_movin_2017}
Day, M. V., \& Fiske, S. T. (2017). Movin' on {Up}? {How Perceptions} of
{Social Mobility Affect Our Willingness} to {Defend} the {System}.
\emph{Social Psychological and Personality Science}, \emph{8}(3),
267-274. \url{https://doi.org/10.1177/1948550616678454}

\bibitem[\citeproctext]{ref-ferre_welfare_2023}
Ferre, J. C. (2023). Welfare Regimes in Twenty-First-Century {Latin
America}. \emph{Journal of International and Comparative Social Policy},
\emph{39}(2), 101-127. \url{https://doi.org/10.1017/ics.2023.16}

\bibitem[\citeproctext]{ref-flores_top_2020}
Flores, I., Sanhueza, C., Atria, J., \& Mayer, R. (2020). Top {Incomes}
in {Chile}: {A Historical Perspective} on {Income Inequality},
1964--2017. \emph{Review of Income and Wealth}, \emph{66}(4), 850-874.
\url{https://doi.org/10.1111/roiw.12441}

\bibitem[\citeproctext]{ref-gingrich_making_2011}
Gingrich, J. R. (2011). \emph{Making {Markets} in the {Welfare State}:
{The Politics} of {Varying Market Reforms}} (1.ª ed.). Cambridge
University Press. \url{https://doi.org/10.1017/CBO9780511791529}

\bibitem[\citeproctext]{ref-gugushvili_intergenerational_2016c}
Gugushvili, A. (2016). Intergenerational Objective and Subjective
Mobility and Attitudes towards Income Differences: Evidence from
Transition Societies. \emph{Journal of International and Comparative
Social Policy}, \emph{32}(3), 199-219.
\url{https://doi.org/10.1080/21699763.2016.1206482}

\bibitem[\citeproctext]{ref-gugushvili_subjective_2017}
Gugushvili, A. (2017). Subjective {Intergenerational Mobility} and
{Support} for {Welfare State Programmes}.

\bibitem[\citeproctext]{ref-imai_identification_2010}
Imai, K., Keele, L., \& Yamamoto, T. (2010). Identification, {Inference}
and {Sensitivity Analysis} for {Causal Mediation Effects}.
\emph{Statistical Science}, \emph{25}(1).
\url{https://doi.org/10.1214/10-STS321}

\bibitem[\citeproctext]{ref-immergut_it_2020}
Immergut, E. M., \& Schneider, S. M. (2020). Is It Unfair for the
Affluent to Be Able to Purchase {«Better»} Healthcare? {Existential}
Standards and Institutional Norms in Healthcare Attitudes across 28
Countries. \emph{Social Science \& Medicine}, \emph{267}, 113146.
\url{https://doi.org/10.1016/j.socscimed.2020.113146}

\bibitem[\citeproctext]{ref-kluegel_legitimation_1999}
Kluegel, J. R., Mason, D. S., \& Wegener, B. (1999). The {Legitimation}
of {Capitalism} in the {Postcommunist Transition}: {Public Opinion}
about {Market Justice}, 1991-1996. \emph{European Sociological Review},
\emph{15}(3), 251-283. Recuperado de
\url{https://www.jstor.org/stable/522731}

\bibitem[\citeproctext]{ref-koos_moral_2019}
Koos, S., \& Sachweh, P. (2019). The Moral Economies of Market
Societies: Popular Attitudes towards Market Competition, Redistribution
and Reciprocity in Comparative Perspective. \emph{Socio-Economic
Review}, \emph{17}(4), 793-821. \url{https://doi.org/10.1093/ser/mwx045}

\bibitem[\citeproctext]{ref-lane_market_1986}
Lane, R. E. (1986). Market {Justice}, {Political Justice}.
\emph{American Political Science Review}, \emph{80}(2), 383-402.
\url{https://doi.org/10.2307/1958264}

\bibitem[\citeproctext]{ref-lee_fairness_2023}
Lee, J.-S., \& Stacey, M. (2023). Fairness Perceptions of Educational
Inequality: The Effects of Self-Interest and Neoliberal Orientations.
\emph{The Australian Educational Researcher}.
\url{https://doi.org/10.1007/s13384-023-00636-6}

\bibitem[\citeproctext]{ref-lindh_public_2015}
Lindh, A. (2015). Public {Opinion} against {Markets}? {Attitudes}
towards {Market Distribution} of {Social Services} -- {A Comparison} of
17 {Countries}. \emph{Social Policy \& Administration}, \emph{49}(7),
887-910. \url{https://doi.org/10.1111/spol.12105}

\bibitem[\citeproctext]{ref-llorca-jana_historia_2021}
Llorca-Jaña, M., \& Miller, R. M. D. (2021). \emph{{Historia econ{ó}mica
de Chile desde la independencia}}. Santiago de Chile: RIL editores.

\bibitem[\citeproctext]{ref-lopez-roldan_comparative_2021}
López-Roldán, P., \& Fachelli, S. (Eds.). (2021). \emph{Towards a
{Comparative Analysis} of {Social Inequalities} between {Europe} and
{Latin America}}. Cham: Springer International Publishing.
\url{https://doi.org/10.1007/978-3-030-48442-2}

\bibitem[\citeproctext]{ref-madariaga_three_2020}
Madariaga, A. (2020). The Three Pillars of Neoliberalism: {Chile}'s
Economic Policy Trajectory in Comparative Perspective.
\emph{Contemporary Politics}, \emph{26}(3), 308-329.
\url{https://doi.org/10.1080/13569775.2020.1735021}

\bibitem[\citeproctext]{ref-mau_inequality_2015}
Mau, S. (2015). \emph{Inequality, {Marketization} and the {Majority
Class}: {Why Did} the {European Middle Classes Accept Neo-Liberalism}?}
Houndmills: Palgrave Macmillan.

\bibitem[\citeproctext]{ref-molina_its_2019}
Molina, M. D., Bucca, M., \& Macy, M. W. (2019). It's Not Just How the
Game Is Played, It's Whether You Win or Lose. \emph{SCIENCE ADVANCES}.

\bibitem[\citeproctext]{ref-nucleodesociologiacontingente_informe_2020}
Núcleo de Sociología Contingente, {[}NUDESOC{]}. (2020). \emph{Informe
de Resultados Oficial {Encuesta Zona Cero}}. Santiago de Chile.

\bibitem[\citeproctext]{ref-otero_power_2024}
Otero, G., \& Mendoza, M. (2024). The {Power} of {Diversity}: {Class},
{Networks} and {Attitudes Towards Inequality}. \emph{Sociology},
\emph{58}(4), 851-876. \url{https://doi.org/10.1177/00380385231217625}

\bibitem[\citeproctext]{ref-panes_criticas_2020}
Panes, D. (2020). \emph{Cr{í}ticas y Experiencias Obreras En Torno al
Sistema de {AFP}'s En {Chile} (1981-2020)} (Tesis doctoral). Universidad
de Chile, Santiago.

\bibitem[\citeproctext]{ref-schmidt_experience_2011}
Schmidt, A. W. (2011). The Experience of Social Mobility and the
Formation of Attitudes toward Redistribution. En.

\bibitem[\citeproctext]{ref-somma_no_2021}
Somma, N. M., Bargsted, M., Disi Pavlic, R., \& Medel, R. M. (2021). No
Water in the Oasis: The {Chilean Spring} of 2019--2020. \emph{Social
Movement Studies}, \emph{20}(4), 495-502.
\url{https://doi.org/10.1080/14742837.2020.1727737}

\bibitem[\citeproctext]{ref-streeck_how_2016}
Streeck, W. (2016). \emph{How Will Capitalism End? Essays on a Failing
System}. London: Verso.

\bibitem[\citeproctext]{ref-svallforsMoralEconomyClass2006a}
Svallfors, S. (2006). \emph{The {Moral Economy} of {Class}: {Class} and
{Attitudes} in {Comparative Perspective}}. Stanford University Press.

\bibitem[\citeproctext]{ref-svallfors_political_2007}
Svallfors, S. (Ed.). (2007). \emph{The {Political Sociology} of the
{Welfare State}: {Institutions}, {Social Cleavages}, and {Orientations}}
(1.ª ed.). Stanford University Press.
\url{https://doi.org/10.2307/j.ctvr0qv0q}

\bibitem[\citeproctext]{ref-torche_intergenerational_2014}
Torche, F. (2014). Intergenerational {Mobility} and {Inequality}: {The
Latin American Case}. \emph{Annual Review of Sociology}, \emph{40}(1),
619-642. \url{https://doi.org/10.1146/annurev-soc-071811-145521}

\bibitem[\citeproctext]{ref-young_rise_1958}
Young, M. (1958). \emph{The Rise of the Meritocracy}. New Brunswick,
N.J., U.S.A: Transaction Publishers.

\end{CSLReferences}




\end{document}
