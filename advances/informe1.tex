% Options for packages loaded elsewhere
\PassOptionsToPackage{unicode}{hyperref}
\PassOptionsToPackage{hyphens}{url}
\PassOptionsToPackage{dvipsnames,svgnames,x11names}{xcolor}
%
\documentclass[
  12pt,
]{article}

\usepackage{amsmath,amssymb}
\usepackage{setspace}
\usepackage{iftex}
\ifPDFTeX
  \usepackage[T1]{fontenc}
  \usepackage[utf8]{inputenc}
  \usepackage{textcomp} % provide euro and other symbols
\else % if luatex or xetex
  \usepackage{unicode-math}
  \defaultfontfeatures{Scale=MatchLowercase}
  \defaultfontfeatures[\rmfamily]{Ligatures=TeX,Scale=1}
\fi
\usepackage{lmodern}
\ifPDFTeX\else  
    % xetex/luatex font selection
  \setmainfont[]{Times New Roman}
\fi
% Use upquote if available, for straight quotes in verbatim environments
\IfFileExists{upquote.sty}{\usepackage{upquote}}{}
\IfFileExists{microtype.sty}{% use microtype if available
  \usepackage[]{microtype}
  \UseMicrotypeSet[protrusion]{basicmath} % disable protrusion for tt fonts
}{}
\makeatletter
\@ifundefined{KOMAClassName}{% if non-KOMA class
  \IfFileExists{parskip.sty}{%
    \usepackage{parskip}
  }{% else
    \setlength{\parindent}{0pt}
    \setlength{\parskip}{6pt plus 2pt minus 1pt}}
}{% if KOMA class
  \KOMAoptions{parskip=half}}
\makeatother
\usepackage{xcolor}
\usepackage[margin=2cm]{geometry}
\setlength{\emergencystretch}{3em} % prevent overfull lines
\setcounter{secnumdepth}{5}
% Make \paragraph and \subparagraph free-standing
\ifx\paragraph\undefined\else
  \let\oldparagraph\paragraph
  \renewcommand{\paragraph}[1]{\oldparagraph{#1}\mbox{}}
\fi
\ifx\subparagraph\undefined\else
  \let\oldsubparagraph\subparagraph
  \renewcommand{\subparagraph}[1]{\oldsubparagraph{#1}\mbox{}}
\fi


\providecommand{\tightlist}{%
  \setlength{\itemsep}{0pt}\setlength{\parskip}{0pt}}\usepackage{longtable,booktabs,array}
\usepackage{calc} % for calculating minipage widths
% Correct order of tables after \paragraph or \subparagraph
\usepackage{etoolbox}
\makeatletter
\patchcmd\longtable{\par}{\if@noskipsec\mbox{}\fi\par}{}{}
\makeatother
% Allow footnotes in longtable head/foot
\IfFileExists{footnotehyper.sty}{\usepackage{footnotehyper}}{\usepackage{footnote}}
\makesavenoteenv{longtable}
\usepackage{graphicx}
\makeatletter
\def\maxwidth{\ifdim\Gin@nat@width>\linewidth\linewidth\else\Gin@nat@width\fi}
\def\maxheight{\ifdim\Gin@nat@height>\textheight\textheight\else\Gin@nat@height\fi}
\makeatother
% Scale images if necessary, so that they will not overflow the page
% margins by default, and it is still possible to overwrite the defaults
% using explicit options in \includegraphics[width, height, ...]{}
\setkeys{Gin}{width=\maxwidth,height=\maxheight,keepaspectratio}
% Set default figure placement to htbp
\makeatletter
\def\fps@figure{htbp}
\makeatother
% definitions for citeproc citations
\NewDocumentCommand\citeproctext{}{}
\NewDocumentCommand\citeproc{mm}{%
  \begingroup\def\citeproctext{#2}\cite{#1}\endgroup}
\makeatletter
 % allow citations to break across lines
 \let\@cite@ofmt\@firstofone
 % avoid brackets around text for \cite:
 \def\@biblabel#1{}
 \def\@cite#1#2{{#1\if@tempswa , #2\fi}}
\makeatother
\newlength{\cslhangindent}
\setlength{\cslhangindent}{1.5em}
\newlength{\csllabelwidth}
\setlength{\csllabelwidth}{3em}
\newenvironment{CSLReferences}[2] % #1 hanging-indent, #2 entry-spacing
 {\begin{list}{}{%
  \setlength{\itemindent}{0pt}
  \setlength{\leftmargin}{0pt}
  \setlength{\parsep}{0pt}
  % turn on hanging indent if param 1 is 1
  \ifodd #1
   \setlength{\leftmargin}{\cslhangindent}
   \setlength{\itemindent}{-1\cslhangindent}
  \fi
  % set entry spacing
  \setlength{\itemsep}{#2\baselineskip}}}
 {\end{list}}
\usepackage{calc}
\newcommand{\CSLBlock}[1]{\hfill\break\parbox[t]{\linewidth}{\strut\ignorespaces#1\strut}}
\newcommand{\CSLLeftMargin}[1]{\parbox[t]{\csllabelwidth}{\strut#1\strut}}
\newcommand{\CSLRightInline}[1]{\parbox[t]{\linewidth - \csllabelwidth}{\strut#1\strut}}
\newcommand{\CSLIndent}[1]{\hspace{\cslhangindent}#1}

\usepackage[noblocks]{authblk}
\renewcommand*{\Authsep}{, }
\renewcommand*{\Authand}{, }
\renewcommand*{\Authands}{, }
\renewcommand\Affilfont{\small}
\makeatletter
\@ifpackageloaded{caption}{}{\usepackage{caption}}
\AtBeginDocument{%
\ifdefined\contentsname
  \renewcommand*\contentsname{Table of contents}
\else
  \newcommand\contentsname{Table of contents}
\fi
\ifdefined\listfigurename
  \renewcommand*\listfigurename{List of Figures}
\else
  \newcommand\listfigurename{List of Figures}
\fi
\ifdefined\listtablename
  \renewcommand*\listtablename{List of Tables}
\else
  \newcommand\listtablename{List of Tables}
\fi
\ifdefined\figurename
  \renewcommand*\figurename{Figure}
\else
  \newcommand\figurename{Figure}
\fi
\ifdefined\tablename
  \renewcommand*\tablename{Table}
\else
  \newcommand\tablename{Table}
\fi
}
\@ifpackageloaded{float}{}{\usepackage{float}}
\floatstyle{ruled}
\@ifundefined{c@chapter}{\newfloat{codelisting}{h}{lop}}{\newfloat{codelisting}{h}{lop}[chapter]}
\floatname{codelisting}{Listing}
\newcommand*\listoflistings{\listof{codelisting}{List of Listings}}
\makeatother
\makeatletter
\makeatother
\makeatletter
\@ifpackageloaded{caption}{}{\usepackage{caption}}
\@ifpackageloaded{subcaption}{}{\usepackage{subcaption}}
\makeatother
\ifLuaTeX
  \usepackage{selnolig}  % disable illegal ligatures
\fi
\usepackage{bookmark}

\IfFileExists{xurl.sty}{\usepackage{xurl}}{} % add URL line breaks if available
\urlstyle{same} % disable monospaced font for URLs
\hypersetup{
  pdftitle={Preferences for market justice: the role of intergenerational social mobility and beliefs in meritocracy in Chile},
  pdfauthor={Andreas Laffert},
  colorlinks=true,
  linkcolor={blue},
  filecolor={Maroon},
  citecolor={Blue},
  urlcolor={Blue},
  pdfcreator={LaTeX via pandoc}}

\title{Preferences for market justice: the role of intergenerational
social mobility and beliefs in meritocracy in Chile}


  \author{Andreas Laffert}
            \affil{%
                  Instituto de Sociología, Pontificia Universidad
                  Católica de Chile
              }
      
\date{}
\begin{document}
\maketitle

\setstretch{1.15}
\section{Introduction}\label{introduction}

Since the early 1980s, many countries have experienced a widespread
retreatment of universal welfare programs and have adopted a trend
toward privatization and commoditization of various public goods,
welfare policies and social services
(\citeproc{ref-gingrich_making_2011}{Gingrich, 2011};
\citeproc{ref-salamon_marketization_1993}{Salamon, 1993};
\citeproc{ref-streeck_how_2016}{Streeck, 2016}). Neoliberal reforms
reshaped the architecture of welfare-state institutions across Western
democracies and Latin America, extending market criteria into areas of
social reproduction that were traditionally governed by the state. This
shift diminished the role of public provision of core social services
(\citeproc{ref-arrizabalo_milagro_1995}{Arrizabalo, 1995};
\citeproc{ref-busemeyer_welfare_2020}{Busemeyer \& Iversen, 2020}).
Echoing Polanyi's (\citeproc{ref-polanyi_great_1975}{1975}) insight that
markets constitute a distinct moral order, the institutional diffusion
of market rules has fostered a corresponding moral economy---a
constellation of norms and values about fair allocation embedded in, and
reinforced by, social institutions (\citeproc{ref-mau_moral_2014}{Mau,
2014}; \citeproc{ref-svallforsMoralEconomyClass2006a}{Svallfors, 2006}).
Within this moral economy, a growing body of research has been addressed
the extent to which, and the mechanisms through which, citizens deem it
just that access to social services such as health care, pensions, and
education be determined by market-based criteria, a phenomenon labelled
market justice preferences
(\citeproc{ref-busemeyer_skills_2014}{Busemeyer, 2014};
\citeproc{ref-castillo_socialization_2024}{Castillo et al., 2024};
\citeproc{ref-immergut_it_2020}{Immergut \& Schneider, 2020};
\citeproc{ref-koos_moral_2019}{Koos \& Sachweh, 2019};
\citeproc{ref-lindh_public_2015}{Lindh, 2015};
\citeproc{ref-lindh_bringing_2023}{Lindh \& McCall, 2023}).
Understanding these preferences is crucial because they can legitimize
economic inequality by framing it as the natural outcome of fair
competition, driven by individual responsibility and limited state
intervention (\citeproc{ref-svallfors_political_2007}{Svallfors, 2007}).

Existing literature consistently demonstrates that market justice
preferences are shaped by both the economic and institutional contexts
of countries and individuals' positions within social stratification.
Recent studies indicate that in countries with greater public provision
of social services or more extensive welfare states, market justice
preferences tend to be lower
(\citeproc{ref-busemeyer_skills_2014}{Busemeyer, 2014};
\citeproc{ref-immergut_it_2020}{Immergut \& Schneider, 2020}), whereas
in contexts with higher levels of privatization, preferences for
market-based criteria are more pronounced
(\citeproc{ref-castillo_socialization_2024}{Castillo et al., 2024};
\citeproc{ref-lindh_public_2015}{Lindh, 2015}). In such contexts, market
justice preferences tend to rise as individuals ``ascend'' the social
structure, with those in more privileged positions in terms of class,
education, and income holding stronger preferences for market-based
solutions compared to those in more disadvantaged or at-risk positions
(\citeproc{ref-busemeyer_skills_2014}{Busemeyer, 2014};
\citeproc{ref-immergut_it_2020}{Immergut \& Schneider, 2020};
\citeproc{ref-kluegel_legitimation_1999}{Kluegel et al., 1999};
\citeproc{ref-lindh_public_2015}{Lindh, 2015};
\citeproc{ref-svallfors_political_2007}{Svallfors, 2007}). Empirically,
it has been consistently demonstrated that those in socioeconomically
advantaged positions endorse the idea that those with higher incomes
should be able to pay more for better social services in the domains of
education (\citeproc{ref-lee_fairness_2023}{Lee \& Stacey, 2023}),
healthcare (\citeproc{ref-immergut_it_2020}{Immergut \& Schneider,
2020}; \citeproc{ref-vondemknesebeck_are_2016}{Von Dem Knesebeck et al.,
2016}) and old age pensions
(\citeproc{ref-castillo_socialization_2024}{Castillo et al., 2024};
\citeproc{ref-otero_power_2024}{Otero \& Mendoza, 2024}).

Although it is clear that one's social position influences market
preferences, the question of how upward or downward mobility within the
social structure affects these preferences remains unanswered. This is
far from trivial, as many individuals in Western democracies and Latin
America have experienced varying degrees of social mobility
(\citeproc{ref-breen_social_2004}{Breen, 2004};
\citeproc{ref-lopez-roldan_comparative_2021}{López-Roldán \& Fachelli,
2021}). The social origins and destinations of individuals exert
different effects on a range of attitudes
(\citeproc{ref-gugushvili_trends_2014}{Gugushvili, 2014}), while
movement between these positions exposes individuals to distinct
processes and mechanisms that can profoundly influence their
perspectives, shaping how they perceive the world and evaluate what they
consider fair (\citeproc{ref-gugushvili_trends_2014}{Gugushvili,
2014})). Some of the current research on attitudes toward inequality has
concluded that the type of social mobility---upward or downward---that
individuals experience, whether measured objectively or subjectively,
has differentiated effects on support for redistribution
(\citeproc{ref-alesina_intergenerational_2018}{Alesina et al., 2018};
\citeproc{ref-benabou_social_2001}{Benabou \& Ok, 2001};
\citeproc{ref-jaime-castillo_social_2019}{Jaime-Castillo \&
Marqués-Perales, 2019}; \citeproc{ref-schmidt_experience_2011}{Schmidt,
2011}), system legitimacy, and meritocratic beliefs
(\citeproc{ref-day_movin_2017}{Day \& Fiske, 2017};
\citeproc{ref-mijs_belief_2022}{Mijs et al., 2022}), as well as on
tolerance for economic inequality
(\citeproc{ref-gugushvili_intergenerational_2016c}{Gugushvili, 2016a};
\citeproc{ref-shariff_income_2016}{Shariff et al., 2016}), attributions
about poverty and wealth (\citeproc{ref-bucca_merit_2016}{Bucca, 2016};
\citeproc{ref-gugushvili_intergenerational_2016}{Gugushvili, 2016b}),
and support for welfare policies and market economies
(\citeproc{ref-gugushvili_subjective_2017}{Gugushvili, 2017};
\citeproc{ref-wen_does_2021}{Wen \& Witteveen, 2021}). Thus, building on
these contributions, expanding the analysis of the effects of social
mobility on market justice preferences could contribute to this emerging
literature and shed light on the mechanisms behind the justification of
inequalities in access to social services.

Market justice preferences are shaped not only by objective
socioeconomic factors but also by perceptions and beliefs about
meritocracy. Meritocracy asserts that inequality is an inherent feature
of societies but can be legitimized through principles such as effort
and talent (\citeproc{ref-davis_principles_2001}{Davis \& Moore, 2001};
\citeproc{ref-young_rise_1958}{Young, 1958}). Previous studies have
shown that individuals with stronger meritocratic beliefs tend to
perceive less inequality, attributing economic differences to individual
achievements (\citeproc{ref-mijs_paradox_2019}{Mijs, 2019};
\citeproc{ref-wilson_role_2003}{Wilson, 2003}). Additionally, these
beliefs justify greater inequality, as they are associated with
attitudes that uphold the legitimization of social differences
(\citeproc{ref-batruch_belief_2023}{Batruch et al., 2023}). In unequal
societies where the distribution of goods and services is predominantly
governed by market logic, such beliefs can play a critical role in the
acceptance of social inequalities. Recent research by Castillo et al.
(\citeproc{ref-castillo_socialization_2024}{2024}) demonstrates that
students in Chile who believe that effort and talent are rewarded in
their country hold stronger preferences for market justice in access to
healthcare, pensions, and education. By incorporating perceptions of
meritocracy into this study, we can better understand how preferences
toward market justice are shaped not only by an individual's structural
social position but also by beliefs about the justification of these
inequalities. Moreover, these beliefs are often solidified early in
life, reinforced by institutions that promote values such as effort and
individual skills as means to upward social mobility
(\citeproc{ref-castillo_socialization_2024}{Castillo et al., 2024};
\citeproc{ref-reynolds_perceptions_2014}{Reynolds \& Xian, 2014}).

Beyond the isolated effects of social mobility and beliefs in
meritocracy on market justice preferences, these factors interact to
shape these preferences in a complex way. One proposed mechanism in the
literature regarding the process of social mobility that leads to
differences in distributive justice preferences is the psychological
mechanism of self-serving bias in causal attribution
(\citeproc{ref-gugushvili_intergenerational_2016c}{Gugushvili, 2016a};
\citeproc{ref-schmidt_experience_2011}{Schmidt, 2011}). Self-serving
bias suggests that people attribute failures, such as downward mobility,
to factors beyond their control, while explaining successes, such as
upward mobility, by pointing to their own merits and effort
(\citeproc{ref-miller_selfserving_1975}{Miller \& Ross, 1975}).
Individuals who experience upward mobility tend to view their position
as a result of their own effort and skills, making them more likely to
support the idea that individuals are responsible for their own success
or failure. Research shows that upward mobility is associated with lower
redistribution preferences
(\citeproc{ref-gugushvili_intergenerational_2016c}{Gugushvili, 2016a};
\citeproc{ref-schmidt_experience_2011}{Schmidt, 2011}) and stronger
legitimacy of income inequality
(\citeproc{ref-shariff_income_2016}{Shariff et al., 2016}). In contrast,
individuals who experience downward mobility tend to attribute their
failure to external factors, such as inequality, and are more supportive
of redistribution and the idea that individuals should not be blamed for
their economic failure
(\citeproc{ref-gugushvili_intergenerational_2016c}{Gugushvili, 2016a};
\citeproc{ref-schmidt_experience_2011}{Schmidt, 2011}). Taken together,
I suggest that meritocratic beliefs may mirror this self-serving bias
mechanism, as they reinforce the tendency to justify one's social status
as the result of personal merits, which is closely related to the
attribution bias.

Focusing in Chile, this article aims to achieve two main objectives:
first, to analyze the extent to which intergenerational social mobility
affects market justice preferences on healthcare, pensions and
education, and second, to examine how meritocratic beliefs may moderate
this relationship. The central argument is that upward social mobility
will have a positive effect on market justice preferences, while
downward mobility will have a negative effect. Additionally,
meritocratic beliefs will moderate this relationship by reflecting a
self-serving bias mechanism of causal attribution.

The Chilean case is particularly intriguing for studying market justice
preferences. Despite being one of the most prosperous nations in Latin
America, it has one of the highest levels of economic inequality in the
region. The poorest 50\% of Chileans capture only 10\% of the total
income and hold negative wealth, while the richest 1\% receives nearly
27\% of the income and controls 49.6\% of the wealth
(\citeproc{ref-chancel_world_2022}{Chancel et al., 2022}). This economic
inequality exists alongside notable social mobility, with significant
upward mobility in recent decades
(\citeproc{ref-espinoza_estratificacion_2013}{Espinoza et al., 2013};
\citeproc{ref-torche_intergenerational_2014}{Torche, 2014}). Chile's
social policy regime is characterized by a welfare model that relies
heavily on private provision in healthcare, pensions, and education
systems, often segmented by individuals' ability to pay and heavily
dependent on state subsidies (\citeproc{ref-boccardo_30_2020}{Boccardo,
2020}).

Against this backdrop, the main research question is: To what extent
does intergenerational social mobility affect market justice
preferences, and how do meritocratic beliefs explain this relationship
in Chile? This study aims to contribute to the literature by providing
evidence from a Latin American developing country, highlighting how
changes in individuals' socioeconomic position and meritocratic beliefs
influence market justice preferences in critical areas like healthcare,
pensions, and education.

\section{Data, variables and methods}\label{data-variables-and-methods}

\subsection{Data}\label{data}

This study draws on data from the Chilean Longitudinal Social Survey
(ELSOC) of the Center for Social Conflict and Cohesion Studies (COES).
The survey is an annual panel collected from 2016 to 2023, comprising
two independent samples (original and refreshment), featuring permanent
and rotating questionnaire modules. It evaluates how individuals think,
feel, and behave regarding conflict and social cohesion in Chile. The
sampling design is complex, probabilistic, clustered, multistage, and
stratified by city size, targeting men and women aged 18--75 who
habitually reside in private dwellings in urban areas spanning 40 cities
(92 municipalities, 13 regions). This analysis focuses on 2023, the most
recent wave, which includes 1,737 respondents in the original sample and
989 in the refreshment sample. Further details on sampling, attrition,
and weighting can be found at https://coes.cl/encuesta-panel/, and the
dataset is publicly available at
https://dataverse.harvard.edu/dataverse/elsoc.

\subsection{Variables}\label{variables}

\emph{Outcome variables}

\textbf{Market justice preferences}: The outcome variables in this study
are market justice preferences. This construct is operationalized
through three variables that address the degree of justification
regarding whether access to social services in healthcare, pensions, and
education should be income conditional. Specifically, the justification
of inequality in healthcare is assessed by the question: ``Is it fair in
Chile that people with higher incomes can access better healthcare than
people with lower incomes?'' The same question is asked for pensions and
education. In all cases, respondents indicate their level of agreement
on a five-point Likert scale ranging from 1 (``strongly disagree'') to 5
(``strongly agree''). Although these items theoretically reflect the
same underlying concept of market justice
(\citeproc{ref-castillo_socialization_2024}{Castillo et al., 2024};
\citeproc{ref-lindh_public_2015}{Lindh, 2015}), they are analyzed
separately because healthcare, pensions, and education may prompt
distinct response patterns
(\citeproc{ref-busemeyer_skills_2014}{Busemeyer, 2014};
\citeproc{ref-immergut_it_2020}{Immergut \& Schneider, 2020};
\citeproc{ref-lee_fairness_2023}{Lee \& Stacey, 2023}).

\emph{Independent variables}

\textbf{Social mobility}: Intergenerational social mobility is treated
as an exposure indicating whether respondents occupy a different class
position from their fathers. Following Breen and Ermisch's
(\citeproc{ref-breen_effects_2024}{2024}) framework for estimating
causal effects of mobility, the variable is constructed in two stages.

\begin{enumerate}
\def\labelenumi{\arabic{enumi}.}
\item
  Class assignment. Using the International Socio-Economic Index of
  Occupational Status (ISEI) derived from three-digit ISCO-08 codes,
  both paternal (origin) and respondent (destination) occupations are
  grouped into three strata:

  \begin{itemize}
  \tightlist
  \item
    Upper (codes 100--299: managers, directors, professionals)
  \item
    Middle (codes 300--499: skilled white‑collar workers)
  \item
    Lower (codes 500 and above: manual, semi‑skilled, or unskilled
    workers)
  \end{itemize}
\item
  Propensity-score estimation. To model mobility as a treatment---i.e.,
  a transition from origin \emph{j} to destination \emph{k}---I estimate
  multinomial logit propensity scores using covariates that influence an
  individual's likelihood of moving between strata: (a) father's
  educational level, (b) presence of both parents at age 15, (c)
  nationality, (d) age, (e) sex, and (f) ethnicity.
\end{enumerate}

These propensity scores are subsequently employed to adjust for
selection into mobility when assessing its effect on market justice
preferences.

\textbf{Meritocracy}: Meritocratic perception is operationalized through
two components: one addressing effort and another focusing on talent
(\citeproc{ref-young_rise_1958}{Young, 1958}). The item used to gauge
effort is: ``In Chile, people are rewarded for their efforts,'' while
the item for talent is: ``In Chile, people are rewarded for their
intelligence and skills.'' In both cases, respondents indicate their
level of agreement on a five-point Likert scale, ranging from 1
(``strongly disagree'') to 5 (``strongly agree'').

\emph{Controls}

Sociodemographic and attitudinal variables are included to control for
potential composition effects in the population. In terms of
sociodemographic characteristics, I incorporate per capita household
income quantile, educational level (1=Primary or below, 2=Secondary,
3=Technical, 4=University or above), age (in years), and sex (1=Male,
2=Female), which have been shown to significantly influence market
justice preferences (\citeproc{ref-castillo_socialization_2024}{Castillo
et al., 2024}; \citeproc{ref-lindh_public_2015}{Lindh, 2015}). Regarding
attitudinal variables, I include political identification (1=Left,
2=Center, 3=Right, 4=No identification) and subjective social status
(ranging from 1 to 10) because they may confound the relationship
between market justice preferences and social mobility and meritocracy.

\subsection{Methods}\label{methods}

Following Breen and Ermisch's (\citeproc{ref-breen_effects_2024}{2024})
strategy for estimating the causal effect of social mobility on market
justice preferences, I employ linear regression models with
inverse-probability weights (IPW) for mobility conditional on class of
origin. This approach allows estimation of the average causal effect of
moving from an origin class \emph{j} to a destination class \emph{k} by
comparing individuals who actually reach \emph{k} with their
counterfactual outcome had they instead moved to an alternative
destination \emph{k′} (\citeproc{ref-breen_effects_2024}{Breen \&
Ermisch, 2024}). The resulting estimand is the average treatment effect
on the treated (ATT).

All the analyses will be conducted using R software.

\section{References}\label{references}

\phantomsection\label{refs}
\begin{CSLReferences}{1}{0}
\bibitem[\citeproctext]{ref-alesina_intergenerational_2018}
Alesina, A., Stantcheva, S., \& Teso, E. (2018). Intergenerational
{Mobility} and {Preferences} for {Redistribution}. \emph{American
Economic Review}, \emph{108}(2), 521--554.
\url{https://doi.org/10.1257/aer.20162015}

\bibitem[\citeproctext]{ref-arrizabalo_milagro_1995}
Arrizabalo, X. (1995). \emph{{Milagro o quimera: la econom{í}a chilena
durante la dictadura}}. Libros de la Catarata.

\bibitem[\citeproctext]{ref-batruch_belief_2023}
Batruch, A., Jetten, J., Van De Werfhorst, H., Darnon, C., \& Butera, F.
(2023). Belief in {School Meritocracy} and the {Legitimization} of
{Social} and {Income Inequality}. \emph{Social Psychological and
Personality Science}, \emph{14}(5), 621--635.
\url{https://doi.org/10.1177/19485506221111017}

\bibitem[\citeproctext]{ref-benabou_social_2001}
Benabou, R., \& Ok, E. A. (2001). Social {Mobility} and the {Demand} for
{Redistribution}: {The Poum Hypothesis}. \emph{The Quarterly Journal of
Economics}, \emph{116}(2), 447--487.
\url{https://doi.org/10.1162/00335530151144078}

\bibitem[\citeproctext]{ref-boccardo_30_2020}
Boccardo, G. (2020). \emph{30 a{ñ}os de privatizaciones en {Chile}: Lo
que la pandemia revel{ó}} (Nodo XXI). Santiago.

\bibitem[\citeproctext]{ref-breen_social_2004}
Breen, R. (2004). \emph{Social {Mobility} in {Europe}}. OUP Oxford.

\bibitem[\citeproctext]{ref-breen_effects_2024}
Breen, R., \& Ermisch, J. (2024). The {Effects} of {Social Mobility}.
\emph{Sociological Science}, \emph{11}, 467--488.
\url{https://doi.org/10.15195/v11.a17}

\bibitem[\citeproctext]{ref-bucca_merit_2016}
Bucca, M. (2016). Merit and blame in unequal societies: {Explaining
Latin Americans}' beliefs about wealth and poverty. \emph{Research in
Social Stratification and Mobility}, \emph{44}, 98--112.
\url{https://doi.org/10.1016/j.rssm.2016.02.005}

\bibitem[\citeproctext]{ref-busemeyer_skills_2014}
Busemeyer, M. R. (2014). \emph{Skills and {Inequality}: {Partisan
Politics} and the {Political Economy} of {Education Reforms} in {Western
Welfare States}}. Cambridge University Press.

\bibitem[\citeproctext]{ref-busemeyer_welfare_2020}
Busemeyer, M. R., \& Iversen, T. (2020). The {Welfare State} with
{Private Alternatives}: {The Transformation} of {Popular Support} for
{Social Insurance}. \emph{The Journal of Politics}, \emph{82}(2),
671--686. \url{https://doi.org/10.1086/706980}

\bibitem[\citeproctext]{ref-castillo_socialization_2024}
Castillo, J. C., Salgado, M., Carrasco, K., \& Laffert, A. (2024). The
{Socialization} of {Meritocracy} and {Market Justice Preferences} at
{School}. \emph{Societies}, \emph{14}(11), 214.
\url{https://doi.org/10.3390/soc14110214}

\bibitem[\citeproctext]{ref-chancel_world_2022}
Chancel, L., Piketty, T., Saez, E., \& Zucman, G. (2022). World
inequality report 2022.
https://bibliotecadigital.ccb.org.co/handle/11520/27510.

\bibitem[\citeproctext]{ref-davis_principles_2001}
Davis, K., \& Moore, W. E. (2001). Some {Principles} of
{Stratification}. In \emph{Social {Stratification}, {Class}, {Race}, and
{Gender} in {Sociological Perspective}, {Second Edition}} (2nd ed.).
Routledge.

\bibitem[\citeproctext]{ref-day_movin_2017}
Day, M. V., \& Fiske, S. T. (2017). Movin' on {Up}? {How Perceptions} of
{Social Mobility Affect Our Willingness} to {Defend} the {System}.
\emph{Social Psychological and Personality Science}, \emph{8}(3),
267--274. \url{https://doi.org/10.1177/1948550616678454}

\bibitem[\citeproctext]{ref-espinoza_estratificacion_2013}
Espinoza, V., Barozet, E., \& Méndez, M. L. (2013). {Estratificaci{ó}n y
movilidad social bajo un}.

\bibitem[\citeproctext]{ref-gingrich_making_2011}
Gingrich, J. R. (2011). \emph{Making {Markets} in the {Welfare State}:
{The Politics} of {Varying Market Reforms}} (1st ed.). Cambridge
University Press. \url{https://doi.org/10.1017/CBO9780511791529}

\bibitem[\citeproctext]{ref-gugushvili_trends_2014}
Gugushvili, A. (2014). Trends, {Covariates} and {Consequences} of
{Intergenerational Social Mobility} in {Post- Socialist Societies}.

\bibitem[\citeproctext]{ref-gugushvili_intergenerational_2016c}
Gugushvili, A. (2016a). Intergenerational objective and subjective
mobility and attitudes towards income differences: Evidence from
transition societies. \emph{Journal of International and Comparative
Social Policy}, \emph{32}(3), 199--219.
\url{https://doi.org/10.1080/21699763.2016.1206482}

\bibitem[\citeproctext]{ref-gugushvili_intergenerational_2016}
Gugushvili, A. (2016b). Intergenerational {Social Mobility} and {Popular
Explanations} of {Poverty}: {A Comparative Perspective}. \emph{Social
Justice Research}, \emph{29}(4), 402--428.
\url{https://doi.org/10.1007/s11211-016-0275-9}

\bibitem[\citeproctext]{ref-gugushvili_subjective_2017}
Gugushvili, A. (2017). Subjective {Intergenerational Mobility} and
{Support} for {Welfare State Programmes}.

\bibitem[\citeproctext]{ref-immergut_it_2020}
Immergut, E. M., \& Schneider, S. M. (2020). Is it unfair for the
affluent to be able to purchase {``better''} healthcare? {Existential}
standards and institutional norms in healthcare attitudes across 28
countries. \emph{Social Science \& Medicine}, \emph{267}, 113146.
\url{https://doi.org/10.1016/j.socscimed.2020.113146}

\bibitem[\citeproctext]{ref-jaime-castillo_social_2019}
Jaime-Castillo, A. M., \& Marqués-Perales, I. (2019). Social mobility
and demand for redistribution in {Europe}: A comparative analysis.
\emph{The British Journal of Sociology}, \emph{70}(1), 138--165.
\url{https://doi.org/10.1111/1468-4446.12363}

\bibitem[\citeproctext]{ref-kluegel_legitimation_1999}
Kluegel, J. R., Mason, D. S., \& Wegener, B. (1999). The {Legitimation}
of {Capitalism} in the {Postcommunist Transition}: {Public Opinion}
about {Market Justice}, 1991-1996. \emph{European Sociological Review},
\emph{15}(3), 251--283. Retrieved from
\url{https://www.jstor.org/stable/522731}

\bibitem[\citeproctext]{ref-koos_moral_2019}
Koos, S., \& Sachweh, P. (2019). The moral economies of market
societies: Popular attitudes towards market competition, redistribution
and reciprocity in comparative perspective. \emph{Socio-Economic
Review}, \emph{17}(4), 793--821.
\url{https://doi.org/10.1093/ser/mwx045}

\bibitem[\citeproctext]{ref-lee_fairness_2023}
Lee, J.-S., \& Stacey, M. (2023). Fairness perceptions of educational
inequality: The effects of self-interest and neoliberal orientations.
\emph{The Australian Educational Researcher}.
\url{https://doi.org/10.1007/s13384-023-00636-6}

\bibitem[\citeproctext]{ref-lindh_public_2015}
Lindh, A. (2015). Public {Opinion} against {Markets}? {Attitudes}
towards {Market Distribution} of {Social Services} -- {A Comparison} of
17 {Countries}. \emph{Social Policy \& Administration}, \emph{49}(7),
887--910. \url{https://doi.org/10.1111/spol.12105}

\bibitem[\citeproctext]{ref-lindh_bringing_2023}
Lindh, A., \& McCall, L. (2023). Bringing the market in: An expanded
framework for understanding popular responses to economic inequality.
\emph{Socio-Economic Review}, \emph{21}(2), 1035--1055.
\url{https://doi.org/10.1093/ser/mwac018}

\bibitem[\citeproctext]{ref-lopez-roldan_comparative_2021}
López-Roldán, P., \& Fachelli, S. (Eds.). (2021). \emph{Towards a
{Comparative Analysis} of {Social Inequalities} between {Europe} and
{Latin America}}. Cham: Springer International Publishing.
\url{https://doi.org/10.1007/978-3-030-48442-2}

\bibitem[\citeproctext]{ref-mau_moral_2014}
Mau, S. (2014). \emph{The moral economy of welfare states: {Britain} and
{Germany} compared} (First issued in paperback). London New York NY:
Routledge \& Francis Group.

\bibitem[\citeproctext]{ref-mijs_paradox_2019}
Mijs, J. (2019). The paradox of inequality: Income inequality and belief
in meritocracy go hand in hand. \emph{Socio-Economic Review},
\emph{19}(1), 7--35. \url{https://doi.org/10.1093/ser/mwy051}

\bibitem[\citeproctext]{ref-mijs_belief_2022}
Mijs, J., Daenekindt, S., de Koster, W., \& van der Waal, J. (2022).
Belief in {Meritocracy Reexamined}: {Scrutinizing} the {Role} of
{Subjective Social Mobility}. \emph{Social Psychology Quarterly},
\emph{85}(2), 131--141. \url{https://doi.org/10.1177/01902725211063818}

\bibitem[\citeproctext]{ref-miller_selfserving_1975}
Miller, D. T., \& Ross, M. (1975). Self-serving biases in the
attribution of causality: {Fact} or fiction? \emph{Psychological
Bulletin}, \emph{82}(2), 213--225.
\url{https://doi.org/10.1037/h0076486}

\bibitem[\citeproctext]{ref-otero_power_2024}
Otero, G., \& Mendoza, M. (2024). The {Power} of {Diversity}: {Class},
{Networks} and {Attitudes Towards Inequality}. \emph{Sociology},
\emph{58}(4), 851--876. \url{https://doi.org/10.1177/00380385231217625}

\bibitem[\citeproctext]{ref-polanyi_great_1975}
Polanyi, K. (1975). \emph{The great transformation} (Repr). New York,
NY: Octagon Books.

\bibitem[\citeproctext]{ref-reynolds_perceptions_2014}
Reynolds, J., \& Xian, H. (2014). Perceptions of meritocracy in the land
of opportunity. \emph{Research in Social Stratification and Mobility},
\emph{36}, 121--137. \url{https://doi.org/10.1016/j.rssm.2014.03.001}

\bibitem[\citeproctext]{ref-salamon_marketization_1993}
Salamon, L. M. (1993). The {Marketization} of {Welfare}: {Changing
Nonprofit} and {For-Profit Roles} in the {American Welfare State}.
\emph{Social Service Review}, \emph{67}(1), 16--39.
\url{https://doi.org/10.1086/603963}

\bibitem[\citeproctext]{ref-schmidt_experience_2011}
Schmidt, A. W. (2011). The experience of social mobility and the
formation of attitudes toward redistribution. In.

\bibitem[\citeproctext]{ref-shariff_income_2016}
Shariff, A. F., Wiwad, D., \& Aknin, L. B. (2016). Income {Mobility
Breeds Tolerance} for {Income Inequality}: {Cross-National} and
{Experimental Evidence}. \emph{Perspectives on Psychological Science},
\emph{11}(3), 373--380. \url{https://doi.org/10.1177/1745691616635596}

\bibitem[\citeproctext]{ref-streeck_how_2016}
Streeck, W. (2016). \emph{How will capitalism end? Essays on a failing
system}. London: Verso.

\bibitem[\citeproctext]{ref-svallforsMoralEconomyClass2006a}
Svallfors, S. (2006). \emph{The {Moral Economy} of {Class}: {Class} and
{Attitudes} in {Comparative Perspective}}. Stanford University Press.

\bibitem[\citeproctext]{ref-svallfors_political_2007}
Svallfors, S. (Ed.). (2007). \emph{The {Political Sociology} of the
{Welfare State}: {Institutions}, {Social Cleavages}, and {Orientations}}
(1st ed.). Stanford University Press.
\url{https://doi.org/10.2307/j.ctvr0qv0q}

\bibitem[\citeproctext]{ref-torche_intergenerational_2014}
Torche, F. (2014). Intergenerational {Mobility} and {Inequality}: {The
Latin American Case}. \emph{Annual Review of Sociology}, \emph{40}(1),
619--642. \url{https://doi.org/10.1146/annurev-soc-071811-145521}

\bibitem[\citeproctext]{ref-vondemknesebeck_are_2016}
Von Dem Knesebeck, O., Vonneilich, N., \& Kim, T. J. (2016). Are health
care inequalities unfair? {A} study on public attitudes in 23 countries.
\emph{International Journal for Equity in Health}, \emph{15}(1), 61.
\url{https://doi.org/10.1186/s12939-016-0350-8}

\bibitem[\citeproctext]{ref-wen_does_2021}
Wen, F., \& Witteveen, D. (2021). Does perceived social mobility shape
attitudes toward government and family educational investment?
\emph{Social Science Research}, \emph{98}, 102579.
\url{https://doi.org/10.1016/j.ssresearch.2021.102579}

\bibitem[\citeproctext]{ref-wilson_role_2003}
Wilson, C. (2003). The {Role} of a {Merit Principle} in {Distributive
Justice}. \emph{The Journal of Ethics}, \emph{7}(3), 277--314.
\url{https://doi.org/10.1023/A:1024667228488}

\bibitem[\citeproctext]{ref-young_rise_1958}
Young, M. (1958). \emph{The rise of the meritocracy}. New Brunswick,
N.J., U.S.A: Transaction Publishers.

\end{CSLReferences}



\end{document}
