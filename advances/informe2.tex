% Options for packages loaded elsewhere
\PassOptionsToPackage{unicode}{hyperref}
\PassOptionsToPackage{hyphens}{url}
\PassOptionsToPackage{dvipsnames,svgnames,x11names}{xcolor}
%
\documentclass[
  12pt,
]{article}

\usepackage{amsmath,amssymb}
\usepackage{setspace}
\usepackage{iftex}
\ifPDFTeX
  \usepackage[T1]{fontenc}
  \usepackage[utf8]{inputenc}
  \usepackage{textcomp} % provide euro and other symbols
\else % if luatex or xetex
  \usepackage{unicode-math}
  \defaultfontfeatures{Scale=MatchLowercase}
  \defaultfontfeatures[\rmfamily]{Ligatures=TeX,Scale=1}
\fi
\usepackage{lmodern}
\ifPDFTeX\else  
    % xetex/luatex font selection
  \setmainfont[]{Times New Roman}
\fi
% Use upquote if available, for straight quotes in verbatim environments
\IfFileExists{upquote.sty}{\usepackage{upquote}}{}
\IfFileExists{microtype.sty}{% use microtype if available
  \usepackage[]{microtype}
  \UseMicrotypeSet[protrusion]{basicmath} % disable protrusion for tt fonts
}{}
\makeatletter
\@ifundefined{KOMAClassName}{% if non-KOMA class
  \IfFileExists{parskip.sty}{%
    \usepackage{parskip}
  }{% else
    \setlength{\parindent}{0pt}
    \setlength{\parskip}{6pt plus 2pt minus 1pt}}
}{% if KOMA class
  \KOMAoptions{parskip=half}}
\makeatother
\usepackage{xcolor}
\usepackage[margin=2cm]{geometry}
\setlength{\emergencystretch}{3em} % prevent overfull lines
\setcounter{secnumdepth}{5}
% Make \paragraph and \subparagraph free-standing
\ifx\paragraph\undefined\else
  \let\oldparagraph\paragraph
  \renewcommand{\paragraph}[1]{\oldparagraph{#1}\mbox{}}
\fi
\ifx\subparagraph\undefined\else
  \let\oldsubparagraph\subparagraph
  \renewcommand{\subparagraph}[1]{\oldsubparagraph{#1}\mbox{}}
\fi


\providecommand{\tightlist}{%
  \setlength{\itemsep}{0pt}\setlength{\parskip}{0pt}}\usepackage{longtable,booktabs,array}
\usepackage{calc} % for calculating minipage widths
% Correct order of tables after \paragraph or \subparagraph
\usepackage{etoolbox}
\makeatletter
\patchcmd\longtable{\par}{\if@noskipsec\mbox{}\fi\par}{}{}
\makeatother
% Allow footnotes in longtable head/foot
\IfFileExists{footnotehyper.sty}{\usepackage{footnotehyper}}{\usepackage{footnote}}
\makesavenoteenv{longtable}
\usepackage{graphicx}
\makeatletter
\def\maxwidth{\ifdim\Gin@nat@width>\linewidth\linewidth\else\Gin@nat@width\fi}
\def\maxheight{\ifdim\Gin@nat@height>\textheight\textheight\else\Gin@nat@height\fi}
\makeatother
% Scale images if necessary, so that they will not overflow the page
% margins by default, and it is still possible to overwrite the defaults
% using explicit options in \includegraphics[width, height, ...]{}
\setkeys{Gin}{width=\maxwidth,height=\maxheight,keepaspectratio}
% Set default figure placement to htbp
\makeatletter
\def\fps@figure{htbp}
\makeatother
% definitions for citeproc citations
\NewDocumentCommand\citeproctext{}{}
\NewDocumentCommand\citeproc{mm}{%
  \begingroup\def\citeproctext{#2}\cite{#1}\endgroup}
\makeatletter
 % allow citations to break across lines
 \let\@cite@ofmt\@firstofone
 % avoid brackets around text for \cite:
 \def\@biblabel#1{}
 \def\@cite#1#2{{#1\if@tempswa , #2\fi}}
\makeatother
\newlength{\cslhangindent}
\setlength{\cslhangindent}{1.5em}
\newlength{\csllabelwidth}
\setlength{\csllabelwidth}{3em}
\newenvironment{CSLReferences}[2] % #1 hanging-indent, #2 entry-spacing
 {\begin{list}{}{%
  \setlength{\itemindent}{0pt}
  \setlength{\leftmargin}{0pt}
  \setlength{\parsep}{0pt}
  % turn on hanging indent if param 1 is 1
  \ifodd #1
   \setlength{\leftmargin}{\cslhangindent}
   \setlength{\itemindent}{-1\cslhangindent}
  \fi
  % set entry spacing
  \setlength{\itemsep}{#2\baselineskip}}}
 {\end{list}}
\usepackage{calc}
\newcommand{\CSLBlock}[1]{\hfill\break\parbox[t]{\linewidth}{\strut\ignorespaces#1\strut}}
\newcommand{\CSLLeftMargin}[1]{\parbox[t]{\csllabelwidth}{\strut#1\strut}}
\newcommand{\CSLRightInline}[1]{\parbox[t]{\linewidth - \csllabelwidth}{\strut#1\strut}}
\newcommand{\CSLIndent}[1]{\hspace{\cslhangindent}#1}

\usepackage[noblocks]{authblk}
\renewcommand*{\Authsep}{, }
\renewcommand*{\Authand}{, }
\renewcommand*{\Authands}{, }
\renewcommand\Affilfont{\small}
\makeatletter
\@ifpackageloaded{caption}{}{\usepackage{caption}}
\AtBeginDocument{%
\ifdefined\contentsname
  \renewcommand*\contentsname{Table of contents}
\else
  \newcommand\contentsname{Table of contents}
\fi
\ifdefined\listfigurename
  \renewcommand*\listfigurename{List of Figures}
\else
  \newcommand\listfigurename{List of Figures}
\fi
\ifdefined\listtablename
  \renewcommand*\listtablename{List of Tables}
\else
  \newcommand\listtablename{List of Tables}
\fi
\ifdefined\figurename
  \renewcommand*\figurename{Figure}
\else
  \newcommand\figurename{Figure}
\fi
\ifdefined\tablename
  \renewcommand*\tablename{Table}
\else
  \newcommand\tablename{Table}
\fi
}
\@ifpackageloaded{float}{}{\usepackage{float}}
\floatstyle{ruled}
\@ifundefined{c@chapter}{\newfloat{codelisting}{h}{lop}}{\newfloat{codelisting}{h}{lop}[chapter]}
\floatname{codelisting}{Listing}
\newcommand*\listoflistings{\listof{codelisting}{List of Listings}}
\makeatother
\makeatletter
\makeatother
\makeatletter
\@ifpackageloaded{caption}{}{\usepackage{caption}}
\@ifpackageloaded{subcaption}{}{\usepackage{subcaption}}
\makeatother
\ifLuaTeX
  \usepackage{selnolig}  % disable illegal ligatures
\fi
\usepackage{bookmark}

\IfFileExists{xurl.sty}{\usepackage{xurl}}{} % add URL line breaks if available
\urlstyle{same} % disable monospaced font for URLs
\hypersetup{
  pdftitle={Preferences for mercantilization: the role of intergenerational social mobility and beliefs in meritocracy in Chile},
  pdfauthor={Andreas Laffert Tamayo},
  colorlinks=true,
  linkcolor={blue},
  filecolor={Maroon},
  citecolor={Blue},
  urlcolor={Blue},
  pdfcreator={LaTeX via pandoc}}

\title{Preferences for mercantilization: the role of intergenerational
social mobility and beliefs in meritocracy in Chile}


  \author{Andreas Laffert Tamayo}
            \affil{%
                  Instituto de Sociología, Pontificia Universidad
                  Católica de Chile
              }
      
\date{}
\begin{document}
\maketitle

\setstretch{1.15}
\section{Introduction}\label{introduction}

What is the legitimate extent of market inequality in the eyes of the
public? Since the early 1980s, many countries have experienced a
widespread retreat from universal welfare programs and a shift toward
the privatization and commodification of public goods, welfare policies,
and social services (\citeproc{ref-gingrich_making_2011}{Gingrich,
2011}; \citeproc{ref-streeck_how_2016}{Streeck, 2016}). In Latin
America, as elsewhere, neoliberal reforms reshaped welfare-state
institutions by extending market logic into domains of social
reproduction that were traditionally governed by the state
(\citeproc{ref-arrizabalo_milagro_1995}{Arrizabalo, 1995};
\citeproc{ref-ferre_welfare_2023}{Ferre, 2023}). This transformation
reduced the role of public provision and increased the presence of
private actors in core social services
(\citeproc{ref-harvey_breve_2015}{Harvey, 2015}). Echoing Polanyi's
(\citeproc{ref-polanyi_great_1975}{1975}) insight that markets
constitute a distinct moral order, the institutional diffusion of market
rules has fostered a corresponding moral economy: a constellation of
norms and values concerning fair allocation, embedded in institutions
and shaping individual subjectivities
(\citeproc{ref-mau_inequality_2015}{Mau, 2015};
\citeproc{ref-svallforsMoralEconomyClass2006a}{Svallfors, 2006}). Within
this framework, a growing body of research examines the extent to which,
and the mechanisms by which, citizens consider it fair that the
allocation of services like health care, pensions, and education be
governed by market-based criteria---a phenomenon known as \emph{market
justice preferences} (\citeproc{ref-busemeyer_skills_2014}{Busemeyer,
2014}; \citeproc{ref-castillo_socialization_2024}{Castillo et al.,
2024}; \citeproc{ref-immergut_it_2020}{Immergut \& Schneider, 2020};
\citeproc{ref-koos_moral_2019}{Koos \& Sachweh, 2019};
\citeproc{ref-lindh_public_2015}{Lindh, 2015};
\citeproc{ref-lindh_bringing_2023}{Lindh \& McCall, 2023}).
Understanding these preferences is crucial, as they contribute to
legitimizing economic inequality by framing it as the fair result of
individual responsibility and limited state intervention
(\citeproc{ref-mau_inequality_2015}{Mau, 2015}).

Existing literature shows that market justice preferences are shaped by
both the economic and institutional context and individuals' positions
within social stratification. Grounded in the notion that economic
institutions influence people's normative attitudes
(\citeproc{ref-immergut_theoretical_1998}{Immergut, 1998}), studies find
that countries with stronger public provision or more expansive welfare
states exhibit lower levels of market justice preferences
(\citeproc{ref-busemeyer_skills_2014}{Busemeyer, 2014};
\citeproc{ref-immergut_it_2020}{Immergut \& Schneider, 2020}), while
more privatized contexts show stronger support for market-based criteria
(\citeproc{ref-castillo_socialization_2024}{Castillo et al., 2024};
\citeproc{ref-lindh_public_2015}{Lindh, 2015}). In such contexts, market
justice preferences tend to rise as individuals ``ascend'' the social
structure, with those in more privileged positions in terms of class,
education, and income holding stronger preferences for market-based
solutions compared to those in more disadvantaged or at-risk positions
(\citeproc{ref-castillo_socialization_2024}{Castillo et al., 2024};
\citeproc{ref-immergut_it_2020}{Immergut \& Schneider, 2020};
\citeproc{ref-lee_fairness_2023}{Lee \& Stacey, 2023};
\citeproc{ref-lindh_public_2015}{Lindh, 2015};
\citeproc{ref-otero_power_2024}{Otero \& Mendoza, 2024};
\citeproc{ref-svallfors_political_2007}{Svallfors, 2007};
\citeproc{ref-vondemknesebeck_are_2016}{Von Dem Knesebeck et al.,
2016}).

Market justice preferences are shaped not only by objective
socioeconomic conditions but also by popular beliefs about inequality.
Among these, meritocracy is a key normative principle underpinning
market-based distributive preferences
(\citeproc{ref-mau_inequality_2015}{Mau, 2015}). It frames inequality as
inevitable but justifiable through effort and talent
(\citeproc{ref-davis_principles_2001}{Davis \& Moore, 2001};
\citeproc{ref-young_rise_1958}{Young, 1958}). Studies show that
individuals with stronger meritocratic beliefs tend to perceive less
inequality and legitimize it by attributing economic differences to
personal achievement (\citeproc{ref-batruch_belief_2023}{Batruch et al.,
2023}; \citeproc{ref-mijs_paradox_2019}{Mijs, 2019};
\citeproc{ref-wilson_role_2003}{Wilson, 2003}). In highly unequal
societies where access to services is largely governed by market logic,
such beliefs play a critical role in normalizing inequality. Recent
evidence from Chile shows that students who believe effort and talent
are rewarded in their country express stronger preferences for
market-based access to healthcare, pensions, and education
(\citeproc{ref-castillo_socialization_2024}{Castillo et al., 2024}).

Although it is clear that one's social position influences market
justice preferences, the question of how upward or downward mobility
within the social structure affects these preferences remains
unanswered. This question is far from trivial, especially in Latin
America, where many have experienced various forms of mobility amid high
economic inequality and deep welfare privatization
(\citeproc{ref-ferre_welfare_2023}{Ferre, 2023};
\citeproc{ref-lopez-roldan_comparative_2021}{López-Roldán \& Fachelli,
2021}; \citeproc{ref-torche_intergenerational_2014}{Torche, 2014}).
Social origins and destinations affect attitudes toward inequality in
distinct ways (\citeproc{ref-day_movin_2017}{Day \& Fiske, 2017};
\citeproc{ref-gugushvili_intergenerational_2016}{Gugushvili, 2016b},
\citeproc{ref-gugushvili_subjective_2017}{2017};
\citeproc{ref-jaime-castillo_social_2019}{Jaime-Castillo \&
Marqués-Perales, 2019}; \citeproc{ref-mijs_belief_2022}{Mijs et al.,
2022}; \citeproc{ref-wen_does_2021}{Wen \& Witteveen, 2021}), while
movement between these positions exposes individuals to different
experiences and mechanisms that shape their views on what is fair
(\citeproc{ref-gugushvili_trends_2014}{Gugushvili, 2014};
\citeproc{ref-mau_inequality_2015}{Mau, 2015}). Building on this
research, examining the effects of social mobility on market justice
preferences can help to illuminate how inequalities in access to social
services are justified among individuals who have experienced, or not,
changes in their social standing, and what are the normative mechanisms
that guide this justification (\citeproc{ref-mau_inequality_2015}{Mau,
2015}).

Beyond their isolated effects, social mobility and meritocratic beliefs
interact in complex ways to shape market justice preferences. Among
others, a key mechanism proposed in the literature to explain how
mobility influences distributive justice preferences is the
psychological process of self-serving bias in causal attribution
(\citeproc{ref-gugushvili_intergenerational_2016c}{Gugushvili, 2016a};
\citeproc{ref-schmidt_experience_2011}{Schmidt, 2011}). This bias
suggests that individuals attribute failures---such as downward
mobility---to external factors, while crediting successes---such as
upward mobility---to their own merit and effort
(\citeproc{ref-miller_selfserving_1975}{Miller \& Ross, 1975}). Those
who experience upward mobility tend to view their social position as
earned, making them more likely to believe that individuals are
responsible for their own success or failure. Research shows that upward
mobility is associated with weaker preferences for redistribution
(\citeproc{ref-alesina_intergenerational_2018}{Alesina et al., 2018};
\citeproc{ref-gugushvili_intergenerational_2016c}{Gugushvili, 2016a};
\citeproc{ref-jaime-castillo_social_2019}{Jaime-Castillo \&
Marqués-Perales, 2019}; \citeproc{ref-schmidt_experience_2011}{Schmidt,
2011}) and stronger legitimacy of income inequality
(\citeproc{ref-shariff_income_2016}{Shariff et al., 2016}). In contrast,
individuals who experience downward mobility tend to blame structural
factors like inequality and are more supportive of redistribution while
rejecting merit-based explanations
(\citeproc{ref-gugushvili_trends_2014}{Gugushvili, 2014}). Taken
together, I argue that meritocratic beliefs may reinforce this
self-serving attribution mechanism by legitimizing one's social status
as the outcome of personal merit, closely tied to attribution bias.

Against this background, this article pursues two main objectives:
first, to analyze the extent to which intergenerational social mobility
influences market justice preferences regarding healthcare, pensions,
and education; and second, to examine how meritocratic beliefs may
moderate this relationship. Building on a theoretical framework that
emphasizes how neoliberal transformations---particularly through the
privatization and commodification of key areas of social
reproduction---have profoundly reshaped processes of subject formation
(\citeproc{ref-mau_inequality_2015}{Mau, 2015}), the central argument is
that upward mobility increases support for market justice preferences,
while downward mobility decreases it. Moreover, meritocratic beliefs are
expected to moderate this relationship by reflecting a self-serving
attribution bias, whereby individuals justify their social position in
terms of personal merit.

This study focuses on Chile, a particularly intriguing case for
examining market justice preferences. Despite significant economic
growth and poverty reduction, Chile has some of the highest levels of
inequality in Latin America and among OECD countries
(\citeproc{ref-chancel_world_2022}{Chancel et al., 2022};
\citeproc{ref-flores_top_2020}{Flores et al., 2020}). This inequality
coexists with short-range upward mobility among lower-class segments
moving into middle strata, though strong barriers remain to reaching
higher positions (\citeproc{ref-espinoza_estratificacion_2013}{Espinoza
et al., 2013}; \citeproc{ref-torche_intergenerational_2014}{Torche,
2014}). What makes Chile especially salient is that much of this
inequality is rooted in deep neoliberal reforms that institutionalized
the privatization and commodification of key social sectors
(\citeproc{ref-madariaga_three_2020}{Madariaga, 2020}). Introduced
during the dictatorship (1973--1989) and expanded in democracy, these
reforms enabled the unprecedented emergence of markets in health,
pensions, and education, with provision segmented by individuals'
ability to pay and supported by public subsidies
(\citeproc{ref-boccardo_30_2020}{Boccardo, 2020}). In parallel---and
despite waves of protest against inequality and commodification from
2006 to 2019 (\citeproc{ref-somma_no_2021}{Somma et al.,
2021})---Chilean subjectivities have been increasingly shaped by
neoliberal discourses and market logics, influencing their attitudes
toward inequality and welfare distribution
(\citeproc{ref-araujo_desafios_2012}{Araujo \& Martuccelli, 2012};
\citeproc{ref-canalesceron_sujeto_2021}{Canales Cerón et al., 2021}).

In this context, the questions that guide this research are as follows:

\begin{enumerate}
\def\labelenumi{(\arabic{enumi})}
\tightlist
\item
  To what extent does intergenerational social mobility influence market
  justice preferences regarding healthcare, pensions, and education in
  Chile?
\item
  How do meritocratic beliefs condition or moderate this relationship in
  the Chilean context?
\end{enumerate}

To address these questions, this study draws on large-scale,
representative survey data collected in 2018 from the urban Chilean
population aged 18 to 75 (n = 2,726). The next section outlines the
theoretical framework linking market justice preferences, social
mobility, and meritocratic beliefs, and proposes a set of hypotheses.
This is followed by a description of the data, variables, and analytical
strategy. The final sections present the empirical findings, offer an
interpretation of the results, and conclude with a discussion of their
implications.

\section{Theoretical and empirical
background}\label{theoretical-and-empirical-background}

\subsection{Market justice
preferences}\label{market-justice-preferences}

Although redistribution typically refers to the state's capacity to
reallocate resources from the advantaged to the vulnerable, market
institutions also play a central role in distributing socially valuable
goods and rewards (\citeproc{ref-koos_moral_2019}{Koos \& Sachweh,
2019}; \citeproc{ref-lindh_bringing_2023}{Lindh \& McCall, 2023}). As
Polanyi (\citeproc{ref-polanyi_great_1975}{1975}) observed, economic
integration in capitalist societies is primarily organized through
market exchange, governed by a self-regulating price system embedded in
institutional frameworks. These institutions are not mere aggregates of
individual behavior, but social realities endowed with rules,
mechanisms, and normative meanings that shape everyday thinking
(\citeproc{ref-immergut_theoretical_1998}{Immergut, 1998};
\citeproc{ref-koos_moral_2019}{Koos \& Sachweh, 2019}). In this sense,
the economic order is mirrored in a moral economy: collectively shared
norms and beliefs about justice in distribution, embedded and reinforced
through institutions (\citeproc{ref-mau_inequality_2015}{Mau, 2015};
\citeproc{ref-svallfors_political_2007}{Svallfors, 2007}). While most
research from this perspective has focused on welfare institutions,
recent scholarship has brought the market back into focus as a site of
distributive justice beliefs and institutional responses to inequality
(\citeproc{ref-immergut_theoretical_1998}{Immergut, 1998}). In many
countries, privatization and commodification have expanded market logic
into core areas of social reproduction, such as healthcare, education,
pensions, and social security, deepening inequality in access to these
services (\citeproc{ref-ferre_welfare_2023}{Ferre, 2023};
\citeproc{ref-gingrich_making_2011}{Gingrich, 2011}). Yet public support
for market-based welfare provision has grown, even in traditional
welfare states, particularly among higher-income groups who view private
alternatives as more efficient or higher in quality
(\citeproc{ref-busemeyer_welfare_2020}{Busemeyer \& Iversen, 2020}).
This shift calls for broader inquiry into how institutions like markets
structure access to resources and legitimize inequalities
(\citeproc{ref-lindh_public_2015}{Lindh, 2015};
\citeproc{ref-mau_inequality_2015}{Mau, 2015}).

The legitimacy of market-based inequalities is closely tied to beliefs
about distributive justice grounded in market principles. Beyond
stratification structures and socioeconomic conditions, the study of
inequality also examines the beliefs individuals hold about the causes
of social inequality, the normative principles that underpin those
beliefs, the factors that shape them, and their consequences for
attitudes and behavior (\citeproc{ref-kluegel_beliefs_1981}{Kluegel \&
Smith, 1981}). In this regard, empirical research on distributive
justice focuses on individuals' conceptions of how goods and rewards
\emph{should} be distributed in society
(\citeproc{ref-jasso_distributive_2016}{Guillermina Jasso et al.,
2016}). This line of inquiry allows for examining the extent to which
economic inequality is perceived as just, or in a certain sense,
legitimate (\citeproc{ref-castillo_legitimacy_2011}{Castillo, 2011}).
One prominent strand investigates the legitimacy of wage inequality,
particularly salary differences across occupations
(\citeproc{ref-castillo_legitimacy_2011}{Castillo, 2011};
\citeproc{ref-jasso_justice_1978}{Guillermina Jasso, 1978};
\citeproc{ref-jasso_gender_1999}{G. Jasso \& Wegener, 1999}). Another
examines how individuals justify market-generated inequalities in access
to core social services such as healthcare, education, and pensions.
Here, legitimacy stems from the belief that access to these goods should
follow market-based criteria
(\citeproc{ref-castillo_socialization_2024}{Castillo et al., 2024};
\citeproc{ref-lindh_public_2015}{Lindh, 2015}). In such views, these
services are treated as legitimate commodities; goods that can be
traded, priced, and evaluated through market logic
(\citeproc{ref-busemeyer_welfare_2020}{Busemeyer \& Iversen, 2020}).

Market justice preferences refer to normative beliefs that legitimize
the idea that access to core social services---such as healthcare,
education, or pensions---should be determined by market-based criteria.
Following Janmaat's (\citeproc{ref-janmaat_subjective_2013}{2013, p.
359}) distinction, these preferences fall under the category of
``beliefs,'' understood as normative ideas about what inequality should
look like, as opposed to ``perceptions,'' which refer to subjective
evaluations of existing inequality. Market justice preferences reflect
the view that access to these services should depend on individuals'
ability to pay, thus justifying inequalities generated by market
mechanisms (\citeproc{ref-kluegel_legitimation_1999}{Kluegel et al.,
1999}; \citeproc{ref-lindh_public_2015}{Lindh, 2015}). The concept draws
on Lane's (\citeproc{ref-lane_market_1986}{1986}) classic contrast
between market justice and political justice: the former is grounded in
the idea of earned deserts---where rewards reflect effort, productivity,
and skill---while the latter prioritizes need and equality, typically
expressed in welfare state policies. Lane argued that markets and states
differ in purpose (efficiency vs.~need), logic (individual
vs.~collective), and fairness criteria (merit vs.~equality). Market
justice assumes that markets are neutral, self-regulating systems in
which fair procedures yield outcomes proportional to merit. Inequality,
from this perspective, is not only expected but legitimate---as long as
it arises from fair competition. In this way, market justice offers a
moral lens through which individuals can view the commodification and
stratified access to social services as fair and justified
(\citeproc{ref-kluegel_legitimation_1999}{Kluegel et al., 1999};
\citeproc{ref-lindh_public_2015}{Lindh, 2015}).

The empirical study of market justice preferences has relied on diverse
strategies to operationalize how individuals evaluate inequalities
produced by market allocation. One common approach assesses whether
people consider it fair that access to key social services---such as
healthcare, education, or pensions---depends on income. This strategy
builds on foundational work by Kluegel and Smith
(\citeproc{ref-kluegel_beliefs_1981}{1981}), who explored the normative
justifications behind support for economic inequality. More recent
studies have expanded this logic beyond income to include
market-mediated access to welfare goods. For instance, Immergut and
Schneider (\citeproc{ref-immergut_it_2020}{2020}) and von dem Knesebeck
et al. (\citeproc{ref-vondemknesebeck_are_2016}{2016}) examine whether
respondents believe it is just that those with higher incomes receive
better healthcare. Similarly, Lee and Stacey
(\citeproc{ref-lee_fairness_2023}{2023}) apply this approach to
educational opportunities. These studies typically use targeted survey
items that ask respondents to evaluate specific distributive scenarios,
allowing for comparative analyses across countries or welfare regimes.
Complementing these efforts, more recent research has introduced
composite indicators designed to capture broader orientations toward
market-based allocation. For example, Castillo et al.
(\citeproc{ref-castillo_socialization_2024}{2024}) propose a single-item
measure that summarizes individual support for income-based access
across multiple domains---health, education, and pensions---within the
Chilean context. Together, these instruments aim to capture to what
extent individuals perceive market-generated inequality as justified.

Comparative empirical research has identified several individual-level
factors that influence support for market justice. Individuals in more
advantaged socioeconomic positions---those with higher income,
education, and occupational status---are consistently more likely to
endorse market-based distributive principles
(\citeproc{ref-koos_moral_2019}{Koos \& Sachweh, 2019};
\citeproc{ref-lindh_public_2015}{Lindh, 2015};
\citeproc{ref-svallfors_political_2007}{Svallfors, 2007}). For example,
Lindh (\citeproc{ref-lindh_public_2015}{2015}) finds that individuals
from the service class are more likely to support market-based access to
healthcare and education than skilled and unskilled workers across 17
relatively affluent countries. In a comparative analysis, Svallfors
(\citeproc{ref-svallfors_political_2007}{2007}) observes that this
expected class pattern appears clearly only in Sweden, where support for
private education and healthcare varies systematically by class.
Busemeyer (\citeproc{ref-busemeyer_skills_2014}{2014}) similarly shows
that support for private education is stronger among high-income groups,
while Immergut and Schneider (\citeproc{ref-immergut_it_2020}{2020}) and
von dem Knesebeck et al. (\citeproc{ref-vondemknesebeck_are_2016}{2016})
report comparable findings for healthcare, suggesting that wealthier
individuals perceive private provision as a means of maintaining
relative advantage. In Chile, Otero and Mendoza
(\citeproc{ref-otero_power_2024}{2024}) show that individuals with
higher income and university education express stronger support for
market allocation in healthcare, education, and pensions. Beliefs about
inequality and political orientation also matter. In Chile, Castillo et
al. (\citeproc{ref-castillo_socialization_2024}{2024}) show that
individuals who have strong meritocratic beliefs are more likely to
support market-based distribution, while Lee and Stacy
(\citeproc{ref-lee_fairness_2023}{2023}) in Australia suggest that these
preferences are also greater as people lean toward economic
conservatism. In this sense, market justice preferences are shaped by
the interplay between structural position and normative reasoning.

However, individual characteristics alone do not fully explain variation
in support for market justice. A growing body of cross-national research
highlights the importance of institutional arrangements in shaping these
preferences. For instance, Immergut and Schneider
(\citeproc{ref-immergut_it_2020}{2020}) find that in countries with
higher public spending on healthcare, individuals are less likely to
view income-based access as fair. Similarly, Busemeyer
(\citeproc{ref-busemeyer_skills_2014}{2014}) shows that increased public
investment in education is associated with lower support for privatized
provision. Conversely, Lindh (\citeproc{ref-lindh_public_2015}{2015})
finds that in countries with more market-oriented welfare systems,
support for market-based distribution tends to be higher, suggesting
that individual attitudes often align with institutional outputs. These
findings are consistent with neo-institutionalist and policy feedback
theories (\citeproc{ref-campbell_institutional_2020}{Campbell, 2020}),
which argue that institutions do more than redistribute resources---they
also shape the normative categories through which individuals assess who
is deserving of support
(\citeproc{ref-immergut_theoretical_1998}{Immergut, 1998}). In this
view, dominant values and preferences are both shaped by and embedded in
institutional configurations
(\citeproc{ref-busemeyer_skills_2014}{Busemeyer, 2014}), reinforcing the
notion that institutions are not neutral structures, but active
producers of the moral frameworks that legitimize or challenge
inequality.

\subsection{Social mobility}\label{social-mobility}

The study of social mobility, its drivers and consequences has long been
central to sociology. Classic theorists explored not only movement
across social hierarchies but also its broader implications for class
conflict, norm stability, and institutional change
(\citeproc{ref-breen_effects_2024}{Breen \& Ermisch, 2024}). Sorokin
(\citeproc{ref-sorokin_social_1927}{1927}) introduced the concept
formally, defining mobility as the shift of individuals, values, or
objects between positions within a stratification system, and
distinguishing between horizontal and vertical forms. Later work
differentiated intergenerational from intragenerational mobility, as
well as absolute mobility---driven by structural change---from relative
mobility, which captures the extent to which origins constrain
destinations (\citeproc{ref-eyles_social_2022}{Eyles et al., 2022}). In
Latin America, research shows high absolute but low relative mobility
(\citeproc{ref-bucca_merit_2016}{Bucca, 2016}): although educational
expansion and economic modernization have enabled some upward movement,
status reproduction remains strong, especially among elites
(\citeproc{ref-lopez-roldan_comparative_2021}{López-Roldán \& Fachelli,
2021}; \citeproc{ref-torche_intergenerational_2014}{Torche, 2014}). This
reflects deep structural inequality, segmented education systems,
stratified labor markets, and legacies of dependent development that
restrict access to mobility channels and reinforce the intergenerational
transmission of advantage.

Beyond mapping mobility patterns, growing research has examined its
subjective and attitudinal effects. Mobility effects---defined as
outcomes resulting from movement between origin and destination classes
(\citeproc{ref-breen_effects_2024}{Breen \& Ermisch, 2024})---have long
attracted theoretical interest. Sorokin's (1959) dissociative hypothesis
posits that mobility, whether upward or downward, may produce
psychological strain due to conflicting norms between class contexts,
leading to lower life satisfaction. Similarly, Lenski
(\citeproc{ref-lenski_status_1954}{1954}) argued that status
inconsistency---mismatches among education, income, and occupation---can
undermine well-being. These ideas underpin extensive empirical work
linking intergenerational mobility to outcomes such as life
satisfaction, mental health, and stress
(\citeproc{ref-gugushvili_heterogeneous_2024}{Gugushvili, 2024};
\citeproc{ref-hadjar_does_2015}{Hadjar \& Samuel, 2015};
\citeproc{ref-prag_subjective_2021}{Präg \& Gugushvili, 2021}).

Research on intergenerational social mobility has increasingly examined
its effects on attitudes toward economic inequality. A key area of
inquiry focuses on how upward and downward mobility influence support
for redistribution, though findings remain mixed. Alesina et al.
(\citeproc{ref-alesina_intergenerational_2018}{2018}) show that
individuals with pessimistic expectations about their
mobility---particularly those anticipating downward movement---are more
likely to support generous redistributive policies. Similarly, Ares
(\citeproc{ref-ares_changing_2020}{2020}) finds that upwardly mobile
individuals tend to be less supportive of state-led redistribution
compared to those who have experienced downward mobility. Comparative
studies by Schmidt (\citeproc{ref-schmidt_experience_2011}{2011}) and
Gugushvili (\citeproc{ref-gugushvili_subjective_2017}{Gugushvili, 2017})
likewise report that subjective upward mobility is associated with
weaker preferences for redistribution, while downward mobility
strengthens redistributive support. However, recent causal evidence from
Breen and Ermisch (\citeproc{ref-breen_effects_2024}{2024}) suggests the
opposite: upward mobility may increase redistributive preferences, while
downward mobility may reduce them.

Beyond redistribution, other studies have explored the impact of
mobility on broader beliefs about inequality. Gugushvili
(\citeproc{ref-gugushvili_intergenerational_2016}{2016b}) finds that
upwardly mobile individuals are more likely to adopt individualistic
attributions of poverty and to legitimize income inequality,
particularly in post-socialist societies. Similarly, Bucca
(\citeproc{ref-bucca_merit_2016}{2016}) shows that subjective upward
mobility reinforces individualistic explanations of wealth in seven
Latin American countries. At the macro level, Shariff et al.
(\citeproc{ref-shariff_income_2016}{2016}) demonstrate that higher
levels of national economic mobility correlate with greater tolerance of
inequality. In contrast, Day and Fiske
(\citeproc{ref-day_movin_2017}{2017}) find that low perceived mobility
undermines belief in meritocracy and a just world, thereby weakening
system justification. Taken together, this body of research suggests
that mobility shapes attitudes toward inequality and justice through
multiple, and sometimes contradictory, mechanisms.

The literature on the effects of social mobility has proposed various
mechanisms to explain how and why changes in social position may
influence individual outcomes
(\citeproc{ref-helgason_class_2025}{Helgason \& Rehm, 2025}). One of the
most prominent is the self-interest mechanism, which posits that
individuals who experience upward or downward mobility undergo a shift
in their material interests, thereby altering their perceptions and
preferences (\citeproc{ref-ares_changing_2020}{Ares, 2020};
\citeproc{ref-helgason_longterm_2023}{Helgason \& Rehm, 2023};
\citeproc{ref-langsaether_explaining_2022}{Langsæther et al., 2022}).
Closely related to this logic is the Prospect of Upward Mobility (POUM)
hypothesis, which suggests that individuals may oppose redistribution
not because of their current position, but because they anticipate
improving their status in the future
(\citeproc{ref-benabou_social_2001}{Benabou \& Ok, 2001}). A second line
of explanation draws on the framework of theories of socialization.
Within this tradition, hypotheses such as acculturation, socialization,
and status maximization propose that individuals adjust their attitudes
based on the norms and values of either their class of origin or their
destination, or a combination of both
(\citeproc{ref-jaime-castillo_social_2019}{Jaime-Castillo \&
Marqués-Perales, 2019}). However, most of these mechanisms focus
primarily on the indirect effects of origin and destination positions,
rather than on the direct effect of experience of movement itself. In
response to this gap, recent research has highlighted the role of
self-serving bias in causal attribution processes, suggesting that
individuals tend to explain their mobility trajectories in ways that
justify their current position, which in turn shapes their beliefs and
preferences
(\citeproc{ref-gugushvili_intergenerational_2016c}{Gugushvili, 2016a};
\citeproc{ref-molina_its_2019}{Molina et al., 2019};
\citeproc{ref-schmidt_experience_2011}{Schmidt, 2011}).

The self-serving bias mechanism builds on a value-oriented perspective,
emphasizing that individuals' experiences of social mobility shape their
causal attributions, which in turn influence their beliefs about justice
and distributive preferences
(\citeproc{ref-gugushvili_trends_2014}{Gugushvili, 2014}). Causal
attribution refers to the process through which individuals generate
explanations for their own behavior and outcomes, as well as those of
others (\citeproc{ref-gugushvili_intergenerational_2016}{Gugushvili,
2016b}). In this view, people's interpretations of economic inequality
depend on whether they believe such disparities reflect unequal
individual contributions. Individuals who adopt an internal attribution
framework tend to see success or failure as rooted in personal
characteristics such as effort, talent, or merit. In contrast, those who
regard inequality as unjust are more likely to adopt an external
attribution model, viewing outcomes as the result of structural barriers
beyond individual control (\citeproc{ref-kluegel_beliefs_1981}{Kluegel
\& Smith, 1981}). This mechanism, often described as intrapersonal
causal attribution, reflects how people explain their own socioeconomic
positions---typically attributing their successes to internal qualities
while blaming failures on external circumstances
(\citeproc{ref-miller_selfserving_1975}{Miller \& Ross, 1975}). Over
time, individuals may revise their beliefs and attitudes: while early
views are shaped by their social origin, these are later adjusted in
light of personal experiences of mobility and the perceived role of
ascribed versus achieved factors in determining socioeconomic outcomes
(\citeproc{ref-gugushvili_intergenerational_2016}{Gugushvili, 2016b}).

Empirical research has provided support for the self-serving bias
mechanism in shaping redistributive preferences and attitudes toward the
legitimacy of inequality. For instance, Schmidt
(\citeproc{ref-schmidt_experience_2011}{2011}) finds that individuals
who experience upward mobility are more likely to interpret their
success as the result of personal effort or merit, and consequently
perceive redistribution as less necessary. Conversely, individuals who
experience downward mobility tend to attribute their decline to external
circumstances---such as structural inequality or unemployment---and show
stronger support for redistribution. In a comparative analysis across
different welfare domains, Gugushvili
(\citeproc{ref-gugushvili_subjective_2017}{2017}) demonstrates that
upward mobility is associated with lower support for government spending
on housing and pensions, while individuals who experience downward
mobility express lower support for healthcare and education spending,
but favor increased investment in housing and pensions, reflecting the
material nature of these domains. Moreover, Gugushvili
(\citeproc{ref-gugushvili_intergenerational_2016c}{2016a}) finds that
upward mobility is linked to greater justification of income inequality,
suggesting that improved social standing reinforces an attributional
view in which success is seen as the result of individual
characteristics---thus legitimizing inequality as a fair outcome.

These findings suggest that causal attribution processes mediate the
relationship between mobility experiences and justice-related attitudes.
Accordingly, this study proposes the following hypothesis:

\(H1\): Experiencing upward (downward) social mobility is positively
(negatively) associated with greater support for market justice in
healthcare, pensions, and education.

\subsection{Meritocracy}\label{meritocracy}

Meritocracy constitutes a central ideological framework for legitimizing
different types of social inequality, for instance through market
justice beliefs. Rooted in the belief that rewards and positions should
be allocated based on individual effort and talent, meritocracy operates
as a normative ideal and a descriptive belief about how society
functions. As initially conceptualized by Michael Young
(\citeproc{ref-young_rise_1958}{1958}), the term was meant to critique a
system in which merit-based stratification becomes a new form of
inequality. However, over time, meritocracy has been widely supported in
many societies as a fair and desirable principle of distribution,
particularly within liberal democracies and market-oriented economies
(\citeproc{ref-mijs_paradox_2019}{Mijs, 2019};
\citeproc{ref-sandel_tyranny_2020}{Sandel, 2020}). From a sociological
perspective, the belief in meritocracy is more than a cognitive
assessment; it reflects a moral lens through which individuals interpret
inequality. People who believe that success results from hard work and
talent are more likely to view social and economic disparities as
legitimate (\citeproc{ref-batruch_belief_2023}{Batruch et al., 2023};
\citeproc{ref-castillo_meritocracia_2019}{Castillo et al., 2019}).
Conversely, if they see outcomes as driven by luck, social origin, or
systemic barriers, inequality is more likely to be perceived as unjust.
This distinction becomes crucial in societies with persistent structural
inequality, where public narratives often emphasize personal
responsibility and merit while overlooking entrenched disadvantages.

I adopt a multidimensional perspective on meritocracy, distinguishing
between two key dimensions: effort-based and talent-based perceptions.
This distinction is essential, as it captures different pathways through
which individuals justify inequality
(\citeproc{ref-young_rise_1958}{Young, 1958}). Effort-based meritocracy
emphasizes hard work and perseverance as the basis for social rewards,
aligning closely with cultural narratives of personal responsibility. A
talent-based meritocracy, by contrast, emphasizes intelligence and
innate abilities, which are often perceived as less malleable and more
unequally distributed. Both dimensions have been shown to correlate with
acceptance of inequality, but they may carry distinct implications for
how inequality is justified in specific domains
(\citeproc{ref-castillo_multidimensional_2023}{Castillo et al., 2023}).
The relevance of this distinction is supported by recent studies, which
show that individuals respond differently to these dimensions. For
instance, perceptions that effort is rewarded in society are more
strongly associated with positive evaluations of fairness and acceptance
of unequal outcomes (\citeproc{ref-batruch_belief_2023}{Batruch et al.,
2023}; \citeproc{ref-wiederkehr_belief_2015}{Wiederkehr et al., 2015};
\citeproc{ref-wilson_role_2003}{Wilson, 2003}). This may be because
effort is seen as a controllable and morally virtuous trait, whereas
talent is often perceived as a natural advantage. Consequently,
effort-based meritocracy is likely more potent in legitimizing
inequality, particularly in neoliberal contexts.

These dimensions of meritocracy reflect how respondents perceive
society's distributive logic, regardless of whether they endorse
meritocratic principles. This distinction aligns with recent findings
indicating that individuals distinguish between how merit is perceived
in society and how it should ideally operate, which in turn shapes their
preferences for redistribution and justice
(\citeproc{ref-tejero-peregrina_perceived_2025}{Tejero-Peregrina et al.,
2025}). Meritocratic beliefs serve as symbolic justifications for
unequal outcomes, particularly when access is stratified by income or
social opportunity. Prior studies have shown that individuals who
perceive higher levels of meritocracy tend to express stronger support
for unequal distributions that reflect market outcomes
(\citeproc{ref-castillo_socialization_2024}{Castillo et al., 2024};
\citeproc{ref-castillo_meritocracia_2019}{Castillo et al., 2019}).

In addition to influencing individual attitudes toward inequality,
meritocratic beliefs can contribute to social division and the
stigmatization of disadvantaged groups. Recent research has demonstrated
that exposure to meritocratic narratives can reinforce the belief that
poverty results from individual failings rather than systemic
conditions, reducing support for redistributive measures and increasing
the stigmatization of the poor (\citeproc{ref-hoyt_mindsets_2023}{Hoyt
et al., 2023}). This reinforces negative stereotypes and reduces empathy
toward individuals from lower socioeconomic backgrounds. Moreover,
Busemeyer et al. (\citeproc{ref-busemeyer_positive_2021}{2021}) argues
that meritocratic narratives can serve as feedback mechanisms that shape
public opinion and well-being by framing individuals' understanding of
welfare outcomes as deserved or undeserved within existing institutional
structures. This psychological mechanism highlights the normative power
of meritocracy in stabilizing unequal systems by shaping political
attitudes and personal perceptions of success and failure.

Importantly, recent research has explored how meritocratic beliefs
interact with experiences of intergenerational mobility to shape
distributive attitudes. The belief that one's success is earned can lead
upwardly mobile individuals to internalize meritocratic narratives and
justify existing inequalities, reinforcing support for market justice
(\citeproc{ref-gugushvili_intergenerational_2016c}{Gugushvili, 2016a};
\citeproc{ref-molina_its_2019}{Molina et al., 2019}). Conversely,
downwardly mobile individuals who maintain strong meritocratic beliefs
may interpret their status as a personal failure, reducing their support
for redistribution (\citeproc{ref-day_movin_2017}{Day \& Fiske, 2017}).
At the macro level, Shariff et al.
(\citeproc{ref-shariff_income_2016}{2016}) show that higher perceived
mobility increases tolerance for inequality, suggesting that meritocracy
and mobility are mutually reinforcing.

Taken together, this literature supports the idea that meritocratic
beliefs can moderate the relationship between mobility and market
justice preferences. Individuals who experience mobility---especially
upward---may draw on meritocratic narratives to legitimize both their
own status and broader inequalities, thereby strengthening their support
for market-based distribution. Accordingly, the second hypothesis of
this study is:

\(H2\): The positive (negative) relationship between upward (downward)
social mobility and support for market justice in healthcare, pensions,
and education is moderated by meritocratic beliefs; specifically, this
association is stronger (weaker) among individuals with higher
perceptions of meritocracy.

\subsection{The Chilean context}\label{the-chilean-context}

\section{Data, variables and methods}\label{data-variables-and-methods}

\subsection{Data}\label{data}

This study draws on data from the Chilean Longitudinal Social Survey
(ELSOC) of the Center for Social Conflict and Cohesion Studies (COES).
The survey is an annual panel collected from 2016 to 2023, comprising
two independent samples (original and refreshment), featuring permanent
and rotating questionnaire modules. It evaluates how individuals think,
feel, and behave regarding conflict and social cohesion in Chile. The
sampling design is complex, probabilistic, clustered, multistage, and
stratified by city size, targeting men and women aged 18--75 who
habitually reside in private dwellings in urban areas spanning 40 cities
(92 municipalities, 13 regions). This analysis focuses on 2023, the most
recent wave, which includes 1,737 respondents in the original sample and
989 in the refreshment sample. Further details on sampling, attrition,
and weighting can be found at https://coes.cl/encuesta-panel/, and the
dataset is publicly available at
https://dataverse.harvard.edu/dataverse/elsoc.

\subsection{Variables}\label{variables}

\emph{Outcome variables}

\textbf{Market justice preferences}: The outcome variables in this study
are market justice preferences. This construct is operationalized
through three variables that address the degree of justification
regarding whether access to social services in healthcare, pensions, and
education should be income conditional. Specifically, the justification
of inequality in healthcare is assessed by the question: ``Is it fair in
Chile that people with higher incomes can access better healthcare than
people with lower incomes?'' The same question is asked for pensions and
education. In all cases, respondents indicate their level of agreement
on a five-point Likert scale ranging from 1 (``strongly disagree'') to 5
(``strongly agree''). Although these items theoretically reflect the
same underlying concept of market justice
(\citeproc{ref-castillo_socialization_2024}{Castillo et al., 2024};
\citeproc{ref-lindh_public_2015}{Lindh, 2015}), they are analyzed
separately because healthcare, pensions, and education may prompt
distinct response patterns
(\citeproc{ref-busemeyer_skills_2014}{Busemeyer, 2014};
\citeproc{ref-immergut_it_2020}{Immergut \& Schneider, 2020};
\citeproc{ref-lee_fairness_2023}{Lee \& Stacey, 2023}).

\emph{Independent variables}

\textbf{Social mobility}: Intergenerational social mobility is treated
as an exposure indicating whether respondents occupy a different class
position from their fathers. Following Breen and Ermisch's
(\citeproc{ref-breen_effects_2024}{2024}) framework for estimating
causal effects of mobility, the variable is constructed in two stages.

\begin{enumerate}
\def\labelenumi{\arabic{enumi}.}
\item
  Class assignment. Using the International Socio-Economic Index of
  Occupational Status (ISEI) derived from three-digit ISCO-08 codes,
  both paternal (origin) and respondent (destination) occupations are
  grouped into three strata:

  \begin{itemize}
  \tightlist
  \item
    Upper (codes 100-299: managers, directors, professionals)
  \item
    Middle (codes 300-499: skilled white-collar workers)
  \item
    Lower (codes 500 and above: manual, semi-skilled, or unskilled
    workers)
  \end{itemize}
\item
  Propensity-score estimation. To model mobility as a treatment---i.e.,
  a transition from origin \emph{j} to destination \emph{k}---I estimate
  multinomial logit propensity scores using covariates that influence an
  individual's likelihood of moving between strata: (a) father's
  educational level, (b) presence of both parents at age 15, (c)
  nationality, (d) age, (e) sex, and (f) ethnicity.
\end{enumerate}

These propensity scores are subsequently employed to adjust for
selection into mobility when assessing its effect on market justice
preferences.

\textbf{Meritocracy}: Meritocratic perception is operationalized through
two components: one addressing effort and another focusing on talent
(\citeproc{ref-young_rise_1958}{Young, 1958}). The item used to gauge
effort is: ``In Chile, people are rewarded for their efforts,'' while
the item for talent is: ``In Chile, people are rewarded for their
intelligence and skills.'' In both cases, respondents indicate their
level of agreement on a five-point Likert scale, ranging from 1
(``strongly disagree'') to 5 (``strongly agree'').

\emph{Controls}

Sociodemographic and attitudinal variables are included to control for
potential composition effects in the population. In terms of
sociodemographic characteristics, I incorporate per capita household
income quantile, educational level (1=Primary or below, 2=Secondary,
3=Technical, 4=University or above), age (in years), and sex (1=Male,
2=Female), which have been shown to significantly influence market
justice preferences (\citeproc{ref-castillo_socialization_2024}{Castillo
et al., 2024}; \citeproc{ref-lindh_public_2015}{Lindh, 2015}). Regarding
attitudinal variables, I include political identification (1=Left,
2=Center, 3=Right, 4=No identification) and subjective social status
(ranging from 1 to 10) because they may confound the relationship
between market justice preferences and social mobility and meritocracy.

\subsection{Methods}\label{methods}

Following Breen and Ermisch's (\citeproc{ref-breen_effects_2024}{2024})
strategy for estimating the causal effect of social mobility on market
justice preferences, I employ linear regression models with
inverse-probability weights (IPW) for mobility conditional on class of
origin. This approach allows estimation of the average causal effect of
moving from an origin class \emph{j} to a destination class \emph{k} by
comparing individuals who actually reach \emph{k} with their
counterfactual outcome had they instead moved to an alternative
destination \emph{k′} (\citeproc{ref-breen_effects_2024}{Breen \&
Ermisch, 2024}). The resulting estimand is the average treatment effect
on the treated (ATT).

All the analyses will be conducted using R software.

\section{References}\label{references}

\phantomsection\label{refs}
\begin{CSLReferences}{1}{0}
\bibitem[\citeproctext]{ref-alesina_intergenerational_2018}
Alesina, A., Stantcheva, S., \& Teso, E. (2018). Intergenerational
{Mobility} and {Preferences} for {Redistribution}. \emph{American
Economic Review}, \emph{108}(2), 521--554.
\url{https://doi.org/10.1257/aer.20162015}

\bibitem[\citeproctext]{ref-araujo_desafios_2012}
Araujo, K., \& Martuccelli, D. (2012). \emph{Desaf{í}os comunes: Retrato
de la sociedad chilena y sus individuos} (1a. ed). Santiago: LOM
Ediciones.

\bibitem[\citeproctext]{ref-ares_changing_2020}
Ares, M. (2020). Changing classes, changing preferences: How social
class mobility affects economic preferences. \emph{West European
Politics}, \emph{43}(6), 1211--1237.
\url{https://doi.org/10.1080/01402382.2019.1644575}

\bibitem[\citeproctext]{ref-arrizabalo_milagro_1995}
Arrizabalo, X. (1995). \emph{{Milagro o quimera: la econom{í}a chilena
durante la dictadura}}. Libros de la Catarata.

\bibitem[\citeproctext]{ref-batruch_belief_2023}
Batruch, A., Jetten, J., Van De Werfhorst, H., Darnon, C., \& Butera, F.
(2023). Belief in {School Meritocracy} and the {Legitimization} of
{Social} and {Income Inequality}. \emph{Social Psychological and
Personality Science}, \emph{14}(5), 621--635.
\url{https://doi.org/10.1177/19485506221111017}

\bibitem[\citeproctext]{ref-benabou_social_2001}
Benabou, R., \& Ok, E. A. (2001). Social {Mobility} and the {Demand} for
{Redistribution}: {The Poum Hypothesis}. \emph{The Quarterly Journal of
Economics}, \emph{116}(2), 447--487.
\url{https://doi.org/10.1162/00335530151144078}

\bibitem[\citeproctext]{ref-boccardo_30_2020}
Boccardo, G. (2020). \emph{30 a{ñ}os de privatizaciones en {Chile}: Lo
que la pandemia revel{ó}} (Nodo XXI). Santiago.

\bibitem[\citeproctext]{ref-breen_effects_2024}
Breen, R., \& Ermisch, J. (2024). The {Effects} of {Social Mobility}.
\emph{Sociological Science}, \emph{11}, 467--488.
\url{https://doi.org/10.15195/v11.a17}

\bibitem[\citeproctext]{ref-bucca_merit_2016}
Bucca, M. (2016). Merit and blame in unequal societies: {Explaining
Latin Americans}' beliefs about wealth and poverty. \emph{Research in
Social Stratification and Mobility}, \emph{44}, 98--112.
\url{https://doi.org/10.1016/j.rssm.2016.02.005}

\bibitem[\citeproctext]{ref-busemeyer_skills_2014}
Busemeyer, M. (2014). \emph{Skills and {Inequality}: {Partisan Politics}
and the {Political Economy} of {Education Reforms} in {Western Welfare
States}}. Cambridge University Press.

\bibitem[\citeproctext]{ref-busemeyer_positive_2021}
Busemeyer, M., Abrassart, A., \& Nezi, R. (2021). Beyond {Positive} and
{Negative}: {New Perspectives} on {Feedback Effects} in {Public Opinion}
on the {Welfare State}. \emph{British Journal of Political Science},
\emph{51}(1), 137--162. \url{https://doi.org/10.1017/S0007123418000534}

\bibitem[\citeproctext]{ref-busemeyer_welfare_2020}
Busemeyer, M., \& Iversen, T. (2020). The {Welfare State} with {Private
Alternatives}: {The Transformation} of {Popular Support} for {Social
Insurance}. \emph{The Journal of Politics}, \emph{82}(2), 671--686.
\url{https://doi.org/10.1086/706980}

\bibitem[\citeproctext]{ref-campbell_institutional_2020}
Campbell, J. L. (2020). \emph{Institutional {Change} and
{Globalization}}. Princeton University Press.
\url{https://doi.org/10.2307/j.ctv131bw68}

\bibitem[\citeproctext]{ref-canalesceron_sujeto_2021}
Canales Cerón, M., Orellana Calderón, V. S., \& Guajardo Mañán, F.
(2021). Sujeto y cotidiano en la era neoliberal: El caso de la
educaci{ó}n chilena. \emph{Revista Mexicana de Ciencias Pol{í}ticas y
Sociales}, \emph{67}(244).
\url{https://doi.org/10.22201/fcpys.2448492xe.2022.244.70386}

\bibitem[\citeproctext]{ref-castillo_legitimacy_2011}
Castillo, J. C. (2011). Legitimacy of {Inequality} in a {Highly Unequal
Context}: {Evidence} from the {Chilean Case}. \emph{Social Justice
Research}, \emph{24}(4), 314--340.
\url{https://doi.org/10.1007/s11211-011-0144-5}

\bibitem[\citeproctext]{ref-castillo_multidimensional_2023}
Castillo, J. C., Iturra, J., Maldonado, L., Atria, J., \& Meneses, F.
(2023). A {Multidimensional Approach} for {Measuring Meritocratic
Beliefs}: {Advantages}, {Limitations} and {Alternatives} to the {ISSP
Social Inequality Survey}. \emph{International Journal of Sociology},
1--25. \url{https://doi.org/10.1080/00207659.2023.2274712}

\bibitem[\citeproctext]{ref-castillo_socialization_2024}
Castillo, J. C., Salgado, M., Carrasco, K., \& Laffert, A. (2024). The
{Socialization} of {Meritocracy} and {Market Justice Preferences} at
{School}. \emph{Societies}, \emph{14}(11), 214.
\url{https://doi.org/10.3390/soc14110214}

\bibitem[\citeproctext]{ref-castillo_meritocracia_2019}
Castillo, J. C., Torres, A., Atria, J., \& Maldonado, L. (2019).
Meritocracia y desigualdad econ{ó}mica: {Percepciones}, preferencias e
implicancias. \emph{Revista Internacional de Sociolog{í}a},
\emph{77}(1), 117. \url{https://doi.org/10.3989/ris.2019.77.1.17.114}

\bibitem[\citeproctext]{ref-chancel_world_2022}
Chancel, L., Piketty, T., Saez, E., \& Zucman, G. (2022). World
inequality report 2022.
https://bibliotecadigital.ccb.org.co/handle/11520/27510.

\bibitem[\citeproctext]{ref-davis_principles_2001}
Davis, K., \& Moore, W. E. (2001). Some {Principles} of
{Stratification}. In \emph{Social {Stratification}, {Class}, {Race}, and
{Gender} in {Sociological Perspective}, {Second Edition}} (2nd ed.).
Routledge.

\bibitem[\citeproctext]{ref-day_movin_2017}
Day, M. V., \& Fiske, S. T. (2017). Movin' on {Up}? {How Perceptions} of
{Social Mobility Affect Our Willingness} to {Defend} the {System}.
\emph{Social Psychological and Personality Science}, \emph{8}(3),
267--274. \url{https://doi.org/10.1177/1948550616678454}

\bibitem[\citeproctext]{ref-espinoza_estratificacion_2013}
Espinoza, V., Barozet, E., \& Méndez, M. L. (2013). {Estratificaci{ó}n y
movilidad social bajo un}.

\bibitem[\citeproctext]{ref-eyles_social_2022}
Eyles, A., Major, L. E., \& Machin, S. (2022). \emph{Social {Mobility} -
{Past}, {Present} and {Future}: {The State} of {Play} in {Social
Mobility}, on the 25th {Anniversary} of the {Sutton Trust}}. Sutton
Trust.

\bibitem[\citeproctext]{ref-ferre_welfare_2023}
Ferre, J. C. (2023). Welfare regimes in twenty-first-century {Latin
America}. \emph{Journal of International and Comparative Social Policy},
\emph{39}(2), 101--127. \url{https://doi.org/10.1017/ics.2023.16}

\bibitem[\citeproctext]{ref-flores_top_2020}
Flores, I., Sanhueza, C., Atria, J., \& Mayer, R. (2020). Top {Incomes}
in {Chile}: {A Historical Perspective} on {Income Inequality},
1964--2017. \emph{Review of Income and Wealth}, \emph{66}(4), 850--874.
\url{https://doi.org/10.1111/roiw.12441}

\bibitem[\citeproctext]{ref-gingrich_making_2011}
Gingrich, J. R. (2011). \emph{Making {Markets} in the {Welfare State}:
{The Politics} of {Varying Market Reforms}} (1st ed.). Cambridge
University Press. \url{https://doi.org/10.1017/CBO9780511791529}

\bibitem[\citeproctext]{ref-gugushvili_trends_2014}
Gugushvili, A. (2014). Trends, {Covariates} and {Consequences} of
{Intergenerational Social Mobility} in {Post- Socialist Societies}.

\bibitem[\citeproctext]{ref-gugushvili_intergenerational_2016c}
Gugushvili, A. (2016a). Intergenerational objective and subjective
mobility and attitudes towards income differences: Evidence from
transition societies. \emph{Journal of International and Comparative
Social Policy}, \emph{32}(3), 199--219.
\url{https://doi.org/10.1080/21699763.2016.1206482}

\bibitem[\citeproctext]{ref-gugushvili_intergenerational_2016}
Gugushvili, A. (2016b). Intergenerational {Social Mobility} and {Popular
Explanations} of {Poverty}: {A Comparative Perspective}. \emph{Social
Justice Research}, \emph{29}(4), 402--428.
\url{https://doi.org/10.1007/s11211-016-0275-9}

\bibitem[\citeproctext]{ref-gugushvili_subjective_2017}
Gugushvili, A. (2017). Subjective {Intergenerational Mobility} and
{Support} for {Welfare State Programmes}.

\bibitem[\citeproctext]{ref-gugushvili_heterogeneous_2024}
Gugushvili, A. (2024). The heterogeneous well-being effects of
intergenerational mobility perceptions. \emph{Journal of Health
Psychology}, \emph{29}(2), 99--112.
\url{https://doi.org/10.1177/13591053231187345}

\bibitem[\citeproctext]{ref-hadjar_does_2015}
Hadjar, A., \& Samuel, R. (2015). Does upward social mobility increase
life satisfaction? {A} longitudinal analysis using {British} and {Swiss}
panel data. \emph{Research in Social Stratification and Mobility},
\emph{39}, 48--58. \url{https://doi.org/10.1016/j.rssm.2014.12.002}

\bibitem[\citeproctext]{ref-harvey_breve_2015}
Harvey, D. (2015). \emph{{Breve historia del neoliberalismo}}. Madrid
(Espa{ñ}a): Ediciones Akal.

\bibitem[\citeproctext]{ref-helgason_longterm_2023}
Helgason, A. F., \& Rehm, P. (2023). Long-term income trajectories and
the evolution of political attitudes. \emph{European Journal of
Political Research}, \emph{62}(1), 264--284.
\url{https://doi.org/10.1111/1475-6765.12506}

\bibitem[\citeproctext]{ref-helgason_class_2025}
Helgason, A. F., \& Rehm, P. (2025). Class experiences and the long-term
evolution of economic values. \emph{Social Forces}, \emph{103}(3),
1125--1143. \url{https://doi.org/10.1093/sf/soae135}

\bibitem[\citeproctext]{ref-hoyt_mindsets_2023}
Hoyt, C. L., Burnette, J. L., Billingsley, J., Becker, W., \& Babij, A.
D. (2023). Mindsets of poverty: {Implications} for redistributive policy
support. \emph{Analyses of Social Issues and Public Policy},
\emph{23}(3), 668--693. \url{https://doi.org/10.1111/asap.12367}

\bibitem[\citeproctext]{ref-immergut_theoretical_1998}
Immergut, E. M. (1998). The {Theoretical Core} of the {New
Institutionalism}. \emph{Politics \& Society}, \emph{26}(1), 5--34.
\url{https://doi.org/10.1177/0032329298026001002}

\bibitem[\citeproctext]{ref-immergut_it_2020}
Immergut, E. M., \& Schneider, S. M. (2020). Is it unfair for the
affluent to be able to purchase {``better''} healthcare? {Existential}
standards and institutional norms in healthcare attitudes across 28
countries. \emph{Social Science \& Medicine}, \emph{267}, 113146.
\url{https://doi.org/10.1016/j.socscimed.2020.113146}

\bibitem[\citeproctext]{ref-jaime-castillo_social_2019}
Jaime-Castillo, A. M., \& Marqués-Perales, I. (2019). Social mobility
and demand for redistribution in {Europe}: A comparative analysis.
\emph{The British Journal of Sociology}, \emph{70}(1), 138--165.
\url{https://doi.org/10.1111/1468-4446.12363}

\bibitem[\citeproctext]{ref-janmaat_subjective_2013}
Janmaat, J. G. (2013). Subjective {Inequality}: A {Review} of
{International Comparative Studies} on {People}'s {Views} about
{Inequality}. \emph{European Journal of Sociology}, \emph{54}(3),
357--389. \url{https://doi.org/10.1017/S0003975613000209}

\bibitem[\citeproctext]{ref-jasso_justice_1978}
Jasso, Guillermina. (1978). On the {Justice} of {Earnings}: {A New
Specification} of the {Justice Evaluation Function}. \emph{American
Journal of Sociology}, \emph{83}(6), 1398--1419.
\url{https://doi.org/10.1086/226706}

\bibitem[\citeproctext]{ref-jasso_distributive_2016}
Jasso, Guillermina, Törnblom, K. Y., \& Sabbagh, C. (2016). Distributive
{Justice}. In C. Sabbagh \& M. Schmitt (Eds.), \emph{Handbook of {Social
Justice Theory} and {Research}} (pp. 201--218). New York, NY: Springer.
\url{https://doi.org/10.1007/978-1-4939-3216-0_11}

\bibitem[\citeproctext]{ref-jasso_gender_1999}
Jasso, G., \& Wegener, B. (1999). Gender and {Country Differences} in
the {Sense} of {Justice}: {Justice Evaluation}, {Gender Earnings Gap},
and {Earnings Functions} in {Thirteen Countries}. \emph{International
Journal of Comparative Sociology}, \emph{40}(1), 94--115.
\url{https://doi.org/10.1177/002071529904000106}

\bibitem[\citeproctext]{ref-kluegel_legitimation_1999}
Kluegel, J. R., Mason, D. S., \& Wegener, B. (1999). The {Legitimation}
of {Capitalism} in the {Postcommunist Transition}: {Public Opinion}
about {Market Justice}, 1991-1996. \emph{European Sociological Review},
\emph{15}(3), 251--283. Retrieved from
\url{https://www.jstor.org/stable/522731}

\bibitem[\citeproctext]{ref-kluegel_beliefs_1981}
Kluegel, J. R., \& Smith, E. R. (1981). Beliefs {About Stratification}.
\emph{Annual Review of Sociology}, 29--56.

\bibitem[\citeproctext]{ref-koos_moral_2019}
Koos, S., \& Sachweh, P. (2019). The moral economies of market
societies: Popular attitudes towards market competition, redistribution
and reciprocity in comparative perspective. \emph{Socio-Economic
Review}, \emph{17}(4), 793--821.
\url{https://doi.org/10.1093/ser/mwx045}

\bibitem[\citeproctext]{ref-lane_market_1986}
Lane, R. E. (1986). Market {Justice}, {Political Justice}.
\emph{American Political Science Review}, \emph{80}(2), 383--402.
\url{https://doi.org/10.2307/1958264}

\bibitem[\citeproctext]{ref-langsaether_explaining_2022}
Langsæther, P. E., Evans, G., \& O'Grady, T. (2022). Explaining the
{Relationship Between Class Position} and {Political Preferences}: {A
Long-Term Panel Analysis} of {Intra-Generational Class Mobility}.
\emph{British Journal of Political Science}, \emph{52}(2), 958--967.
\url{https://doi.org/10.1017/S0007123420000599}

\bibitem[\citeproctext]{ref-lee_fairness_2023}
Lee, J.-S., \& Stacey, M. (2023). Fairness perceptions of educational
inequality: The effects of self-interest and neoliberal orientations.
\emph{The Australian Educational Researcher}.
\url{https://doi.org/10.1007/s13384-023-00636-6}

\bibitem[\citeproctext]{ref-lenski_status_1954}
Lenski, G. E. (1954). Status {Crystallization}: {A Non-Vertical
Dimension} of {Social Status}. \emph{American Sociological Review},
\emph{19}(4), 405. \url{https://doi.org/10.2307/2087459}

\bibitem[\citeproctext]{ref-lindh_public_2015}
Lindh, A. (2015). Public {Opinion} against {Markets}? {Attitudes}
towards {Market Distribution} of {Social Services} -- {A Comparison} of
17 {Countries}. \emph{Social Policy \& Administration}, \emph{49}(7),
887--910. \url{https://doi.org/10.1111/spol.12105}

\bibitem[\citeproctext]{ref-lindh_bringing_2023}
Lindh, A., \& McCall, L. (2023). Bringing the market in: An expanded
framework for understanding popular responses to economic inequality.
\emph{Socio-Economic Review}, \emph{21}(2), 1035--1055.
\url{https://doi.org/10.1093/ser/mwac018}

\bibitem[\citeproctext]{ref-lopez-roldan_comparative_2021}
López-Roldán, P., \& Fachelli, S. (Eds.). (2021). \emph{Towards a
{Comparative Analysis} of {Social Inequalities} between {Europe} and
{Latin America}}. Cham: Springer International Publishing.
\url{https://doi.org/10.1007/978-3-030-48442-2}

\bibitem[\citeproctext]{ref-madariaga_three_2020}
Madariaga, A. (2020). The three pillars of neoliberalism: {Chile}'s
economic policy trajectory in comparative perspective.
\emph{Contemporary Politics}, \emph{26}(3), 308--329.
\url{https://doi.org/10.1080/13569775.2020.1735021}

\bibitem[\citeproctext]{ref-mau_inequality_2015}
Mau, S. (2015). \emph{Inequality, {Marketization} and the {Majority
Class}: {Why Did} the {European Middle Classes Accept Neo-Liberalism}?}
Houndmills: Palgrave Macmillan.

\bibitem[\citeproctext]{ref-mijs_paradox_2019}
Mijs, J. (2019). The paradox of inequality: Income inequality and belief
in meritocracy go hand in hand. \emph{Socio-Economic Review},
\emph{19}(1), 7--35. \url{https://doi.org/10.1093/ser/mwy051}

\bibitem[\citeproctext]{ref-mijs_belief_2022}
Mijs, J., Daenekindt, S., de Koster, W., \& van der Waal, J. (2022).
Belief in {Meritocracy Reexamined}: {Scrutinizing} the {Role} of
{Subjective Social Mobility}. \emph{Social Psychology Quarterly},
\emph{85}(2), 131--141. \url{https://doi.org/10.1177/01902725211063818}

\bibitem[\citeproctext]{ref-miller_selfserving_1975}
Miller, D. T., \& Ross, M. (1975). Self-serving biases in the
attribution of causality: {Fact} or fiction? \emph{Psychological
Bulletin}, \emph{82}(2), 213--225.
\url{https://doi.org/10.1037/h0076486}

\bibitem[\citeproctext]{ref-molina_its_2019}
Molina, M. D., Bucca, M., \& Macy, M. W. (2019). It's not just how the
game is played, it's whether you win or lose. \emph{SCIENCE ADVANCES}.

\bibitem[\citeproctext]{ref-otero_power_2024}
Otero, G., \& Mendoza, M. (2024). The {Power} of {Diversity}: {Class},
{Networks} and {Attitudes Towards Inequality}. \emph{Sociology},
\emph{58}(4), 851--876. \url{https://doi.org/10.1177/00380385231217625}

\bibitem[\citeproctext]{ref-polanyi_great_1975}
Polanyi, K. (1975). \emph{The great transformation} (Repr). New York,
NY: Octagon Books.

\bibitem[\citeproctext]{ref-prag_subjective_2021}
Präg, P., \& Gugushvili, A. (2021). Subjective social mobility and
health in {Germany}. \emph{European Societies}, \emph{23}(4), 464--486.
\url{https://doi.org/10.1080/14616696.2021.1887916}

\bibitem[\citeproctext]{ref-sandel_tyranny_2020}
Sandel, M. J. (2020). \emph{The tyranny of merit: {What}'s become of the
common good?} (First edition). New York: {Farrar, Straus and Giroux}.

\bibitem[\citeproctext]{ref-schmidt_experience_2011}
Schmidt, A. W. (2011). The experience of social mobility and the
formation of attitudes toward redistribution. In.

\bibitem[\citeproctext]{ref-shariff_income_2016}
Shariff, A. F., Wiwad, D., \& Aknin, L. B. (2016). Income {Mobility
Breeds Tolerance} for {Income Inequality}: {Cross-National} and
{Experimental Evidence}. \emph{Perspectives on Psychological Science},
\emph{11}(3), 373--380. \url{https://doi.org/10.1177/1745691616635596}

\bibitem[\citeproctext]{ref-somma_no_2021}
Somma, N. M., Bargsted, M., Disi Pavlic, R., \& Medel, R. M. (2021). No
water in the oasis: The {Chilean Spring} of 2019--2020. \emph{Social
Movement Studies}, \emph{20}(4), 495--502.
\url{https://doi.org/10.1080/14742837.2020.1727737}

\bibitem[\citeproctext]{ref-sorokin_social_1927}
Sorokin, P. A. (1927). \emph{Social {Mobility}}. Harper \& Brothers.

\bibitem[\citeproctext]{ref-streeck_how_2016}
Streeck, W. (2016). \emph{How will capitalism end? Essays on a failing
system}. London: Verso.

\bibitem[\citeproctext]{ref-svallforsMoralEconomyClass2006a}
Svallfors, S. (2006). \emph{The {Moral Economy} of {Class}: {Class} and
{Attitudes} in {Comparative Perspective}}. Stanford University Press.

\bibitem[\citeproctext]{ref-svallfors_political_2007}
Svallfors, S. (Ed.). (2007). \emph{The {Political Sociology} of the
{Welfare State}: {Institutions}, {Social Cleavages}, and {Orientations}}
(1st ed.). Stanford University Press.
\url{https://doi.org/10.2307/j.ctvr0qv0q}

\bibitem[\citeproctext]{ref-tejero-peregrina_perceived_2025}
Tejero-Peregrina, L., Willis, G., Sánchez-Rodríguez, Á., \&
Rodríguez-Bailón, R. (2025). From {Perceived Economic Inequality} to
{Support} for {Redistribution}: {The Role} of {Meritocracy Perception}.
\emph{International Review of Social Psychology}, \emph{38}(1), 4.
\url{https://doi.org/10.5334/irsp.1013}

\bibitem[\citeproctext]{ref-torche_intergenerational_2014}
Torche, F. (2014). Intergenerational {Mobility} and {Inequality}: {The
Latin American Case}. \emph{Annual Review of Sociology}, \emph{40}(1),
619--642. \url{https://doi.org/10.1146/annurev-soc-071811-145521}

\bibitem[\citeproctext]{ref-vondemknesebeck_are_2016}
Von Dem Knesebeck, O., Vonneilich, N., \& Kim, T. J. (2016). Are health
care inequalities unfair? {A} study on public attitudes in 23 countries.
\emph{International Journal for Equity in Health}, \emph{15}(1), 61.
\url{https://doi.org/10.1186/s12939-016-0350-8}

\bibitem[\citeproctext]{ref-wen_does_2021}
Wen, F., \& Witteveen, D. (2021). Does perceived social mobility shape
attitudes toward government and family educational investment?
\emph{Social Science Research}, \emph{98}, 102579.
\url{https://doi.org/10.1016/j.ssresearch.2021.102579}

\bibitem[\citeproctext]{ref-wiederkehr_belief_2015}
Wiederkehr, V., Bonnot, V., Krauth-Gruber, S., \& Darnon, C. (2015).
Belief in school meritocracy as a system-justifying tool for low status
students. \emph{Frontiers in Psychology}, \emph{6}.

\bibitem[\citeproctext]{ref-wilson_role_2003}
Wilson, C. (2003). The {Role} of a {Merit Principle} in {Distributive
Justice}. \emph{The Journal of Ethics}, \emph{7}(3), 277--314.
\url{https://doi.org/10.1023/A:1024667228488}

\bibitem[\citeproctext]{ref-young_rise_1958}
Young, M. (1958). \emph{The rise of the meritocracy}. New Brunswick,
N.J., U.S.A: Transaction Publishers.

\end{CSLReferences}



\end{document}
