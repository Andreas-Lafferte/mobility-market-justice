% Options for packages loaded elsewhere
\PassOptionsToPackage{unicode}{hyperref}
\PassOptionsToPackage{hyphens}{url}
\PassOptionsToPackage{dvipsnames,svgnames,x11names}{xcolor}
%
\documentclass[
  12pt,
]{article}

\usepackage{amsmath,amssymb}
\usepackage{setspace}
\usepackage{iftex}
\ifPDFTeX
  \usepackage[T1]{fontenc}
  \usepackage[utf8]{inputenc}
  \usepackage{textcomp} % provide euro and other symbols
\else % if luatex or xetex
  \usepackage{unicode-math}
  \defaultfontfeatures{Scale=MatchLowercase}
  \defaultfontfeatures[\rmfamily]{Ligatures=TeX,Scale=1}
\fi
\usepackage{lmodern}
\ifPDFTeX\else  
    % xetex/luatex font selection
  \setmainfont[]{Times New Roman}
\fi
% Use upquote if available, for straight quotes in verbatim environments
\IfFileExists{upquote.sty}{\usepackage{upquote}}{}
\IfFileExists{microtype.sty}{% use microtype if available
  \usepackage[]{microtype}
  \UseMicrotypeSet[protrusion]{basicmath} % disable protrusion for tt fonts
}{}
\makeatletter
\@ifundefined{KOMAClassName}{% if non-KOMA class
  \IfFileExists{parskip.sty}{%
    \usepackage{parskip}
  }{% else
    \setlength{\parindent}{0pt}
    \setlength{\parskip}{6pt plus 2pt minus 1pt}}
}{% if KOMA class
  \KOMAoptions{parskip=half}}
\makeatother
\usepackage{xcolor}
\usepackage[margin=2cm]{geometry}
\setlength{\emergencystretch}{3em} % prevent overfull lines
\setcounter{secnumdepth}{5}
% Make \paragraph and \subparagraph free-standing
\ifx\paragraph\undefined\else
  \let\oldparagraph\paragraph
  \renewcommand{\paragraph}[1]{\oldparagraph{#1}\mbox{}}
\fi
\ifx\subparagraph\undefined\else
  \let\oldsubparagraph\subparagraph
  \renewcommand{\subparagraph}[1]{\oldsubparagraph{#1}\mbox{}}
\fi


\providecommand{\tightlist}{%
  \setlength{\itemsep}{0pt}\setlength{\parskip}{0pt}}\usepackage{longtable,booktabs,array}
\usepackage{calc} % for calculating minipage widths
% Correct order of tables after \paragraph or \subparagraph
\usepackage{etoolbox}
\makeatletter
\patchcmd\longtable{\par}{\if@noskipsec\mbox{}\fi\par}{}{}
\makeatother
% Allow footnotes in longtable head/foot
\IfFileExists{footnotehyper.sty}{\usepackage{footnotehyper}}{\usepackage{footnote}}
\makesavenoteenv{longtable}
\usepackage{graphicx}
\makeatletter
\def\maxwidth{\ifdim\Gin@nat@width>\linewidth\linewidth\else\Gin@nat@width\fi}
\def\maxheight{\ifdim\Gin@nat@height>\textheight\textheight\else\Gin@nat@height\fi}
\makeatother
% Scale images if necessary, so that they will not overflow the page
% margins by default, and it is still possible to overwrite the defaults
% using explicit options in \includegraphics[width, height, ...]{}
\setkeys{Gin}{width=\maxwidth,height=\maxheight,keepaspectratio}
% Set default figure placement to htbp
\makeatletter
\def\fps@figure{htbp}
\makeatother
% definitions for citeproc citations
\NewDocumentCommand\citeproctext{}{}
\NewDocumentCommand\citeproc{mm}{%
  \begingroup\def\citeproctext{#2}\cite{#1}\endgroup}
\makeatletter
 % allow citations to break across lines
 \let\@cite@ofmt\@firstofone
 % avoid brackets around text for \cite:
 \def\@biblabel#1{}
 \def\@cite#1#2{{#1\if@tempswa , #2\fi}}
\makeatother
\newlength{\cslhangindent}
\setlength{\cslhangindent}{1.5em}
\newlength{\csllabelwidth}
\setlength{\csllabelwidth}{3em}
\newenvironment{CSLReferences}[2] % #1 hanging-indent, #2 entry-spacing
 {\begin{list}{}{%
  \setlength{\itemindent}{0pt}
  \setlength{\leftmargin}{0pt}
  \setlength{\parsep}{0pt}
  % turn on hanging indent if param 1 is 1
  \ifodd #1
   \setlength{\leftmargin}{\cslhangindent}
   \setlength{\itemindent}{-1\cslhangindent}
  \fi
  % set entry spacing
  \setlength{\itemsep}{#2\baselineskip}}}
 {\end{list}}
\usepackage{calc}
\newcommand{\CSLBlock}[1]{\hfill\break\parbox[t]{\linewidth}{\strut\ignorespaces#1\strut}}
\newcommand{\CSLLeftMargin}[1]{\parbox[t]{\csllabelwidth}{\strut#1\strut}}
\newcommand{\CSLRightInline}[1]{\parbox[t]{\linewidth - \csllabelwidth}{\strut#1\strut}}
\newcommand{\CSLIndent}[1]{\hspace{\cslhangindent}#1}

\usepackage[noblocks]{authblk}
\renewcommand*{\Authsep}{, }
\renewcommand*{\Authand}{, }
\renewcommand*{\Authands}{, }
\renewcommand\Affilfont{\small}
\makeatletter
\@ifpackageloaded{caption}{}{\usepackage{caption}}
\AtBeginDocument{%
\ifdefined\contentsname
  \renewcommand*\contentsname{Table of contents}
\else
  \newcommand\contentsname{Table of contents}
\fi
\ifdefined\listfigurename
  \renewcommand*\listfigurename{List of Figures}
\else
  \newcommand\listfigurename{List of Figures}
\fi
\ifdefined\listtablename
  \renewcommand*\listtablename{List of Tables}
\else
  \newcommand\listtablename{List of Tables}
\fi
\ifdefined\figurename
  \renewcommand*\figurename{Figure}
\else
  \newcommand\figurename{Figure}
\fi
\ifdefined\tablename
  \renewcommand*\tablename{Table}
\else
  \newcommand\tablename{Table}
\fi
}
\@ifpackageloaded{float}{}{\usepackage{float}}
\floatstyle{ruled}
\@ifundefined{c@chapter}{\newfloat{codelisting}{h}{lop}}{\newfloat{codelisting}{h}{lop}[chapter]}
\floatname{codelisting}{Listing}
\newcommand*\listoflistings{\listof{codelisting}{List of Listings}}
\makeatother
\makeatletter
\makeatother
\makeatletter
\@ifpackageloaded{caption}{}{\usepackage{caption}}
\@ifpackageloaded{subcaption}{}{\usepackage{subcaption}}
\makeatother
\ifLuaTeX
  \usepackage{selnolig}  % disable illegal ligatures
\fi
\usepackage{bookmark}

\IfFileExists{xurl.sty}{\usepackage{xurl}}{} % add URL line breaks if available
\urlstyle{same} % disable monospaced font for URLs
\hypersetup{
  pdftitle={Preferences for market justice: the role of intergenerational social mobility and beliefs in meritocracy in Chile},
  pdfauthor={Andreas Laffert},
  colorlinks=true,
  linkcolor={blue},
  filecolor={Maroon},
  citecolor={Blue},
  urlcolor={Blue},
  pdfcreator={LaTeX via pandoc}}

\title{Preferences for market justice: the role of intergenerational
social mobility and beliefs in meritocracy in Chile}


  \author{Andreas Laffert}
            \affil{%
                  Instituto de Sociología, Pontificia Universidad
                  Católica de Chile
              }
      
\date{}
\begin{document}
\maketitle

\setstretch{1.15}
\section{Introduction}\label{introduction}

What makes market-driven inequalities appear legitimate in the eyes of
the public? Since the early 1980s, many countries have experienced a
widespread retreat from universal welfare programs and a shift toward
the privatization and commodification of public goods, welfare policies,
and social services (\citeproc{ref-gingrich_making_2011}{Gingrich,
2011}; \citeproc{ref-streeck_how_2016}{Streeck, 2016}). In Latin
America, as elsewhere, neoliberal reforms reshaped welfare-state
institutions by extending market logic into domains of social
reproduction that were traditionally governed by the state
(\citeproc{ref-arrizabalo_milagro_1995}{Arrizabalo, 1995};
\citeproc{ref-ferre_welfare_2023}{Ferre, 2023}). This transformation
reduced the role of public provision and increased the presence of
private actors in core social services
(\citeproc{ref-harvey_breve_2015}{Harvey, 2015}). Echoing Polanyi's
(\citeproc{ref-polanyi_great_1975}{1975}) insight that markets
constitute a distinct moral order, the institutional diffusion of market
rules has fostered a corresponding moral economy: a constellation of
norms and values concerning fair allocation, embedded in institutions
and shaping individual subjectivities
(\citeproc{ref-mau_inequality_2015}{Mau, 2015};
\citeproc{ref-svallforsMoralEconomyClass2006a}{Svallfors, 2006}). Within
this framework, a growing body of research examines the extent to which,
and the mechanisms by which, citizens consider it fair that access to
services like health care, pensions, and education be governed by
market-based criteria---a phenomenon known as \emph{market justice
preferences} (\citeproc{ref-busemeyer_skills_2014}{Busemeyer, 2014};
\citeproc{ref-castillo_socialization_2024}{Castillo et al., 2024};
\citeproc{ref-immergut_it_2020}{Immergut \& Schneider, 2020};
\citeproc{ref-koos_moral_2019}{Koos \& Sachweh, 2019};
\citeproc{ref-lindh_public_2015}{Lindh, 2015};
\citeproc{ref-lindh_bringing_2023}{Lindh \& McCall, 2023}).
Understanding these preferences is crucial, as they contribute to
legitimizing economic inequality by framing it as the fair result of
individual responsibility and limited state intervention
(\citeproc{ref-mau_inequality_2015}{Mau, 2015}).

Existing literature shows that market justice preferences are shaped by
both the economic and institutional context and individuals' positions
within social stratification. Grounded in the notion that institutions
influence individual attitudes
(\citeproc{ref-jacksonActorsInstitutions2010}{Jackson, 2010}), studies
find that countries with stronger public provision or more expansive
welfare states exhibit lower levels of market justice preferences
(\citeproc{ref-busemeyer_skills_2014}{Busemeyer, 2014};
\citeproc{ref-immergut_it_2020}{Immergut \& Schneider, 2020}), while
more privatized contexts show stronger support for market-based criteria
(\citeproc{ref-castillo_socialization_2024}{Castillo et al., 2024};
\citeproc{ref-lindh_public_2015}{Lindh, 2015}). In such contexts, market
justice preferences tend to rise as individuals ``ascend'' the social
structure, with those in more privileged positions in terms of class,
education, and income holding stronger preferences for market-based
solutions compared to those in more disadvantaged or at-risk positions
(\citeproc{ref-castillo_socialization_2024}{Castillo et al., 2024};
\citeproc{ref-immergut_it_2020}{Immergut \& Schneider, 2020};
\citeproc{ref-lee_fairness_2023}{Lee \& Stacey, 2023};
\citeproc{ref-lindh_public_2015}{Lindh, 2015};
\citeproc{ref-otero_power_2024}{Otero \& Mendoza, 2024};
\citeproc{ref-svallfors_political_2007}{Svallfors, 2007};
\citeproc{ref-vondemknesebeck_are_2016}{Von Dem Knesebeck et al.,
2016}).

Market justice preferences are shaped not only by objective
socioeconomic conditions but also by popular beliefs about inequality.
Among these, meritocracy is a key normative principle underpinning
market-based distributive preferences
(\citeproc{ref-mau_inequality_2015}{Mau, 2015}). It frames inequality as
inevitable but justifiable through effort and talent
(\citeproc{ref-davis_principles_2001}{Davis \& Moore, 2001};
\citeproc{ref-young_rise_1958}{Young, 1958}). Studies show that
individuals with stronger meritocratic beliefs tend to perceive less
inequality and legitimize it by attributing economic differences to
personal achievement (\citeproc{ref-batruch_belief_2023}{Batruch et al.,
2023}; \citeproc{ref-mijs_paradox_2019}{Mijs, 2019};
\citeproc{ref-wilson_role_2003}{Wilson, 2003}). In highly unequal
societies where access to services is largely governed by market logic,
such beliefs play a critical role in normalizing inequality. Recent
evidence from Chile shows that students who believe effort and talent
are rewarded in their country express stronger preferences for
market-based access to healthcare, pensions, and education
(\citeproc{ref-castillo_socialization_2024}{Castillo et al., 2024}).

Although it is clear that one's social position influences market
preferences, the question of how upward or downward mobility within the
social structure affects these preferences remains unanswered. This
question is far from trivial, especially in Latin America, where many
have experienced various forms of mobility amid high economic inequality
and deep welfare privatization (\citeproc{ref-ferre_welfare_2023}{Ferre,
2023}; \citeproc{ref-lopez-roldan_comparative_2021}{López-Roldán \&
Fachelli, 2021}; \citeproc{ref-torche_intergenerational_2014}{Torche,
2014}). Social origins and destinations affect attitudes toward
inequality in distinct ways (\citeproc{ref-day_movin_2017}{Day \& Fiske,
2017}; \citeproc{ref-gugushvili_intergenerational_2016}{Gugushvili,
2016b}, \citeproc{ref-gugushvili_subjective_2017}{2017};
\citeproc{ref-jaime-castillo_social_2019}{Jaime-Castillo \&
Marqués-Perales, 2019}; \citeproc{ref-mijs_belief_2022}{Mijs et al.,
2022}; \citeproc{ref-wen_does_2021}{Wen \& Witteveen, 2021}), while
movement between these positions exposes individuals to different
experiences and mechanisms that shape their views on what is fair
(\citeproc{ref-gugushvili_trends_2014}{Gugushvili, 2014};
\citeproc{ref-mau_inequality_2015}{Mau, 2015}). Building on this
research, examining the effects of social mobility on market justice
preferences can help to illuminate how inequalities in access to social
services are justified among individuals who have experienced, or not,
changes in their social standing, and what are the normative mechanisms
that guide this justification (\citeproc{ref-mau_inequality_2015}{Mau,
2015}).

Beyond their isolated effects, social mobility and meritocratic beliefs
interact in complex ways to shape market justice preferences. A key
mechanism proposed in the literature to explain how mobility influences
distributive justice preferences is the psychological process of
self-serving bias in causal attribution
(\citeproc{ref-gugushvili_intergenerational_2016c}{Gugushvili, 2016a};
\citeproc{ref-schmidt_experience_2011}{Schmidt, 2011}). This bias
suggests that individuals attribute failures---such as downward
mobility---to external factors, while crediting successes---such as
upward mobility---to their own merit and effort
(\citeproc{ref-miller_selfserving_1975}{Miller \& Ross, 1975}). Those
who experience upward mobility tend to view their social position as
earned, making them more likely to believe that individuals are
responsible for their own success or failure. Research shows that upward
mobility is associated with weaker preferences for redistribution
(\citeproc{ref-alesina_intergenerational_2018}{Alesina et al., 2018};
\citeproc{ref-gugushvili_intergenerational_2016c}{Gugushvili, 2016a};
\citeproc{ref-jaime-castillo_social_2019}{Jaime-Castillo \&
Marqués-Perales, 2019}; \citeproc{ref-schmidt_experience_2011}{Schmidt,
2011}) and stronger legitimacy of income inequality
(\citeproc{ref-shariff_income_2016}{Shariff et al., 2016}). In contrast,
individuals who experience downward mobility tend to blame structural
factors like inequality and are more supportive of redistribution while
rejecting merit-based explanations
(\citeproc{ref-gugushvili_trends_2014}{Gugushvili, 2014}). Taken
together, I argue that meritocratic beliefs may reflect---or
reinforce---this self-serving attribution mechanism by legitimizing
one's social status as the outcome of personal merit, closely tied to
attribution bias.

Against this background, this article pursues two main objectives:
first, to analyze the extent to which intergenerational social mobility
influences market justice preferences regarding healthcare, pensions,
and education; and second, to examine how meritocratic beliefs may
moderate this relationship. Building on a theoretical framework that
emphasizes how neoliberal transformations---particularly through the
privatization and commodification of key areas of social
reproduction---have profoundly reshaped processes of subject formation
(\citeproc{ref-mau_inequality_2015}{Mau, 2015}), the central argument is
that upward mobility increases support for market justice preferences,
while downward mobility decreases it. Moreover, meritocratic beliefs are
expected to moderate this relationship by reflecting a self-serving
attribution bias, whereby individuals justify their social position in
terms of personal merit.

This study focuses on Chile, a particularly intriguing case for
examining market justice preferences. Despite significant economic
growth and poverty reduction, Chile has some of the highest levels of
inequality in Latin America and among OECD countries
(\citeproc{ref-chancel_world_2022}{Chancel et al., 2022};
\citeproc{ref-flores_top_2020}{Flores et al., 2020}). This inequality
coexists with short-range upward mobility among lower-class segments
moving into middle strata, though strong barriers remain to reaching
higher positions (\citeproc{ref-espinoza_estratificacion_2013}{Espinoza
et al., 2013}; \citeproc{ref-torche_intergenerational_2014}{Torche,
2014}). What makes Chile especially salient is that much of this
inequality is rooted in deep neoliberal reforms that institutionalized
the privatization and commodification of key social sectors
(\citeproc{ref-madariaga_three_2020}{Madariaga, 2020}). Introduced
during the dictatorship (1973--1989) and expanded in democracy, these
reforms enabled the unprecedented emergence of markets in health,
pensions, and education, with provision segmented by individuals'
ability to pay and supported by public subsidies
(\citeproc{ref-boccardo_30_2020}{Boccardo, 2020}). In parallel---and
despite waves of protest against inequality and commodification from
2006 to 2019 (\citeproc{ref-somma_no_2021}{Somma et al.,
2021})---Chilean subjectivities have been increasingly shaped by
neoliberal discourses and market logics, influencing their attitudes
toward inequality and welfare distribution
(\citeproc{ref-arteagaaguirre_politicas_2015}{Arteaga Aguirre \& Iñigo
Valderrama, 2015}; \citeproc{ref-castillo_socialization_2024}{Castillo
et al., 2024}).

In this context, the questions that guide this research are as follows:

\begin{enumerate}
\def\labelenumi{(\arabic{enumi})}
\tightlist
\item
  To what extent does intergenerational social mobility influence market
  justice preferences regarding healthcare, pensions, and education in
  Chile?
\item
  How do meritocratic beliefs condition or moderate this relationship in
  the Chilean context?
\end{enumerate}

To address these questions, this study draws on large-scale,
representative survey data collected in 2018 from the urban Chilean
population aged 18 to 75 (n = 2,983). The next section outlines the
theoretical framework linking market justice preferences, social
mobility, and meritocratic beliefs, and proposes a set of hypotheses.
This is followed by a description of the data, variables, and analytical
strategy. The final sections present the empirical findings, offer an
interpretation of the results, and conclude with a discussion of their
implications.

\section{Theoretical and empirical
background}\label{theoretical-and-empirical-background}

\subsection{Market justice
preferences}\label{market-justice-preferences}

\subsection{Social mobility}\label{social-mobility}

\subsection{Meritocracy}\label{meritocracy}

\subsection{The Chilean context}\label{the-chilean-context}

\section{Data, variables and methods}\label{data-variables-and-methods}

\subsection{Data}\label{data}

This study draws on data from the Chilean Longitudinal Social Survey
(ELSOC) of the Center for Social Conflict and Cohesion Studies (COES).
The survey is an annual panel collected from 2016 to 2023, comprising
two independent samples (original and refreshment), featuring permanent
and rotating questionnaire modules. It evaluates how individuals think,
feel, and behave regarding conflict and social cohesion in Chile. The
sampling design is complex, probabilistic, clustered, multistage, and
stratified by city size, targeting men and women aged 18--75 who
habitually reside in private dwellings in urban areas spanning 40 cities
(92 municipalities, 13 regions). This analysis focuses on 2023, the most
recent wave, which includes 1,737 respondents in the original sample and
989 in the refreshment sample. Further details on sampling, attrition,
and weighting can be found at https://coes.cl/encuesta-panel/, and the
dataset is publicly available at
https://dataverse.harvard.edu/dataverse/elsoc.

\subsection{Variables}\label{variables}

\emph{Outcome variables}

\textbf{Market justice preferences}: The outcome variables in this study
are market justice preferences. This construct is operationalized
through three variables that address the degree of justification
regarding whether access to social services in healthcare, pensions, and
education should be income conditional. Specifically, the justification
of inequality in healthcare is assessed by the question: ``Is it fair in
Chile that people with higher incomes can access better healthcare than
people with lower incomes?'' The same question is asked for pensions and
education. In all cases, respondents indicate their level of agreement
on a five-point Likert scale ranging from 1 (``strongly disagree'') to 5
(``strongly agree''). Although these items theoretically reflect the
same underlying concept of market justice
(\citeproc{ref-castillo_socialization_2024}{Castillo et al., 2024};
\citeproc{ref-lindh_public_2015}{Lindh, 2015}), they are analyzed
separately because healthcare, pensions, and education may prompt
distinct response patterns
(\citeproc{ref-busemeyer_skills_2014}{Busemeyer, 2014};
\citeproc{ref-immergut_it_2020}{Immergut \& Schneider, 2020};
\citeproc{ref-lee_fairness_2023}{Lee \& Stacey, 2023}).

\emph{Independent variables}

\textbf{Social mobility}: Intergenerational social mobility is treated
as an exposure indicating whether respondents occupy a different class
position from their fathers. Following Breen and Ermisch's
(\citeproc{ref-breen_effects_2024}{2024}) framework for estimating
causal effects of mobility, the variable is constructed in two stages.

\begin{enumerate}
\def\labelenumi{\arabic{enumi}.}
\item
  Class assignment. Using the International Socio-Economic Index of
  Occupational Status (ISEI) derived from three-digit ISCO-08 codes,
  both paternal (origin) and respondent (destination) occupations are
  grouped into three strata:

  \begin{itemize}
  \tightlist
  \item
    Upper (codes 100--299: managers, directors, professionals)
  \item
    Middle (codes 300--499: skilled white‑collar workers)
  \item
    Lower (codes 500 and above: manual, semi‑skilled, or unskilled
    workers)
  \end{itemize}
\item
  Propensity-score estimation. To model mobility as a treatment---i.e.,
  a transition from origin \emph{j} to destination \emph{k}---I estimate
  multinomial logit propensity scores using covariates that influence an
  individual's likelihood of moving between strata: (a) father's
  educational level, (b) presence of both parents at age 15, (c)
  nationality, (d) age, (e) sex, and (f) ethnicity.
\end{enumerate}

These propensity scores are subsequently employed to adjust for
selection into mobility when assessing its effect on market justice
preferences.

\textbf{Meritocracy}: Meritocratic perception is operationalized through
two components: one addressing effort and another focusing on talent
(\citeproc{ref-young_rise_1958}{Young, 1958}). The item used to gauge
effort is: ``In Chile, people are rewarded for their efforts,'' while
the item for talent is: ``In Chile, people are rewarded for their
intelligence and skills.'' In both cases, respondents indicate their
level of agreement on a five-point Likert scale, ranging from 1
(``strongly disagree'') to 5 (``strongly agree'').

\emph{Controls}

Sociodemographic and attitudinal variables are included to control for
potential composition effects in the population. In terms of
sociodemographic characteristics, I incorporate per capita household
income quantile, educational level (1=Primary or below, 2=Secondary,
3=Technical, 4=University or above), age (in years), and sex (1=Male,
2=Female), which have been shown to significantly influence market
justice preferences (\citeproc{ref-castillo_socialization_2024}{Castillo
et al., 2024}; \citeproc{ref-lindh_public_2015}{Lindh, 2015}). Regarding
attitudinal variables, I include political identification (1=Left,
2=Center, 3=Right, 4=No identification) and subjective social status
(ranging from 1 to 10) because they may confound the relationship
between market justice preferences and social mobility and meritocracy.

\subsection{Methods}\label{methods}

Following Breen and Ermisch's (\citeproc{ref-breen_effects_2024}{2024})
strategy for estimating the causal effect of social mobility on market
justice preferences, I employ linear regression models with
inverse-probability weights (IPW) for mobility conditional on class of
origin. This approach allows estimation of the average causal effect of
moving from an origin class \emph{j} to a destination class \emph{k} by
comparing individuals who actually reach \emph{k} with their
counterfactual outcome had they instead moved to an alternative
destination \emph{k′} (\citeproc{ref-breen_effects_2024}{Breen \&
Ermisch, 2024}). The resulting estimand is the average treatment effect
on the treated (ATT).

All the analyses will be conducted using R software.

\section{References}\label{references}

\phantomsection\label{refs}
\begin{CSLReferences}{1}{0}
\bibitem[\citeproctext]{ref-alesina_intergenerational_2018}
Alesina, A., Stantcheva, S., \& Teso, E. (2018). Intergenerational
{Mobility} and {Preferences} for {Redistribution}. \emph{American
Economic Review}, \emph{108}(2), 521--554.
\url{https://doi.org/10.1257/aer.20162015}

\bibitem[\citeproctext]{ref-arrizabalo_milagro_1995}
Arrizabalo, X. (1995). \emph{{Milagro o quimera: la econom{í}a chilena
durante la dictadura}}. Libros de la Catarata.

\bibitem[\citeproctext]{ref-arteagaaguirre_politicas_2015}
Arteaga Aguirre, C., \& Iñigo Valderrama, I. (2015). {Pol{í}ticas
sociales, modelo de desarrollo y subjetividad de grupos vulnerables en
Chile}.

\bibitem[\citeproctext]{ref-batruch_belief_2023}
Batruch, A., Jetten, J., Van De Werfhorst, H., Darnon, C., \& Butera, F.
(2023). Belief in {School Meritocracy} and the {Legitimization} of
{Social} and {Income Inequality}. \emph{Social Psychological and
Personality Science}, \emph{14}(5), 621--635.
\url{https://doi.org/10.1177/19485506221111017}

\bibitem[\citeproctext]{ref-boccardo_30_2020}
Boccardo, G. (2020). \emph{30 a{ñ}os de privatizaciones en {Chile}: Lo
que la pandemia revel{ó}} (Nodo XXI). Santiago.

\bibitem[\citeproctext]{ref-breen_effects_2024}
Breen, R., \& Ermisch, J. (2024). The {Effects} of {Social Mobility}.
\emph{Sociological Science}, \emph{11}, 467--488.
\url{https://doi.org/10.15195/v11.a17}

\bibitem[\citeproctext]{ref-busemeyer_skills_2014}
Busemeyer, M. R. (2014). \emph{Skills and {Inequality}: {Partisan
Politics} and the {Political Economy} of {Education Reforms} in {Western
Welfare States}}. Cambridge University Press.

\bibitem[\citeproctext]{ref-castillo_socialization_2024}
Castillo, J. C., Salgado, M., Carrasco, K., \& Laffert, A. (2024). The
{Socialization} of {Meritocracy} and {Market Justice Preferences} at
{School}. \emph{Societies}, \emph{14}(11), 214.
\url{https://doi.org/10.3390/soc14110214}

\bibitem[\citeproctext]{ref-chancel_world_2022}
Chancel, L., Piketty, T., Saez, E., \& Zucman, G. (2022). World
inequality report 2022.
https://bibliotecadigital.ccb.org.co/handle/11520/27510.

\bibitem[\citeproctext]{ref-davis_principles_2001}
Davis, K., \& Moore, W. E. (2001). Some {Principles} of
{Stratification}. In \emph{Social {Stratification}, {Class}, {Race}, and
{Gender} in {Sociological Perspective}, {Second Edition}} (2nd ed.).
Routledge.

\bibitem[\citeproctext]{ref-day_movin_2017}
Day, M. V., \& Fiske, S. T. (2017). Movin' on {Up}? {How Perceptions} of
{Social Mobility Affect Our Willingness} to {Defend} the {System}.
\emph{Social Psychological and Personality Science}, \emph{8}(3),
267--274. \url{https://doi.org/10.1177/1948550616678454}

\bibitem[\citeproctext]{ref-espinoza_estratificacion_2013}
Espinoza, V., Barozet, E., \& Méndez, M. L. (2013). {Estratificaci{ó}n y
movilidad social bajo un}.

\bibitem[\citeproctext]{ref-ferre_welfare_2023}
Ferre, J. C. (2023). Welfare regimes in twenty-first-century {Latin
America}. \emph{Journal of International and Comparative Social Policy},
\emph{39}(2), 101--127. \url{https://doi.org/10.1017/ics.2023.16}

\bibitem[\citeproctext]{ref-flores_top_2020}
Flores, I., Sanhueza, C., Atria, J., \& Mayer, R. (2020). Top {Incomes}
in {Chile}: {A Historical Perspective} on {Income Inequality},
1964--2017. \emph{Review of Income and Wealth}, \emph{66}(4), 850--874.
\url{https://doi.org/10.1111/roiw.12441}

\bibitem[\citeproctext]{ref-gingrich_making_2011}
Gingrich, J. R. (2011). \emph{Making {Markets} in the {Welfare State}:
{The Politics} of {Varying Market Reforms}} (1st ed.). Cambridge
University Press. \url{https://doi.org/10.1017/CBO9780511791529}

\bibitem[\citeproctext]{ref-gugushvili_trends_2014}
Gugushvili, A. (2014). Trends, {Covariates} and {Consequences} of
{Intergenerational Social Mobility} in {Post- Socialist Societies}.

\bibitem[\citeproctext]{ref-gugushvili_intergenerational_2016c}
Gugushvili, A. (2016a). Intergenerational objective and subjective
mobility and attitudes towards income differences: Evidence from
transition societies. \emph{Journal of International and Comparative
Social Policy}, \emph{32}(3), 199--219.
\url{https://doi.org/10.1080/21699763.2016.1206482}

\bibitem[\citeproctext]{ref-gugushvili_intergenerational_2016}
Gugushvili, A. (2016b). Intergenerational {Social Mobility} and {Popular
Explanations} of {Poverty}: {A Comparative Perspective}. \emph{Social
Justice Research}, \emph{29}(4), 402--428.
\url{https://doi.org/10.1007/s11211-016-0275-9}

\bibitem[\citeproctext]{ref-gugushvili_subjective_2017}
Gugushvili, A. (2017). Subjective {Intergenerational Mobility} and
{Support} for {Welfare State Programmes}.

\bibitem[\citeproctext]{ref-harvey_breve_2015}
Harvey, D. (2015). \emph{{Breve historia del neoliberalismo}}. Madrid
(Espa{ñ}a): Ediciones Akal.

\bibitem[\citeproctext]{ref-immergut_it_2020}
Immergut, E. M., \& Schneider, S. M. (2020). Is it unfair for the
affluent to be able to purchase {``better''} healthcare? {Existential}
standards and institutional norms in healthcare attitudes across 28
countries. \emph{Social Science \& Medicine}, \emph{267}, 113146.
\url{https://doi.org/10.1016/j.socscimed.2020.113146}

\bibitem[\citeproctext]{ref-jacksonActorsInstitutions2010}
Jackson, G. (2010). Actors and institutions. In G. Morgan, J. L.
Campbell, C. Crouch, O. K. Pedersen, \& R. Whitley (Eds.), \emph{The
{Oxford Handbook} of {Comparative Institutional Analysis}} (1st ed.).
Oxford University Press.
\url{https://doi.org/10.1093/oxfordhb/9780199233762.001.0001}

\bibitem[\citeproctext]{ref-jaime-castillo_social_2019}
Jaime-Castillo, A. M., \& Marqués-Perales, I. (2019). Social mobility
and demand for redistribution in {Europe}: A comparative analysis.
\emph{The British Journal of Sociology}, \emph{70}(1), 138--165.
\url{https://doi.org/10.1111/1468-4446.12363}

\bibitem[\citeproctext]{ref-koos_moral_2019}
Koos, S., \& Sachweh, P. (2019). The moral economies of market
societies: Popular attitudes towards market competition, redistribution
and reciprocity in comparative perspective. \emph{Socio-Economic
Review}, \emph{17}(4), 793--821.
\url{https://doi.org/10.1093/ser/mwx045}

\bibitem[\citeproctext]{ref-lee_fairness_2023}
Lee, J.-S., \& Stacey, M. (2023). Fairness perceptions of educational
inequality: The effects of self-interest and neoliberal orientations.
\emph{The Australian Educational Researcher}.
\url{https://doi.org/10.1007/s13384-023-00636-6}

\bibitem[\citeproctext]{ref-lindh_public_2015}
Lindh, A. (2015). Public {Opinion} against {Markets}? {Attitudes}
towards {Market Distribution} of {Social Services} -- {A Comparison} of
17 {Countries}. \emph{Social Policy \& Administration}, \emph{49}(7),
887--910. \url{https://doi.org/10.1111/spol.12105}

\bibitem[\citeproctext]{ref-lindh_bringing_2023}
Lindh, A., \& McCall, L. (2023). Bringing the market in: An expanded
framework for understanding popular responses to economic inequality.
\emph{Socio-Economic Review}, \emph{21}(2), 1035--1055.
\url{https://doi.org/10.1093/ser/mwac018}

\bibitem[\citeproctext]{ref-lopez-roldan_comparative_2021}
López-Roldán, P., \& Fachelli, S. (Eds.). (2021). \emph{Towards a
{Comparative Analysis} of {Social Inequalities} between {Europe} and
{Latin America}}. Cham: Springer International Publishing.
\url{https://doi.org/10.1007/978-3-030-48442-2}

\bibitem[\citeproctext]{ref-madariaga_three_2020}
Madariaga, A. (2020). The three pillars of neoliberalism: {Chile}'s
economic policy trajectory in comparative perspective.
\emph{Contemporary Politics}, \emph{26}(3), 308--329.
\url{https://doi.org/10.1080/13569775.2020.1735021}

\bibitem[\citeproctext]{ref-mau_inequality_2015}
Mau, S. (2015). \emph{Inequality, {Marketization} and the {Majority
Class}: {Why Did} the {European Middle Classes Accept Neo-Liberalism}?}
Houndmills: Palgrave Macmillan.

\bibitem[\citeproctext]{ref-mijs_paradox_2019}
Mijs, J. (2019). The paradox of inequality: Income inequality and belief
in meritocracy go hand in hand. \emph{Socio-Economic Review},
\emph{19}(1), 7--35. \url{https://doi.org/10.1093/ser/mwy051}

\bibitem[\citeproctext]{ref-mijs_belief_2022}
Mijs, J., Daenekindt, S., de Koster, W., \& van der Waal, J. (2022).
Belief in {Meritocracy Reexamined}: {Scrutinizing} the {Role} of
{Subjective Social Mobility}. \emph{Social Psychology Quarterly},
\emph{85}(2), 131--141. \url{https://doi.org/10.1177/01902725211063818}

\bibitem[\citeproctext]{ref-miller_selfserving_1975}
Miller, D. T., \& Ross, M. (1975). Self-serving biases in the
attribution of causality: {Fact} or fiction? \emph{Psychological
Bulletin}, \emph{82}(2), 213--225.
\url{https://doi.org/10.1037/h0076486}

\bibitem[\citeproctext]{ref-otero_power_2024}
Otero, G., \& Mendoza, M. (2024). The {Power} of {Diversity}: {Class},
{Networks} and {Attitudes Towards Inequality}. \emph{Sociology},
\emph{58}(4), 851--876. \url{https://doi.org/10.1177/00380385231217625}

\bibitem[\citeproctext]{ref-polanyi_great_1975}
Polanyi, K. (1975). \emph{The great transformation} (Repr). New York,
NY: Octagon Books.

\bibitem[\citeproctext]{ref-schmidt_experience_2011}
Schmidt, A. W. (2011). The experience of social mobility and the
formation of attitudes toward redistribution. In.

\bibitem[\citeproctext]{ref-shariff_income_2016}
Shariff, A. F., Wiwad, D., \& Aknin, L. B. (2016). Income {Mobility
Breeds Tolerance} for {Income Inequality}: {Cross-National} and
{Experimental Evidence}. \emph{Perspectives on Psychological Science},
\emph{11}(3), 373--380. \url{https://doi.org/10.1177/1745691616635596}

\bibitem[\citeproctext]{ref-somma_no_2021}
Somma, N. M., Bargsted, M., Disi Pavlic, R., \& Medel, R. M. (2021). No
water in the oasis: The {Chilean Spring} of 2019--2020. \emph{Social
Movement Studies}, \emph{20}(4), 495--502.
\url{https://doi.org/10.1080/14742837.2020.1727737}

\bibitem[\citeproctext]{ref-streeck_how_2016}
Streeck, W. (2016). \emph{How will capitalism end? Essays on a failing
system}. London: Verso.

\bibitem[\citeproctext]{ref-svallforsMoralEconomyClass2006a}
Svallfors, S. (2006). \emph{The {Moral Economy} of {Class}: {Class} and
{Attitudes} in {Comparative Perspective}}. Stanford University Press.

\bibitem[\citeproctext]{ref-svallfors_political_2007}
Svallfors, S. (Ed.). (2007). \emph{The {Political Sociology} of the
{Welfare State}: {Institutions}, {Social Cleavages}, and {Orientations}}
(1st ed.). Stanford University Press.
\url{https://doi.org/10.2307/j.ctvr0qv0q}

\bibitem[\citeproctext]{ref-torche_intergenerational_2014}
Torche, F. (2014). Intergenerational {Mobility} and {Inequality}: {The
Latin American Case}. \emph{Annual Review of Sociology}, \emph{40}(1),
619--642. \url{https://doi.org/10.1146/annurev-soc-071811-145521}

\bibitem[\citeproctext]{ref-vondemknesebeck_are_2016}
Von Dem Knesebeck, O., Vonneilich, N., \& Kim, T. J. (2016). Are health
care inequalities unfair? {A} study on public attitudes in 23 countries.
\emph{International Journal for Equity in Health}, \emph{15}(1), 61.
\url{https://doi.org/10.1186/s12939-016-0350-8}

\bibitem[\citeproctext]{ref-wen_does_2021}
Wen, F., \& Witteveen, D. (2021). Does perceived social mobility shape
attitudes toward government and family educational investment?
\emph{Social Science Research}, \emph{98}, 102579.
\url{https://doi.org/10.1016/j.ssresearch.2021.102579}

\bibitem[\citeproctext]{ref-wilson_role_2003}
Wilson, C. (2003). The {Role} of a {Merit Principle} in {Distributive
Justice}. \emph{The Journal of Ethics}, \emph{7}(3), 277--314.
\url{https://doi.org/10.1023/A:1024667228488}

\bibitem[\citeproctext]{ref-young_rise_1958}
Young, M. (1958). \emph{The rise of the meritocracy}. New Brunswick,
N.J., U.S.A: Transaction Publishers.

\end{CSLReferences}



\end{document}
