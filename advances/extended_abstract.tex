% Options for packages loaded elsewhere
\PassOptionsToPackage{unicode}{hyperref}
\PassOptionsToPackage{hyphens}{url}
\PassOptionsToPackage{dvipsnames,svgnames,x11names}{xcolor}
%
\documentclass[
  12pt,
]{article}

\usepackage{amsmath,amssymb}
\usepackage{setspace}
\usepackage{iftex}
\ifPDFTeX
  \usepackage[T1]{fontenc}
  \usepackage[utf8]{inputenc}
  \usepackage{textcomp} % provide euro and other symbols
\else % if luatex or xetex
  \usepackage{unicode-math}
  \defaultfontfeatures{Scale=MatchLowercase}
  \defaultfontfeatures[\rmfamily]{Ligatures=TeX,Scale=1}
\fi
\usepackage{lmodern}
\ifPDFTeX\else  
    % xetex/luatex font selection
  \setmainfont[]{Times New Roman}
\fi
% Use upquote if available, for straight quotes in verbatim environments
\IfFileExists{upquote.sty}{\usepackage{upquote}}{}
\IfFileExists{microtype.sty}{% use microtype if available
  \usepackage[]{microtype}
  \UseMicrotypeSet[protrusion]{basicmath} % disable protrusion for tt fonts
}{}
\makeatletter
\@ifundefined{KOMAClassName}{% if non-KOMA class
  \IfFileExists{parskip.sty}{%
    \usepackage{parskip}
  }{% else
    \setlength{\parindent}{0pt}
    \setlength{\parskip}{6pt plus 2pt minus 1pt}}
}{% if KOMA class
  \KOMAoptions{parskip=half}}
\makeatother
\usepackage{xcolor}
\usepackage[margin=2cm]{geometry}
\setlength{\emergencystretch}{3em} % prevent overfull lines
\setcounter{secnumdepth}{5}
% Make \paragraph and \subparagraph free-standing
\ifx\paragraph\undefined\else
  \let\oldparagraph\paragraph
  \renewcommand{\paragraph}[1]{\oldparagraph{#1}\mbox{}}
\fi
\ifx\subparagraph\undefined\else
  \let\oldsubparagraph\subparagraph
  \renewcommand{\subparagraph}[1]{\oldsubparagraph{#1}\mbox{}}
\fi


\providecommand{\tightlist}{%
  \setlength{\itemsep}{0pt}\setlength{\parskip}{0pt}}\usepackage{longtable,booktabs,array}
\usepackage{calc} % for calculating minipage widths
% Correct order of tables after \paragraph or \subparagraph
\usepackage{etoolbox}
\makeatletter
\patchcmd\longtable{\par}{\if@noskipsec\mbox{}\fi\par}{}{}
\makeatother
% Allow footnotes in longtable head/foot
\IfFileExists{footnotehyper.sty}{\usepackage{footnotehyper}}{\usepackage{footnote}}
\makesavenoteenv{longtable}
\usepackage{graphicx}
\makeatletter
\def\maxwidth{\ifdim\Gin@nat@width>\linewidth\linewidth\else\Gin@nat@width\fi}
\def\maxheight{\ifdim\Gin@nat@height>\textheight\textheight\else\Gin@nat@height\fi}
\makeatother
% Scale images if necessary, so that they will not overflow the page
% margins by default, and it is still possible to overwrite the defaults
% using explicit options in \includegraphics[width, height, ...]{}
\setkeys{Gin}{width=\maxwidth,height=\maxheight,keepaspectratio}
% Set default figure placement to htbp
\makeatletter
\def\fps@figure{htbp}
\makeatother

\usepackage[noblocks]{authblk}
\renewcommand*{\Authsep}{, }
\renewcommand*{\Authand}{, }
\renewcommand*{\Authands}{, }
\renewcommand\Affilfont{\small}
\makeatletter
\@ifpackageloaded{caption}{}{\usepackage{caption}}
\AtBeginDocument{%
\ifdefined\contentsname
  \renewcommand*\contentsname{Table of contents}
\else
  \newcommand\contentsname{Table of contents}
\fi
\ifdefined\listfigurename
  \renewcommand*\listfigurename{List of Figures}
\else
  \newcommand\listfigurename{List of Figures}
\fi
\ifdefined\listtablename
  \renewcommand*\listtablename{List of Tables}
\else
  \newcommand\listtablename{List of Tables}
\fi
\ifdefined\figurename
  \renewcommand*\figurename{Figure}
\else
  \newcommand\figurename{Figure}
\fi
\ifdefined\tablename
  \renewcommand*\tablename{Table}
\else
  \newcommand\tablename{Table}
\fi
}
\@ifpackageloaded{float}{}{\usepackage{float}}
\floatstyle{ruled}
\@ifundefined{c@chapter}{\newfloat{codelisting}{h}{lop}}{\newfloat{codelisting}{h}{lop}[chapter]}
\floatname{codelisting}{Listing}
\newcommand*\listoflistings{\listof{codelisting}{List of Listings}}
\makeatother
\makeatletter
\makeatother
\makeatletter
\@ifpackageloaded{caption}{}{\usepackage{caption}}
\@ifpackageloaded{subcaption}{}{\usepackage{subcaption}}
\makeatother
\ifLuaTeX
  \usepackage{selnolig}  % disable illegal ligatures
\fi
\usepackage{bookmark}

\IfFileExists{xurl.sty}{\usepackage{xurl}}{} % add URL line breaks if available
\urlstyle{same} % disable monospaced font for URLs
\hypersetup{
  pdftitle={Preferences for mercantilization: the role of intergenerational social mobility and beliefs in meritocracy in Chile},
  pdfauthor={Andreas Laffert Tamayo},
  colorlinks=true,
  linkcolor={blue},
  filecolor={Maroon},
  citecolor={Blue},
  urlcolor={Blue},
  pdfcreator={LaTeX via pandoc}}

\title{Preferences for mercantilization: the role of intergenerational
social mobility and beliefs in meritocracy in Chile}


  \author{Andreas Laffert Tamayo}
            \affil{%
                  Instituto de Sociología, Pontificia Universidad
                  Católica de Chile
              }
      
\date{}
\begin{document}
\maketitle

\setstretch{1.15}
This article examines how intergenerational social mobility and
meritocratic beliefs influence public support for market-based
allocation of social services--- a phenomenon known as market justice
preferences. While previous studies have linked individuals' class
position and beliefs about merit to distributive attitudes, less is
known about how social mobility---especially in unequal and privatized
welfare contexts---shapes support for inequality in access to
healthcare, education, and pensions. This study addresses this gap by
analyzing whether upward or downward mobility affects market justice
preferences, and how this relationship is moderated by meritocratic
beliefs.

The analysis focuses on Chile, a paradigmatic case of neoliberal
transformation. Despite sustained economic growth and poverty reduction,
Chile remains one of the most unequal countries in Latin America and the
OECD. Since the 1980s, its welfare system has undergone deep
privatization and commodification, generating a segmented structure of
service provision and moral discourses that justify inequality. In this
context, meritocratic beliefs and mobility trajectories may interact to
shape how individuals evaluate fairness in access to core welfare
domains.

Using data from the 2023 wave of the Chilean Longitudinal Social Survey
(ELSOC), the study analyzes responses from a nationally representative
urban sample (n = 2,726). Market justice preferences are measured
through agreement with income-based inequalities in access to
healthcare, pensions, and education. Intergenerational mobility is
operationalized by comparing respondents' and their fathers'
occupational status using ISEI-based class strata (upper, middle,
lower). To estimate the causal effect of mobility, inverse-probability
weighted linear regressions are implemented following Breen and
Ermisch's framework, allowing for estimation of average treatment
effects on the treated (ATT). Meritocratic beliefs are measured via
perceived rewards for effort and talent.

Results show that upward mobility is associated with stronger market
justice preferences, while downward mobility reduces support for
market-based access. These effects are consistent across service domains
and robust to controls for income, education, age, gender, political
identification, and subjective social status. Moreover, the strength of
meritocratic beliefs moderates this relationship: upwardly mobile
individuals who believe effort and talent are rewarded in Chile express
the highest support for income-based inequalities, consistent with a
self-serving attribution mechanism. In contrast, downwardly mobile
individuals who reject meritocratic beliefs are more critical of
inequality in access.

The findings contribute to the literature by highlighting the dynamic
and interactive nature of distributive preferences. Rather than being
solely determined by fixed class position, attitudes toward market
justice are shaped by experiences of mobility and the normative
frameworks individuals use to interpret them. In contexts of high
inequality and welfare commodification, such as Chile, upward mobility
may legitimize neoliberal moral orders by reinforcing beliefs in
individual responsibility and merit-based allocation. These insights
underscore the importance of incorporating mobility trajectories and
belief systems into the analysis of welfare attitudes and inequality
legitimation.



\end{document}
