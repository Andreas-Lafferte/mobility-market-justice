% Options for packages loaded elsewhere
% Options for packages loaded elsewhere
\PassOptionsToPackage{unicode}{hyperref}
\PassOptionsToPackage{hyphens}{url}
\PassOptionsToPackage{dvipsnames,svgnames,x11names}{xcolor}
%
\documentclass[
  13pt,
]{article}
\usepackage{xcolor}
\usepackage[margin=2cm]{geometry}
\usepackage{amsmath,amssymb}
\setcounter{secnumdepth}{5}
\usepackage{iftex}
\ifPDFTeX
  \usepackage[T1]{fontenc}
  \usepackage[utf8]{inputenc}
  \usepackage{textcomp} % provide euro and other symbols
\else % if luatex or xetex
  \usepackage{unicode-math} % this also loads fontspec
  \defaultfontfeatures{Scale=MatchLowercase}
  \defaultfontfeatures[\rmfamily]{Ligatures=TeX,Scale=1}
\fi
\usepackage{lmodern}
\ifPDFTeX\else
  % xetex/luatex font selection
  \setmainfont[]{Times New Roman}
\fi
% Use upquote if available, for straight quotes in verbatim environments
\IfFileExists{upquote.sty}{\usepackage{upquote}}{}
\IfFileExists{microtype.sty}{% use microtype if available
  \usepackage[]{microtype}
  \UseMicrotypeSet[protrusion]{basicmath} % disable protrusion for tt fonts
}{}
\usepackage{setspace}
\makeatletter
\@ifundefined{KOMAClassName}{% if non-KOMA class
  \IfFileExists{parskip.sty}{%
    \usepackage{parskip}
  }{% else
    \setlength{\parindent}{0pt}
    \setlength{\parskip}{6pt plus 2pt minus 1pt}}
}{% if KOMA class
  \KOMAoptions{parskip=half}}
\makeatother
% Make \paragraph and \subparagraph free-standing
\makeatletter
\ifx\paragraph\undefined\else
  \let\oldparagraph\paragraph
  \renewcommand{\paragraph}{
    \@ifstar
      \xxxParagraphStar
      \xxxParagraphNoStar
  }
  \newcommand{\xxxParagraphStar}[1]{\oldparagraph*{#1}\mbox{}}
  \newcommand{\xxxParagraphNoStar}[1]{\oldparagraph{#1}\mbox{}}
\fi
\ifx\subparagraph\undefined\else
  \let\oldsubparagraph\subparagraph
  \renewcommand{\subparagraph}{
    \@ifstar
      \xxxSubParagraphStar
      \xxxSubParagraphNoStar
  }
  \newcommand{\xxxSubParagraphStar}[1]{\oldsubparagraph*{#1}\mbox{}}
  \newcommand{\xxxSubParagraphNoStar}[1]{\oldsubparagraph{#1}\mbox{}}
\fi
\makeatother


\usepackage{longtable,booktabs,array}
\usepackage{calc} % for calculating minipage widths
% Correct order of tables after \paragraph or \subparagraph
\usepackage{etoolbox}
\makeatletter
\patchcmd\longtable{\par}{\if@noskipsec\mbox{}\fi\par}{}{}
\makeatother
% Allow footnotes in longtable head/foot
\IfFileExists{footnotehyper.sty}{\usepackage{footnotehyper}}{\usepackage{footnote}}
\makesavenoteenv{longtable}
\usepackage{graphicx}
\makeatletter
\newsavebox\pandoc@box
\newcommand*\pandocbounded[1]{% scales image to fit in text height/width
  \sbox\pandoc@box{#1}%
  \Gscale@div\@tempa{\textheight}{\dimexpr\ht\pandoc@box+\dp\pandoc@box\relax}%
  \Gscale@div\@tempb{\linewidth}{\wd\pandoc@box}%
  \ifdim\@tempb\p@<\@tempa\p@\let\@tempa\@tempb\fi% select the smaller of both
  \ifdim\@tempa\p@<\p@\scalebox{\@tempa}{\usebox\pandoc@box}%
  \else\usebox{\pandoc@box}%
  \fi%
}
% Set default figure placement to htbp
\def\fps@figure{htbp}
\makeatother


% definitions for citeproc citations
\NewDocumentCommand\citeproctext{}{}
\NewDocumentCommand\citeproc{mm}{%
  \begingroup\def\citeproctext{#2}\cite{#1}\endgroup}
\makeatletter
 % allow citations to break across lines
 \let\@cite@ofmt\@firstofone
 % avoid brackets around text for \cite:
 \def\@biblabel#1{}
 \def\@cite#1#2{{#1\if@tempswa , #2\fi}}
\makeatother
\newlength{\cslhangindent}
\setlength{\cslhangindent}{1.5em}
\newlength{\csllabelwidth}
\setlength{\csllabelwidth}{3em}
\newenvironment{CSLReferences}[2] % #1 hanging-indent, #2 entry-spacing
 {\begin{list}{}{%
  \setlength{\itemindent}{0pt}
  \setlength{\leftmargin}{0pt}
  \setlength{\parsep}{0pt}
  % turn on hanging indent if param 1 is 1
  \ifodd #1
   \setlength{\leftmargin}{\cslhangindent}
   \setlength{\itemindent}{-1\cslhangindent}
  \fi
  % set entry spacing
  \setlength{\itemsep}{#2\baselineskip}}}
 {\end{list}}
\usepackage{calc}
\newcommand{\CSLBlock}[1]{\hfill\break\parbox[t]{\linewidth}{\strut\ignorespaces#1\strut}}
\newcommand{\CSLLeftMargin}[1]{\parbox[t]{\csllabelwidth}{\strut#1\strut}}
\newcommand{\CSLRightInline}[1]{\parbox[t]{\linewidth - \csllabelwidth}{\strut#1\strut}}
\newcommand{\CSLIndent}[1]{\hspace{\cslhangindent}#1}



\setlength{\emergencystretch}{3em} % prevent overfull lines

\providecommand{\tightlist}{%
  \setlength{\itemsep}{0pt}\setlength{\parskip}{0pt}}



 


\usepackage[noblocks]{authblk}
\renewcommand*{\Authsep}{, }
\renewcommand*{\Authand}{, }
\renewcommand*{\Authands}{, }
\renewcommand\Affilfont{\small}
\makeatletter
\@ifpackageloaded{caption}{}{\usepackage{caption}}
\AtBeginDocument{%
\ifdefined\contentsname
  \renewcommand*\contentsname{Table of contents}
\else
  \newcommand\contentsname{Table of contents}
\fi
\ifdefined\listfigurename
  \renewcommand*\listfigurename{List of Figures}
\else
  \newcommand\listfigurename{List of Figures}
\fi
\ifdefined\listtablename
  \renewcommand*\listtablename{List of Tables}
\else
  \newcommand\listtablename{List of Tables}
\fi
\ifdefined\figurename
  \renewcommand*\figurename{Figure}
\else
  \newcommand\figurename{Figure}
\fi
\ifdefined\tablename
  \renewcommand*\tablename{Table}
\else
  \newcommand\tablename{Table}
\fi
}
\@ifpackageloaded{float}{}{\usepackage{float}}
\floatstyle{ruled}
\@ifundefined{c@chapter}{\newfloat{codelisting}{h}{lop}}{\newfloat{codelisting}{h}{lop}[chapter]}
\floatname{codelisting}{Listing}
\newcommand*\listoflistings{\listof{codelisting}{List of Listings}}
\makeatother
\makeatletter
\makeatother
\makeatletter
\@ifpackageloaded{caption}{}{\usepackage{caption}}
\@ifpackageloaded{subcaption}{}{\usepackage{subcaption}}
\makeatother
\usepackage{bookmark}
\IfFileExists{xurl.sty}{\usepackage{xurl}}{} % add URL line breaks if available
\urlstyle{same}
\hypersetup{
  pdftitle={Preferences for the commodification of pensions in Chile: the rol of intergenerational social mobility},
  pdfauthor={Andreas Laffert Tamayo},
  colorlinks=true,
  linkcolor={blue},
  filecolor={Maroon},
  citecolor={Blue},
  urlcolor={Blue},
  pdfcreator={LaTeX via pandoc}}


\title{Preferences for the commodification of pensions in Chile: the rol
of intergenerational social mobility}


  \author{Andreas Laffert Tamayo}
            \affil{%
                  Instituto de Sociología, Pontificia Universidad
                  Católica de Chile
              }
      
\date{}

% Paquete para colores
\usepackage{xcolor}

% Definir color en formato HEX
\definecolor{mylinkcolor}{HTML}{3827a1}

% Ajustes de hipervínculos
\hypersetup{
  colorlinks = true,       % activar links coloreados
  linkcolor  = mylinkcolor,
  citecolor  = mylinkcolor,
  urlcolor   = mylinkcolor
}
\begin{document}
\maketitle
\begin{abstract}
My abstract \newline \textbf{Keywords}: Pension commodification ·
Intergenerational mobility · Causal inference · Meritocracy · Chile
\end{abstract}


\setstretch{1.5}
This document was last modified at 2025-12-01 02:47:28 and it was last
rendered at 2025-12-01 02:47:28.

\section{Introduction}\label{introduction}

What is the legitimate scope of market inequality in the eyes of the
public? Since the early 1980s, many countries have experienced a
rollback of universal welfare programs and a shift toward the
privatisation and commodification of public goods and social services
(\citeproc{ref-gingrich_making_2011}{Gingrich, 2011};
\citeproc{ref-streeck_how_2016}{Streeck, 2016}). In Latin America, these
reforms extended market logic into domains of social reproduction long
governed by the state, shrinking public provision and expanding private
actors in core welfare sectors (\citeproc{ref-ferre_welfare_2023}{Ferre,
2023}). From a moral-economy perspective, this diffusion of market rules
has reconfigured the normative order: through policy feedback, welfare
institutions embed market-based criteria of ``fair'' allocation and
shape how citizens understand deservingness and the balance between
state and market (\citeproc{ref-koos_moral_2019}{Koos \& Sachweh, 2019};
\citeproc{ref-svallforsMoralEconomyClass2006a}{Svallfors, 2006}).
Against this backdrop, growing scholarship examines \emph{market justice
preferences}---the extent to which citizens regard market-based criteria
as a fair basis for the allocation of essential services such as
healthcare, education, and pensions
(\citeproc{ref-castillo_changes_2025}{Castillo, Iturra, et al., 2025};
\citeproc{ref-castillo_perceptions_2025}{Castillo, Laffert, et al.,
2025}; \citeproc{ref-immergut_it_2020}{Immergut \& Schneider, 2020};
\citeproc{ref-lindh_public_2015}{Lindh, 2015};
\citeproc{ref-svallfors_political_2007}{Svallfors, 2007}). These
orientations matter because they legitimize unequal outcomes as the
product of individual responsibility and promote the conception of basic
social services as commodities.

This article examines public support for the commodification of pension
welfare---specifically, whether people consider it fair that access to
better pensions depends on income or contributions. Pension
privatisation lies at the core of Latin America's trajectory of
commodification (\citeproc{ref-huber_political_2000}{Huber \& Stephens,
2000}). As many countries expanded mandatory or voluntary
individual-capitalisation schemes---most radically in Chile---pensions
became organised around funded contribution-based entitlements and
market performance (\citeproc{ref-arenas_sistemas_2019}{Arenas, 2019};
\citeproc{ref-verbic_political_2019}{Verbič \& Spruk, 2019}). This shift
reshapes not only how pension institutions allocate old-age risk and tie
benefit levels to lifetime labour-market trajectories
(\citeproc{ref-madero-cabib_private_2019}{Madero-Cabib et al., 2019};
\citeproc{ref-oecd_pensions_2023}{OECD, 2023}), but also transforms the
justice principles citizens associate with old-age security: they must
judge whether higher benefits should follow market-derived criteria or
solidaristic principles
(\citeproc{ref-borzutzky_pension_2012}{Borzutzky, 2012};
\citeproc{ref-mau_inequality_2015}{Mau, 2015}). Although we know
something about what people consider a ``just'' monetary pension
(\citeproc{ref-castillo_deserving_2019}{Castillo, Olivos, et al., 2019})
and about support for public versus private provision
(\citeproc{ref-busemeyer_welfare_2020}{Busemeyer \& Iversen, 2020};
\citeproc{ref-jaime-castillo_public_2013}{Jaime-Castillo, 2013}), we
still lack systematic evidence on how citizens assess the fairness of
market-dependent access to pensions.

Research shows that support for market-based access to welfare services
is strongly stratified: people with higher socioeconomic status are more
likely to support market-based criteria than disadvantaged groups
(\citeproc{ref-busemeyer_skills_2014}{Busemeyer, 2014};
\citeproc{ref-immergut_it_2020}{Immergut \& Schneider, 2020};
\citeproc{ref-lindh_public_2015}{Lindh, 2015};
\citeproc{ref-svallfors_political_2007}{Svallfors, 2007}). Evidence for
pensions points in the same direction: higher income and greater
financial returns in funded schemes correlate with stronger support for
market-oriented pension provision
(\citeproc{ref-busemeyer_welfare_2020}{Busemeyer \& Iversen, 2020};
\citeproc{ref-kerner_pension_2020}{Kerner, 2020}). Yet we know very
little about how movement within the class structure shapes preferences
for pension commodification. This gap is especially salient in Latin
America, where mobility has unfolded under high inequality and deeply
privatised welfare regimes
(\citeproc{ref-lopez-roldan_comparative_2021}{López-Roldán \& Fachelli,
2021}). Social origins and destinations anchor material interests and
justice principles
(\citeproc{ref-alesina_intergenerational_2018}{Alesina et al., 2018};
\citeproc{ref-gugushvili_subjective_2017}{Gugushvili, 2017};
\citeproc{ref-langsaether_explaining_2022}{Langsæther et al., 2022}),
while movement between these positions exposes individuals to new risks,
resources, and normative environments that reshape fairness judgements
(\citeproc{ref-helgason_longterm_2023}{Helgason \& Rehm, 2023};
\citeproc{ref-jaime-castillo_social_2019}{Jaime-Castillo \&
Marqués-Perales, 2019}). Social mobility can thus foster market-justice
orientations, helping explain how inequalities in access to welfare
services become normatively sustained
(\citeproc{ref-mau_inequality_2015}{Mau, 2015}).

Two channels are particularly relevant to how intergenerational mobility
can shape preferences for market-based welfare: changes in material
interests and shifts in normative frames
(\citeproc{ref-ares_changing_2020}{Ares, 2020};
\citeproc{ref-gugushvili_intergenerational_2016c}{Gugushvili, 2016a};
\citeproc{ref-helgason_longterm_2023}{Helgason \& Rehm, 2023};
\citeproc{ref-jaime-castillo_social_2019}{Jaime-Castillo \&
Marqués-Perales, 2019}). Mobility alters individuals' material stakes by
changing their occupational status, income, and risk exposure. At the
same time, it reshapes their fairness judgments through acculturation
and learning experiences about the norms and values of the destination
class. Empirically, upward mobility is often associated with stronger
meritocratic interpretations of success and fairness, whereas downward
mobility is more closely linked to structural explanations of inequality
without necessarily eroding meritocratic convictions
(\citeproc{ref-bucca_merit_2016}{Bucca, 2016};
\citeproc{ref-deng_its_2025}{Deng \& Wang, 2025};
\citeproc{ref-mijs_belief_2022}{Mijs et al., 2022}). Regarding
market-based preferences, recent research shows that meritocratic
perceptions -- the conviction that effort is rewarded -- are strongly
associated with support for market allocation
(\citeproc{ref-castillo_perceptions_2025}{Castillo, Laffert, et al.,
2025}). In such contexts, it seems plausible that social mobility shapes
preferences for pension commodification through changes in their
material interests and in their meritocratic interpretations of
mobility. Thus, mobility and meritocracy evolve on partly separate
tracks yet intersect systematically
(\citeproc{ref-mau_inequality_2015}{Mau, 2015}).

Against this backdrop, the central aim of this article is to assess
whether intergenerational occupational mobility causally affects support
for the commodification of pensions. A second, more exploratory aim is
to examine whether this effect varies by meritocratic beliefs, using
this heterogeneity as an indication of possible underlying mechanisms. I
argue that social mobility operates through two main pathways
(\citeproc{ref-mau_inequality_2015}{Mau, 2015}). Upward mobility
increases earnings, expected returns in a funded system, and perceived
economic security, making contribution-based access more appealing.
Simultaneously, upward moves reinforce individualistic, effort-based
understandings of success and, thus, the view that markets are just
assignors. Downward mobility tends to heighten economic insecurity,
dependence on public protection, and attachment to external attributions
of poverty, reducing support for market allocation. Meritocratic beliefs
are treated not as mediators but as moderators of these effects: among
strong meritocrats, upward mobility should amplify pro-commodification
attitudes, whereas downward mobility should attenuate -- or even reverse
-- its expected de-commodifying impact, as meritocratic interpretations
of effort lead individuals to view market outcomes as fair regardless of
direction. This inquiry contributes to research on market justice by
identifying the causal effects of social mobility and the role of
meritocratic beliefs.

This study focuses on Chile, an instructive case for examining
preferences for pension commodification. Despite sustained economic
growth, Chile combines high and persistent inequality with short-range
upward mobility and substantial barriers to higher-class positions
(\citeproc{ref-flores_top_2020}{Flores et al., 2020};
\citeproc{ref-salgado_uplifting_2025}{Salgado et al., 2025};
\citeproc{ref-torche_intergenerational_2014}{Torche, 2014}). It was the
first country worldwide in 1981 to fully replace a public pay-as-you-go
system with a mandatory, privately administered defined-contribution
scheme, while the state has progressively expanded a segmented
solidarity pillar to compensate for insufficient benefits
(\citeproc{ref-boccardo_30_2020}{Boccardo, 2020};
\citeproc{ref-madariaga_three_2020}{Madariaga, 2020};
\citeproc{ref-solimano_rise_2021}{Solimano, 2021}). In parallel---and
despite waves of protest from 2016 to 2019 against the private pension
system (\citeproc{ref-somma_no_2021}{Somma et al., 2021})---Chilean
subjectivities have been increasingly shaped by neoliberal discourses
and market logics, influencing attitudes toward pension distribution and
inequality (\citeproc{ref-araujo_desafios_2012}{Araujo \& Martuccelli,
2012}; \citeproc{ref-galvez_con_2023}{Gálvez, 2023}). The combination of
profound inequality, market-driven old-age security, and contested
legitimacy makes Chile a critical case for analysing how mobility and
meritocratic beliefs structure support for pension commodification.

In this context, the questions that guide this research are as follows:

\begin{enumerate}
\def\labelenumi{(\arabic{enumi})}
\tightlist
\item
  How does intergenerational social mobility affect individuals'
  preferences for the commodification of pensions in Chile?
\item
  How do meritocratic beliefs condition this relationship?
\end{enumerate}

To address these questions, the study draws on large-scale,
representative survey data for the urban Chilean population collected in
2016, 2018, and 2023. It adopts a causal framework of mobility effects
inspired by Breen \& Ermisch (\citeproc{ref-breen_effects_2024}{2024}).
The following section outlines the theoretical framework linking
preferences for pension welfare commodification, social mobility, and
expected heterogeneity by meritocratic beliefs and derives hypotheses. A
subsequent section presents the causal identification strategy, the
data, and the analytical approach. The empirical sections report the
findings, and the article concludes by discussing what these results
reveal about the politics of market-based pension welfare and social
mobility.

\section{Theoretical and empirical
background}\label{theoretical-and-empirical-background}

\subsection{Preferences for pensions
commodification}\label{preferences-for-pensions-commodification}

Beyond the state's capacity to redistribute resources, market
institutions are central arenas for allocating socially valuable goods
and risks (\citeproc{ref-lindh_bringing_2023}{Lindh \& McCall, 2023}).
Markets are not mere aggregates of individual choices but social
institutions endowed with rules and normative meanings, built through
``moral projects, saturated with normativity'' that permeate everyday
thinking around fairness, effort and deservingness
(\citeproc{ref-fourcade_moral_2007}{Fourcade \& Healy, 2007};
\citeproc{ref-koos_moral_2019}{Koos \& Sachweh, 2019}). In this sense,
the economic order is mirrored in a moral economy: shared norms and
beliefs about what is considered a just distribution, embedded and
reinforced through institutions and policy feedback
(\citeproc{ref-svallforsMoralEconomyClass2006a}{Svallfors, 2006}). In an
era of privatisation and commodification, market logic has expanded into
core areas of social reproduction---childcare, healthcare, education and
pensions---stratifying access and quality
(\citeproc{ref-ferre_welfare_2023}{Ferre, 2023};
\citeproc{ref-gingrich_making_2011}{Gingrich, 2011}) and generating
feedback effects on public beliefs about welfare and market inequality
(\citeproc{ref-lindh_public_2015}{Lindh, 2015}). In this context,
support for market-based welfare services provision has grown worldwide,
particularly among higher-income groups who view private alternatives as
more efficient or higher-quality
(\citeproc{ref-busemeyer_welfare_2020}{Busemeyer \& Iversen, 2020};
\citeproc{ref-lindh_public_2015}{Lindh, 2015}). These institutional
transformations provide the backdrop for analysing how far citizens
accept the commodification of social protection
(\citeproc{ref-satz_por_2019}{Satz, 2019}).

The legitimacy of market-based welfare is closely tied to beliefs about
distributive justice grounded in market principles. Beyond
stratification structures, citizens hold structured beliefs about how
resources ought to be allocated, reflecting causal attributions for
inequality, normative principles and expectations about deservingness
(\citeproc{ref-kluegel_beliefs_1981}{Kluegel \& Smith, 1981}). Research
on distributive justice examines conceptions of how goods and rewards
should be distributed and when inequality is considered just
(\citeproc{ref-castillo_legitimacy_2011}{Castillo, 2011};
\citeproc{ref-jasso_gender_1999}{Jasso \& Wegener, 1999}), while work on
redistributive preferences focuses on support for state-led mechanisms
that reduce inequality (\citeproc{ref-cavaille_fair_2025}{Cavaillé,
2025}). Situated at this intersection, the emerging literature on market
justice preferences conceptualises the market itself as a site of
redistribution and asks whether individuals regard market criteria as
fair bases for allocating wages and access to core welfare services
(\citeproc{ref-castillo_perceptions_2025}{Castillo, Laffert, et al.,
2025}; \citeproc{ref-lindh_public_2015}{Lindh, 2015};
\citeproc{ref-lindh_bringing_2023}{Lindh \& McCall, 2023}). In this
perspective, the expansion of market logic into welfare domains implies
that social services are treated as legitimate commodities that can be
priced and stratified (\citeproc{ref-busemeyer_welfare_2020}{Busemeyer
\& Iversen, 2020}), and empirical work suggests that such institutional
changes feed back into attitudes, reinforcing market-conforming
understandings of fairness and welfare
(\citeproc{ref-zhu_policy_2015}{Ling Zhu \& Lipsmeyer, 2015}).

Market justice preferences refer to normative beliefs that legitimise
allocating goods and services according to market criteria such as
income and purchasing power
(\citeproc{ref-kluegel_legitimation_1999}{Kluegel et al., 1999};
\citeproc{ref-lindh_public_2015}{Lindh, 2015}). Building on Lane's
(\citeproc{ref-lane_market_1986}{1986}) contrast between market and
political justice, market justice is a distributive principle that
endorses rewarding effort, productivity and skill rather than need and
equality as emphasised in welfare-state policies. In this study, I treat
market justice less as an abstract doctrine and more concretely as
preferences for commodification: support for treating welfare services
as commodities, so that better services follow from higher income or
ability to pay, and the resulting stratified access is seen as fair
(\citeproc{ref-lindh_public_2015}{Lindh, 2015}). Empirically, such
orientations are typically measured with survey items asking whether it
is fair that access to better services depends on income, a strategy
rooted in research on justifications of capitalist inequality
(\citeproc{ref-kluegel_legitimation_1999}{Kluegel et al., 1999}) and
applied to healthcare and education
(\citeproc{ref-castillo_changes_2025}{Castillo, Iturra, et al., 2025};
\citeproc{ref-immergut_it_2020}{Immergut \& Schneider, 2020};
\citeproc{ref-lee_fairness_2023}{Lee \& Stacey, 2023};
\citeproc{ref-vondemknesebeck_are_2016}{Von Dem Knesebeck et al.,
2016}). Moving beyond single domains, Lindh
(\citeproc{ref-lindh_public_2015}{2015}) develops a comparative index of
income-based access to healthcare and education, while Castillo,
Laffert, et al. (\citeproc{ref-castillo_perceptions_2025}{2025}) use the
same single-item measure but add the pensions domain in Chile. These
constructs gauge how individuals view market-generated inequalities as
legitimate and capture two core dimensions of market distribution: the
role of income in determining attainment and the framing of services as
commodities that can be bought and sold according to ability to pay
(\citeproc{ref-lindh_public_2015}{Lindh, 2015}).

This study examines support for the commodification of pensions, defined
as normative judgements on whether better pensions should depend on
income or individual contributions. I treat this as a domain-specific
form of market justice: pensions are conceived as commodities whose
allocation should follow contribution- and return-based, and whose
outcome is seen as fair (\citeproc{ref-lane_market_1986}{Lane, 1986};
\citeproc{ref-mau_inequality_2015}{Mau, 2015}). The spread of funded,
privately managed schemes built on individual capitalisation accounts
has transformed both the distributive and normative structure of old-age
security (\citeproc{ref-arenas_sistemas_2019}{Arenas, 2019};
\citeproc{ref-huber_political_2000}{Huber \& Stephens, 2000}).
Distributively, these schemes reassign risk from the collective to the
individual by tying benefits to formal employment, contribution density,
earnings and financial returns, making retirement income heavily
dependent on labour-market trajectories and capital-market success
(\citeproc{ref-madero-cabib_private_2019}{Madero-Cabib et al., 2019};
\citeproc{ref-oecd_pensions_2023}{OECD, 2023}). Normatively, they recast
pensions as the outcome of personal investment decisions rather than
collective obligations, promoting ideals of individual responsibility,
self-economisation and contribution-based desert
(\citeproc{ref-arza_pension_2008}{Arza, 2008};
\citeproc{ref-borzutzky_pension_2012}{Borzutzky, 2012}). Contributions
are increasingly framed as private assets rather than risk pooling, and
workers are expected to act as employee--entrepreneurs who choose
providers, monitor portfolios and ``make their money work'' over the
life course (\citeproc{ref-borzutzky_chiles_2016}{Borzutzky \& Hyde,
2016}; \citeproc{ref-kerner_pension_2020}{Kerner, 2020};
\citeproc{ref-mau_inequality_2015}{Mau, 2015}). As more people tie their
security expectations to private investments and financial markets,
their material interests become more deeply rooted in the market, and
their imperatives gain stronger normative support
(\citeproc{ref-mau_inequality_2015}{Mau, 2015}). When citizens are asked
whether higher pensions should be allocated to those who earn and
contribute more, they reflect support for the market allocation of a
critical social service and the legitimacy of the resulting risks and
inequalities in old age (\citeproc{ref-busemeyer_welfare_2020}{Busemeyer
\& Iversen, 2020}; \citeproc{ref-lindh_public_2015}{Lindh, 2015}).

Most empirical work on market-based welfare preferences is
cross-national and shows that support for income-based access to
services is closely tied to countries' economic and institutional
contexts. Where public spending and social investment are higher,
support for market allocation of health and education tends to be
weaker, whereas in regimes with more prominent private providers and
heavily privatized services, it is stronger
(\citeproc{ref-busemeyer_skills_2014}{Busemeyer, 2014};
\citeproc{ref-immergut_it_2020}{Immergut \& Schneider, 2020};
\citeproc{ref-lindh_public_2015}{Lindh, 2015}). This literature suggests
that market justice preferences align with existing institutional
designs and outputs, echoing welfare-attitudes and policy-feedback
research, which argues that institutions not only redistribute resources
but also shape the categories through which people judge deservingness
(\citeproc{ref-campbell_institutional_2020}{Campbell, 2020}). Dominant
values and preferences are thus embedded in institutional configurations
(\citeproc{ref-busemeyer_skills_2014}{Busemeyer, 2014}), which actively
produce moral frameworks that legitimize or contest inequality.
Longitudinal evidence from Chile shows that agreement that higher-income
groups should have access to better welfare services rose markedly
between 2016 and 2023, especially for pensions
(\citeproc{ref-castillo_perceptions_2025}{Castillo, Laffert, et al.,
2025}), plausibly reflecting mixed policy feedback effects of the
private pension system (\citeproc{ref-busemeyer_positive_2021}{Busemeyer
et al., 2021}). Comparative research also shows that market justice
preferences are structured by individuals' socioeconomic positions and
broader attitudes to inequality.

\subsection{Social mobility}\label{social-mobility}

The drivers and consequences of social mobility have long been central
to sociology. Social mobility denotes movements between positions in a
stratification system. Classical and later work distinguish
intragenerational (within the life course) from intergenerational
mobility (between parents and children), as well as absolute
mobility---driven by structural change---from relative mobility, which
captures the extent to which social origins constrain destinations
(\citeproc{ref-eyles_social_2022}{Eyles et al., 2022}). Beyond mapping
mobility patterns, a large body of research examines how mobility shapes
attitudes and behaviours. In this literature, ``mobility effects'' refer
to outcomes that arise from movement between origin and destination
classes (\citeproc{ref-breen_effects_2024}{Breen \& Ermisch, 2024}). A
growing number of studies analyse the consequences of intergenerational
mobility for attitudes towards economic inequality (e.g.
\citeproc{ref-alesina_intergenerational_2018}{Alesina et al., 2018};
\citeproc{ref-bucca_merit_2016}{Bucca, 2016};
\citeproc{ref-day_movin_2017}{Day \& Fiske, 2017};
\citeproc{ref-gugushvili_intergenerational_2016}{Gugushvili, 2016b};
\citeproc{ref-helgason_class_2025}{Helgason \& Rehm, 2025}). These
inquiries rest on the idea that, if attitudes to inequality are strongly
stratified by class---because classes embody distinct material interests
and moral economies
(\citeproc{ref-edlundDemocraticClassStruggle2015a}{Edlund \& Lindh,
2015}; \citeproc{ref-kulinClassValuesAttitudes2013}{Kulin \& Svallfors,
2013})---, then people who move between classes should adjust their
attitudes towards those typical of their new class location. This study
contributes to this line of work by examining how intergenerational
mobility is associated with preferences for the commodification of
pensions in a highly stratified and commodified welfare regime such as
Chile's.

The literature on the effects of social mobility identifies several
mechanisms through which changes in class position may affect individual
attitudes (\citeproc{ref-helgason_longterm_2023}{Helgason \& Rehm,
2023}). A first family of accounts emphasises material self-interest and
can be divided into myopic and anticipatory theories. Myopic theories
assume that individuals rapidly align their attitudes with their current
class position, so mobility triggers shifts in material interests,
information and perceived risks that update preferences
(\citeproc{ref-ares_changing_2020}{Ares, 2020};
\citeproc{ref-helgason_longterm_2023}{Helgason \& Rehm, 2023};
\citeproc{ref-langsaether_explaining_2022}{Langsæther et al., 2022}).
Anticipatory theories stress that forward-looking agents align their
current attitudes with expected future income, as in the Prospect of
Upward Mobility (POUM) hypothesis, in which individuals may oppose
redistribution because they anticipate moving up
(\citeproc{ref-benabou_social_2001}{Benabou \& Ok, 2001}).

A second family foregrounds culture and socialisation. Acculturation
accounts posit that people gradually adapt their views to those
prevailing in their destination class through exposure to new
information, class-homophilous networks, group pressure and the
assimilation of principles and normative beliefs
(\citeproc{ref-helgason_class_2025}{Helgason \& Rehm, 2025};
\citeproc{ref-jaime-castillo_social_2019}{Jaime-Castillo \&
Marqués-Perales, 2019}). Socialisation perspectives, instead, argue that
core attitudes are formed early in life in the class of origin through
families, schools, and critical events, and remain largely stable
despite later mobility
(\citeproc{ref-jaime-castillo_social_2019}{Jaime-Castillo \&
Marqués-Perales, 2019}). Status-maximisation theories propose that
individuals align their attitudes with the highest class they have
occupied, so upwardly mobile respondents converge towards their
destination class, while downwardly mobile respondents retain attitudes
from their higher-status origins
(\citeproc{ref-jaime-castillo_social_2019}{Jaime-Castillo \&
Marqués-Perales, 2019}). Helgason and Rehm
(\citeproc{ref-helgason_longterm_2023}{2023},
\citeproc{ref-helgason_class_2025}{2025}) refine these ideas in a
learning framework, in which attitudes evolve slowly through cumulative
class experiences that combine economic resources, socialisation into
class-specific values, information, and networks. Finally, cognitive
mechanisms highlight the self-serving bias in causal attribution:
individuals interpret their mobility trajectories through internal or
external causes in ways that legitimise their current socioeconomic
position, thereby shaping their beliefs about effort, deservingness and
inequality(\citeproc{ref-gugushvili_intergenerational_2016}{Gugushvili,
2016b}; \citeproc{ref-miller_selfserving_1975}{Miller \& Ross, 1975};
\citeproc{ref-molina_its_2019}{Molina et al., 2019}). In what follows, I
focus on two mechanisms that are particularly relevant for market-based
pensions: material self-interest and acculturation into meritocratic
distributive norms.

How intergenerational social mobility shapes preferences for the
commodification of pensions? A plausible mechanism runs through changes
in material self-interest. Social mobility alters individuals' incomes,
job security and access to economic resources
(\citeproc{ref-helgason_longterm_2023}{Helgason \& Rehm, 2023};
\citeproc{ref-mau_inequality_2015}{Mau, 2015}), thereby modifying the
costs and benefits they face in a private individual pension
capitalization system. Material interests can be understood as the
pursuit of economic well-being under the constraints and trade-offs
implied by one's class position
(\citeproc{ref-wrightClassCountsComparative1997a}{Wright, 1997}). Upward
mobility typically entails higher and more stable earnings, greater
capacity to contribute to funded schemes, greater tolerance for
investment risk, and the ability to purchase supplementary coverage,
whereas downward mobility reduces income, undermines contribution
histories, and increases dependence on public or solidarity-based
pillars (\citeproc{ref-galvez_pensiones_2024}{Gálvez et al., 2024};
\citeproc{ref-madero-cabib_private_2019}{Madero-Cabib et al., 2019}).
These diverging pension prospects imply that upwardly and downwardly
mobile individuals face systematically different stakes in market-based
pension systems, making material self-interest a first channel through
which mobility may shape support for pension commodification.

Consistent with this reasoning, existing research indicates that support
for welfare commodification and pension marketisation is strongly
structured by material advantage. Individuals with higher income,
education and occupational class are more likely to endorse market-based
distributive principles and to view private provision as a way to secure
or enhance relative advantage
(\citeproc{ref-busemeyer_skills_2014}{Busemeyer, 2014};
\citeproc{ref-immergut_it_2020}{Immergut \& Schneider, 2020};
\citeproc{ref-koos_moral_2019}{Koos \& Sachweh, 2019};
\citeproc{ref-lindh_public_2015}{Lindh, 2015};
\citeproc{ref-svallfors_political_2007}{Svallfors, 2007};
\citeproc{ref-vondemknesebeck_are_2016}{Von Dem Knesebeck et al.,
2016}). In the pension field, institutional design sharpens these
incentives. Busemeyer \& Iversen
(\citeproc{ref-busemeyer_welfare_2020}{2020}) show that in countries
where private alternatives to public pensions exist, support for public
pension spending among upper-income groups erodes as they exit public
schemes into private plans whose contributions and benefits closely
track their income and risk. In Latin America, Kerner
(\citeproc{ref-kerner_pension_2020}{2020}) finds that higher income and
higher perceived financial returns from pension funds are associated
with preferences for a stronger market role in pensions, suggesting that
favourable returns are interpreted as evidence that pension
neoliberalism ``works''. Evidence from Chile points in the same
direction: higher-income and university-educated respondents express
stronger support for market allocation in healthcare, education and
pensions (\citeproc{ref-castillo_socialization_2024}{Castillo et al.,
2024}; \citeproc{ref-otero_power_2024}{Otero \& Mendoza, 2024}), and
economic elites---especially those socialised in highly prestigious
schools---hold more favourable attitudes towards inequality that can be
traced to their material interests against redistribution
(\citeproc{ref-carranza_what_2024}{Carranza et al., 2024}). Systems that
link benefits to individual contributions thus allow better-off groups
to capitalise on their financial independence and minimise reliance on
public support, implying that mobility into or out of these advantaged
positions should alter individuals' stakes in pension commodification.

Research on redistribution and inequality offers additional, indirect
support for this self-interest mechanism. Individuals with pessimistic
mobility expectations---especially those anticipating downward
movement---display stronger support for redistributive policies
(\citeproc{ref-alesina_intergenerational_2018}{Alesina et al., 2018}).
Studies tracking realised mobility show that, compared to immobility,
upward moves are associated with more economically conservative, less
redistributive attitudes, whereas downward moves strengthen demand for
state intervention and redistribution
(\citeproc{ref-ares_changing_2020}{Ares, 2020};
\citeproc{ref-jaime-castillo_social_2019}{Jaime-Castillo \&
Marqués-Perales, 2019};
\citeproc{ref-langsaether_explaining_2022}{Langsæther et al., 2022};
\citeproc{ref-rodriguez_intergenerational_2024}{Rodriguez \&
Matilla-Garcia, 2024}; \citeproc{ref-schmidt_experience_2011}{Schmidt,
2011}). Over the life course, cumulative exposure to higher-class
positions---whether by staying or moving up---is linked to more
market-liberal economic values, while sustained exposure to lower-class
positions---through stability or downward mobility---reinforces
pro-redistributive orientations
(\citeproc{ref-helgason_longterm_2023}{Helgason \& Rehm, 2023},
\citeproc{ref-helgason_class_2025}{2025}). Domain-specific evidence also
aligns with changing material stakes: Gugushvili
(\citeproc{ref-gugushvili_subjective_2017}{2017}) finds that upward
mobility is associated with lower support for government spending on
housing and pensions, whereas downward mobility reduces support for
healthcare and education but increases support for housing and pensions,
reflecting the material nature of these domains. Beyond redistribution,
upward mobility fosters more individualistic attributions of poverty and
wealth and greater legitimation of income inequality
(\citeproc{ref-bucca_merit_2016}{Bucca, 2016};
\citeproc{ref-gugushvili_intergenerational_2016}{Gugushvili, 2016b},
\citeproc{ref-gugushvili_intergenerational_2016c}{2016a}), and societies
with higher economic mobility display more tolerance of inequality
(\citeproc{ref-shariff_income_2016}{Shariff et al., 2016}). Taken
together, this evidence suggests that social mobility realigns attitudes
with changing material stakes and distributive interests. Although the
present study cannot directly test the underlying mechanism, it treats
myopic self-interest as the central interpretive framework for the
expected effects of intergenerational mobility on support for pension
commodification.

Consequently, the main hypothesis of this research is:

\(H_1\): Compared to immobility, intergenerational upward mobility is
associated with stronger, and downward mobility with weaker, support for
the commodification of pensions.

While material self-interest is the main lens, mobility also reshapes
norms and interpretive frames. Mobile individuals move across social
milieus and networks and are exposed to new standards and beliefs about
fairness and deservingness
(\citeproc{ref-helgason_longterm_2023}{Helgason \& Rehm, 2023}).
Acculturation accounts hold that people gradually adopt the attitudes of
their destination class
(\citeproc{ref-jaime-castillo_social_2019}{Jaime-Castillo \&
Marqués-Perales, 2019}). Mobility can thus reshape how individuals
interpret success, failure and inequality, with upward moves fostering
more individualistic and meritocratic understandings and downward moves
encouraging structural accounts of disadvantage
(\citeproc{ref-gugushvili_subjective_2017}{Gugushvili, 2017};
\citeproc{ref-mau_inequality_2015}{Mau, 2015};
\citeproc{ref-mijs_belief_2022}{Mijs et al., 2022}) Applied to pensions,
those who rise may come to see market-based schemes as fair systems that
reward effort and foresight. In contrast, those who fall may view
commodified pensions as unjust and and that the pension system does not
reward merit. Therefore, shifts in meritocratic norms can be a second
channel through which the mobility effect operates, serving as a
complementary acculturation mechanism that may influence support for
pension commodification, as examined in the next section.

\subsection{Meritocracy}\label{meritocracy}

Meritocracy refers to a distributive system in which individual
merit---typically defined as effort and talent---is the primary
criterion for allocating resources and rewards, rather than social
origins or inherited privilege (\citeproc{ref-sen_merit_2000}{Sen,
2000}; \citeproc{ref-young_rise_1958}{Young, 1958}). Originally coined
by Young in a dystopian critique of a society where power and status are
justified by achievement and mobility only driven by ``merit'', the term
has been re-appropriated as a positive ideal of fairness, especially in
liberal and market-oriented societies
(\citeproc{ref-dubet_repensar_2011}{Dubet, 2011};
\citeproc{ref-mijs_paradox_2019}{Mijs, 2019};
\citeproc{ref-vandewerfhorst_meritocracy_2024}{Van De Werfhorst, 2024}).
From a sociological perspective, belief in meritocracy is not just a
cognitive judgement; it constitutes a moral lens through which
individuals interpret social and economic disparities
(\citeproc{ref-castillo_multidimensional_2023}{Castillo et al., 2023}).
The conviction that economic inequalities are justified because they
reflect differential merit has been identified as a key mechanism behind
the persistence of inequality, since it recasts structural advantages
and disadvantages as outcomes of individual performance and encourages
``winners'' to see their position as earned and ``losers'' to
internalise blame rather than question the underlying structure of
opportunity (\citeproc{ref-mijs_unfulfillable_2016}{Mijs, 2016};
\citeproc{ref-sandel_tyranny_2020}{Sandel, 2020};
\citeproc{ref-wilson_role_2003}{Wilson, 2003}). Understanding how these
beliefs in meritocracy are structured is therefore crucial for analysing
attitudes toward inequality.

Recent research has emphasised the need to decompose the term
``meritocratic beliefs'' and to distinguish between meritocratic
preferences and meritocratic perceptions
(\citeproc{ref-castillo_multidimensional_2023}{Castillo et al., 2023};
\citeproc{ref-zhu_meritocratic_2025}{Li Zhu, 2025}). Preferences refer
to normative ideas about how rewards should be allocated---whether
people think effort and ability ought to determine life
chances---whereas perceptions capture evaluations of how meritocracy
actually operates in society, that is, whether people believe that
existing inequalities reflect merit-based processes
(\citeproc{ref-janmaat_subjective_2013}{Janmaat, 2013}). This
distinction is crucial for understanding how people interpret inequality
in stratified societies because it clarifies the (mis)alignment between
normative support for meritocracy and evaluations about its actual
implementation (\citeproc{ref-castillo_multidimensional_2023}{Castillo
et al., 2023}; \citeproc{ref-lindner_does_2024}{Lindner et al., 2024};
\citeproc{ref-mijs_paradox_2019}{Mijs, 2019};
\citeproc{ref-mijs_how_2022}{Mijs \& Hoy, 2022};
\citeproc{ref-newman_economic_2023}{Newman, 2023}). In this study, I
focus on meritocratic perceptions of effort and talent as key dimensions
of broader meritocratic beliefs, given their central role in shaping
attitudes toward inequality and welfare.

Several empirical studies have examined the social foundations and
consequences of meritocratic perceptions. Individuals with higher levels
of education, income and occupational prestige are more likely to
endorse merit-based explanations for social outcomes
(\citeproc{ref-duru-bellat_who_2012}{Duru-Bellat \& Tenret, 2012};
\citeproc{ref-garcia-sanchez_perceptions_2018}{García-Sánchez et al.,
2018}; \citeproc{ref-mijs_paradox_2019}{Mijs, 2019}). These perceptions
function as a normative framework that legitimises several unequal
outcomes, especially when access to social goods is stratified by income
or social background. Individuals who see their society as meritocratic
tend to show lower support for redistribution
(\citeproc{ref-hoyt_mindsets_2023}{Hoyt et al., 2023};
\citeproc{ref-tejero-peregrina_perceived_2025}{Tejero-Peregrina et al.,
2025}), greater legitimisation of class inequality
(\citeproc{ref-darnon_where_2018}{Darnon et al., 2018}), increased
tolerance of income gaps (\citeproc{ref-batruch_belief_2023}{Batruch et
al., 2023}), more support for system-justifying ideologies
(\citeproc{ref-wiederkehr_belief_2015}{Wiederkehr et al., 2015}), and
even lower perceived levels of economic inequality itself
(\citeproc{ref-castillo_meritocracia_2019}{Castillo, Torres, et al.,
2019}). This evidence provides a basis for extending the study of the
attitudinal consequences of meritocracy to preferences for market-based
welfare.

Preferences for the commodification of social welfare services are
anchored in normative frames. Studies show that support for market-based
welfare is higher among individuals who endorse meritocratic and liberal
principles---that benefits should depend on individual effort and
contributions---and among those with economically conservative
orientations, whereas egalitarian views reduce support for market
allocation (\citeproc{ref-jaime-castillo_public_2013}{Jaime-Castillo,
2013}; \citeproc{ref-lee_fairness_2023}{Lee \& Stacey, 2023};
\citeproc{ref-quadagno_has_2012}{Quadagno \& Pederson, 2012}). In the
pension field, factorial evidence from Chile indicates that merit-based
criteria such as educational attainment and years in the labour force
weigh more heavily than need-based attributes in judgments of just
pension levels, with respondents willing to accept very low pensions for
those perceived as low achievers
(\citeproc{ref-castillo_deserving_2019}{Castillo, Olivos, et al.,
2019}). Recently, Castillo, Laffert, et al.
(\citeproc{ref-castillo_perceptions_2025}{2025}) show that agreement
with the notion that it is fair for higher-income individuals to access
better welfare services was relatively low in 2016 but rose
significantly by 2023, especially in the pension domain. They also find
that individuals who endorse more meritocratic perceptions--specifically
the perception that effort is rewarded-- are more likely to support
market-based access for healthcare, education and pensions. All in all,
meritocratic perceptions are a powerful source of inequality
legitimation, anchoring who is seen as deserving to be better or worse
off, and are strongly linked to how individuals understand unequal life
chances and movements within the social hierarchy.

Social mobility and meritocratic perceptions are closely intertwined in
the acculturation mechanism, although the direction of this link can
vary across trajectories and contexts. A large body of work consistently
shows that upwardly mobile individuals tend to interpret their
trajectory through a meritocratic lens, reading their success as
evidence that effort and ability are rewarded and reinforcing
individualistic attributions of poverty and wealth
(\citeproc{ref-bucca_merit_2016}{Bucca, 2016};
\citeproc{ref-gugushvili_intergenerational_2016}{Gugushvili, 2016b};
\citeproc{ref-mijs_belief_2022}{Mijs et al., 2022};
\citeproc{ref-shariff_income_2016}{Shariff et al., 2016}), even in a
highly unequal and low mobility setting were the narratives of
individual success reinforces (\citeproc{ref-deng_its_2025}{Deng \&
Wang, 2025}). Studies of economic elites in Chile similarly report
strong endorsement of meritocracy, with respondents attributing their
ascent mainly to talent and business or leadership skills
(\citeproc{ref-atria_economic_2020}{Atria et al., 2020}). Evidence on
downward or blocked mobility is more ambivalent. Exposure to
low-mobility contexts and experiences of downward mobility are
associated with weaker beliefs that effort is rewarded, lower just-world
and opportunity beliefs, and stronger structural attributions for
success and failure (\citeproc{ref-davidai_why_2018}{Davidai, 2018};
\citeproc{ref-day_movin_2017}{Day \& Fiske, 2017};
\citeproc{ref-gugushvili_intergenerational_2016}{Gugushvili, 2016b};
\citeproc{ref-mijs_how_2022}{Mijs \& Hoy, 2022}). Yet, under strong
meritocratic convictions, status loss can also be internalised as
personal failure, sustaining system-justifying attitudes and the
legitimation of existing hierarchies as a psychological defence strategy
(\citeproc{ref-day_movin_2017}{Day \& Fiske, 2017};
\citeproc{ref-deng_its_2025}{Deng \& Wang, 2025};
\citeproc{ref-mau_inequality_2015}{Mau, 2015}). In sum, upward mobility
is generally associated with stronger meritocratic perceptions, whereas
downward mobility is more closely linked to structural explanations of
inequality, without necessarily eroding meritocratic convictions.

The ways in which social mobility trajectories shape interpretations of
meritocracy can determine whether mobility translates into stronger or
weaker support for market-based pension welfare. Following Mau's
(\citeproc{ref-mau_inequality_2015}{2015}) account of the ``majority
class'', upward mobility is expected to strengthen market-based welfare
not only because material interests change, but because those who rise
come to read their story as one of individual effort and talent
rewarded. In this view, markets are seen as fair arenas that properly
allocate pensions and other welfare benefits according to merit, so
upwardly mobile individuals with strong meritocratic convictions should
be especially prone to endorse commodified pensions. In contrast,
downward mobility typically fosters de-commodifying demands, yet under
strong meritocratic perceptions, status loss may be read as individually
deserved, weakening---or even reversing---its usual de-commodifying
pattern, as market outcomes are still viewed as broadly fair regardless
of direction. Empirical research on redistribution and attitudes toward
inequality supports this mechanism. Survey studies suggest that
subjective social mobility shapes support for redistribution and pension
spending through self-serving bias
(\citeproc{ref-gugushvili_subjective_2017}{Gugushvili, 2017};
\citeproc{ref-schmidt_experience_2011}{Schmidt, 2011}), although recent
work indicates that this mechanism is evident only under explicit
constraints in experimental designs
(\citeproc{ref-molina_its_2019}{Molina et al., 2019}). More in line with
the acculturation theories, experimental reserach has shown that
exposure to upward-mobility information or self-made narratives reduces
support for redistribution, increases tolerance of inequality and system
justification, and even legitimises exploitative work conditions,
whereas exposure to low or downward mobility produces the opposite
pattern; crucially, these effects are partly mediated by stronger
meritocratic perceptions (\citeproc{ref-deng_its_2025}{Deng \& Wang,
2025}; \citeproc{ref-matamoros-lima_social_2025}{Matamoros-Lima et al.,
2025}; \citeproc{ref-shariff_income_2016}{Shariff et al., 2016}). In
this framework, meritocratic perceptions condition the attitudinal
consequences of mobility by amplifying the pro-market effects of upward
mobility and muting the critical potential of downward mobility; thus, I
treat meritocratic perceptions as a moderating mechanism linking
intergenerational mobility to preferences for pension commodification.

Accordingly, the second set of hypotheses of this study is:

\(H_{2a}\): Under high meritocracy, upward mobility should generate a
stronger increase in support for market-based pension access than under
low meritocracy.

\(H_{2b}\): Under low meritocracy, downward mobility should moderately
reduce support for market-based pension access, whereas under high
meritocracy, this negative effect should be attenuated toward neutral or
slightly positive attitudes.

A summary of the hypotheses and their expected directional patterns is
presented in Table~\ref{tbl-hipo}.

\begin{table}

\caption{\label{tbl-hipo}Overview of hypotheses with expected directional patterns }

\centering{

\centering

\begin{tabular}{p{3.2cm} p{3.2cm} p{4cm} p{5cm}}
\toprule
\textbf{Hypothesis} & \textbf{Focus} & \textbf{Expected pattern} & \textbf{Interpretation} \\
\midrule
\textbf{H1: Average mobility effect} &
Effect of intergenerational mobility on preferences for pension commodification &
Upward mobility $\uparrow$ higher support; downward mobility $\downarrow$ lower support relative to immobility &
Mobility shapes preferences through material stakes and normative frames \\
\midrule
\textbf{H2a: Upward $\times$ Meritocracy} &
Conditional effect of upward mobility by meritocracy &
Under high meritocracy: amplified pro-market effect $\uparrow\uparrow$; under low meritocracy: modest increase $\uparrow$ &
Meritocratic lenses strengthen pro-market reactions to upward mobility \\
\midrule
\textbf{H2b: Downward $\times$ Meritocracy} &
Conditional effect of downward mobility by meritocracy &
Under low meritocracy: moderate decrease $\downarrow$; under high meritocracy: negative effect attenuated $\searrow$ toward neutral or slightly positive attitudes &
Meritocratic lenses weaken de-commodifying reactions to downward mobility \\
\bottomrule
\end{tabular}

}

\end{table}%

\subsection{The Chilean context}\label{the-chilean-context}

Chile offers a compelling case for examining preferences for
market-based pension welfare. Despite sustained economic growth, it
remains one of the most unequal countries in Latin America and the OECD.
The poorest 50\% of the population captures just 10\% of total income
and holds negative net wealth, while the wealthiest 1\% receives nearly
27\% of all income and controls almost half of the country's wealth
(\citeproc{ref-chancel_world_2022}{Chancel et al., 2022}), a pattern
that has changed little since the 1990s
(\citeproc{ref-flores_top_2020}{Flores et al., 2020}). Research on
intergenerational mobility shows that Chile combines high absolute
mobility with mostly short-range moves---mainly from working classes
into lower service-class positions---while access to upper and elite
classes remains tightly restricted, and that despite the expansion of
higher education, relative mobility remains rigid and intergenerational
inequalities persist (\citeproc{ref-espinoza_movilidad_2014}{Espinoza \&
Núñez, 2014}; \citeproc{ref-lopez-roldan_comparative_2021}{López-Roldán
\& Fachelli, 2021}; \citeproc{ref-salgado_uplifting_2025}{Salgado et
al., 2025}; \citeproc{ref-torche_intergenerational_2014}{Torche, 2014}).
Crucially, a significant share of Chile's inequality is rooted in
neoliberal reforms that privatised and commodified key social services,
stratifying and segmenting access to high-quality education, healthcare
and housing between classes (\citeproc{ref-ferre_welfare_2023}{Ferre,
2023}; \citeproc{ref-pnud_desiguales_2017}{PNUD, 2017}).

Introduced under the military dictatorship (1973--1989) and expanded by
democratic governments, neoliberal reforms in Chile embedded market
logic into public services through concessions, subsidies, vouchers and
pro-private regulation (\citeproc{ref-boccardo_30_2020}{Boccardo, 2020};
\citeproc{ref-madariaga_three_2020}{Madariaga, 2020}). This
``crowded-out'' welfare model benefits higher-income groups, leaving
lower-income individuals to rely on limited public options. Scholars
argue that this neoliberal shift repurposed the state to create niches
of capitalist accumulation---from water to education---giving rise to a
model of ``public-service capitalism'' heavily reliant on state funding
(\citeproc{ref-boccardo_30_2020}{Boccardo, 2020}). In this setting,
class trajectories unfold under conditions of high inequality and
market-based provision in core welfare domains, such as pensions.

Chile was the first country to fully replace a public pay-as-you-go
social insurance system with mandatory individual capitalisation: in
1981, the tripartite \emph{``cajas de previsión''}(insurance schemes)
were replaced by a fully funded defined-contribution scheme managed by
private Pension Fund Administrators (AFP)
(\citeproc{ref-arenas_sistemas_2019}{Arenas, 2019};
\citeproc{ref-huber_political_2000}{Huber \& Stephens, 2000}). Framed by
its principal architect, José Piñera, as ``the mother of all battles'',
the reform was a way to roll back the redistributive role of the state,
shift ``freedom of choice'' to individuals and create a ``society of
worker-capitalists'' (\citeproc{ref-borzutzky_pension_2012}{Borzutzky,
2012}; \citeproc{ref-solimano_rise_2021}{Solimano, 2021}). Dependent
workers contribute 10\% of their wages to individual accounts, plus an
administrative fee, while employers contribute nothing; AFPs invest
these mandatory savings in domestic and international financial markets,
and since the 2000s affiliates choose (or are defaulted into)
``multifunds'' (A--E) with different risk profiles, directly exposing
pensions to market volatility
(\citeproc{ref-barriga_pensiones_2024}{Barriga \& Kremerman, 2024};
\citeproc{ref-hyde_chiles_2015}{Hyde \& Borzutzky, 2015}). After more
than four decades, this architecture has produced an open pension
crisis: for most retirees, AFP-financed pensions fall below the minimum
wage and are clearly insufficient to avoid old-age poverty, with
replacement rates around 30\% even among cohorts with long contribution
histories (\citeproc{ref-galvez_pensiones_2024}{Gálvez et al., 2024};
\citeproc{ref-madero-cabib_private_2019}{Madero-Cabib et al., 2019}). To
prevent destitution, the state has been forced to re-enter through
tax-funded solidarity pillars ---the Nuevo Pilar Solidario (2008) and
the Pensión Garantizada Universal (PGU, 2022)--- that partially repair
these failures (\citeproc{ref-boccardo_30_2020}{Boccardo, 2020};
\citeproc{ref-solimano_rise_2021}{Solimano, 2021}) by topping up low or
non-existent contributory pensions for the poorest and lower-middle
older adults, so that for many low earners these tax-funded
contributions are the main component of their total pension income
(\citeproc{ref-galvez_pensiones_2024}{Gálvez et al., 2024}). Meanwhile,
the AFP industry has remained highly concentrated
(\citeproc{ref-hyde_chiles_2015}{Hyde \& Borzutzky, 2015}), posting very
high returns on equity (around US\$ 613 million in 2024) and fostering
powerful national economic actors
(\citeproc{ref-ruiz_formacion_2020}{Ruiz, 2020}), leaving Chile with a
structurally fragile, segmented and regressive pension system that
combines poor pensions, heavy reliance on public subsidies and high,
sustained profits for private financial intermediaries. Chile's
combination of constrained mobility, persistent inequality and deep
privatisation has eroded expectations of upward mobility and fuelled
social discontent (\citeproc{ref-pnud_desiguales_2017}{PNUD, 2017}),
with pensions at the centre of this conflict.

Despite recurrent social unrest, Chile exhibits a paradoxical
coexistence of intense conflict over inequality and its widespread
legitimation, vividly illustrated by pensions. The October 2019 ``social
outburst'', a period of mass protests and severe political repression
throughout the country, crystallised discontent over the privatisation
and commodification of social services, alongside a crisis of political
legitimacy (\citeproc{ref-somma_no_2021}{Somma et al., 2021}). One of
the main conflicts in this period centred on grievances over low
pensions and the profits of private pension funds, building on earlier
mobilisation by the \emph{``No+AFP''} movement since 2016; a survey
during the 2019 protests identified pensions as the top public demand
(\citeproc{ref-nudesoc_informe_2020}{NUDESOC, 2020}). This conflict
helped place comprehensive pension reform on the political agenda after
2022. However, contestation coexists with enduring market-conforming
orientations. Research documents strong meritocratic beliefs and high
tolerance for inequality
(\citeproc{ref-castillo_meritocracia_2019}{Castillo, Torres, et al.,
2019}; \citeproc{ref-mac-clure_justicia_2024}{Mac-Clure et al., 2024}),
and the COVID-19 emergency withdrawals of pension savings gave rise to
the slogan \emph{``Con mi plata NO''}, expressing opposition both to
greater pay-as-you-go elements and to continued AFP control
(\citeproc{ref-galvez_con_2023}{Gálvez, 2023}). This stance reflects the
internalisation of the ``enterprising self''
(\citeproc{ref-mau_inequality_2015}{Mau, 2015}): families navigate
school choice as investment strategies
(\citeproc{ref-canalesceron_sujeto_2021}{Canales Cerón et al., 2021}),
and workers denounce the pension system as illegitimate in its design
and outcomes while simultaneously treating their contributions as
personal investments aimed at maximising returns and asserting that they
themselves ---rather than AFPs, the state or ``society''--- are the
legitimate owners of their pension funds
(\citeproc{ref-panes_criticas_2020}{Panes, 2020}). These ambivalent
representations indicate that market-based moral orientations are deeply
embedded, making Chile a fertile setting to study how support for
pension commodification varies across social groups and meritocratic
beliefs.

\section{Method}\label{method}

Social mobility effects---understood as impacts on an outcome arising
from movements between an origin state and a destination state (e.g.,
social class)---have long been a focus of sociological research
(\citeproc{ref-eyles_social_2022}{Eyles et al., 2022};
\citeproc{ref-langsaether_explaining_2022}{Langsæther et al., 2022}).
Yet, as Breen and Ermisch (\citeproc{ref-breen_effects_2024}{2024, p.
467}) emphasize, most of the mobility hypotheses are, at their core,
individual-level counterfactual comparisons between the observed outcome
under a mobility trajectory and the outcome the same person would have
shown if they had remained in their origin class (or moved to an
alternative destination). Standard specifications (e.g., SAM, DRM, and
mobility-contrast models) struggle to retrieve these inherently
counterfactual quantities due to identification constraints and their
reliance on between-group contrasts constructed from additive terms and
interactions, thereby yielding primarily associational evidence
(\citeproc{ref-breen_effects_2024}{Breen \& Ermisch, 2024};
\citeproc{ref-song_there_2025}{Song \& Zhou, 2025}). Following the
causal framework of Breen and Ermisch
(\citeproc{ref-breen_effects_2024}{2024}), I conceive of the destination
class as a treatment, conditioning on the origin class and estimating
the heterogeneous effects of the destination with observational data
under explicit identification assumptions. To align design and target, I
follow the MIDA template (\citeproc{ref-blair_research_2023}{Blair et
al., 2023}): set out the causal model and assumptions (M), define the
inquiry and estimand (I), describe the data and variables (D), and
detail the answer strategy (A) used to identify the causal effect of
intergenerational occupational mobility on preferences for the
commodification of pensions.

\subsection{A Causal Model for Mobility
Effects}\label{a-causal-model-for-mobility-effects}

Breen and Ermisch (\citeproc{ref-breen_effects_2024}{2024}), building on
a critical reassessment of the inferential limits of standard mobility
models, propose a causal framework that reconceptualizes the ``mobility
effect'' as the treatment effect of reaching a destination class, with
impacts heterogeneous by class of origin. The core claim is that typical
mobility hypotheses are fundamentally within-person counterfactual
comparisons rather than between-person contrasts. Defining mobility as
\(M = D - O\); a change from origin \((O)\) to destination \((D)\),
rather than an additive or interactive combination, places the focus
squarely on the Neyman--Rubin problem: for any individual, we observe
the result in the social position they currently occupy, but not in the
situation of other alternative destinations or immobility.

Exploiting the temporal ordering of origin, destination and outcome,
Breen and Ermisch (\citeproc{ref-breen_effects_2024}{2024}) frame
mobility as a treatment process that motivates the causal question:
``how would the outcome among people from origin \emph{j} who entered
destination \emph{k} have been different if those people had,
counterfactually, entered destination \emph{k'} instead?'' (p.472). The
corresponding estimand is the conditional causal effect of destination
given origin. It is of particular interest when \emph{k' = j}, i.e.,
immobility. This estimand compares movers' observed outcomes with the
hypothetical outcomes those same individuals would have exhibited had
they remained in their origin class. This formulation provides a
coherent potential-outcomes basis for studying how social mobility shape
individual preferences and attitudes.

Following Breen and Ermisch (\citeproc{ref-breen_effects_2024}{2024}),
the identification of causal mobility effects from observational data
requires the following assumptions:

\textbf{Positivity}: For all \((j,c)\) in the support of \((O,C)\) and
for each relevant destination \((k)\), the probability of receiving
\(D=k\) is strictly between 0 and 1; substantively, each type \((O,C)\)
has a nonzero probability of entering each comparison destination.

\begin{equation}\phantomsection\label{eq-posi}{
 0<P(D=k\mid O=j, C=c)<1
}\end{equation}

\textbf{Stable Unit Treatment Value Assumption (SUTVA)}: each unit's
potential outcomes \(Y_i(D)\) does not depend on the mechanism used to
assign treatments (destinations) and by the treatments assigned to other
units (also called no interference), assuming a single, well-defined
version of each treatment (consistency).

\begin{equation}\phantomsection\label{eq-sutva}{
Y_i(D_i) = Y_i\big(D_i, D_{-i}\big)\quad \forall, D_{-i}
}\end{equation}

\textbf{Conditional Independence}: Also known as conditional
unconfoundedness or exchangeability, this assumption emphasizes that,
conditional on origin \(O\) and pre-treatment covariates \(C\) (e.g.,
parental education, ethnicity, cohort, early-life factors), assignment
to \(D\) is as good as random. This underlies IPW and regression
adjustment, which seek exchangeability between mobile (treated) and
immobile individuals (controls).

\begin{equation}\phantomsection\label{eq-indep}{
Y(D)\ \perp\kern-5pt \perp D\ \mid\ (O,C)
}\end{equation}

Establishing the causal relationship between objective social mobility
and subjective outcomes, such as preferences, poses several challenges
for causal inference. Below, I use directed acyclic graphs (DAGs) to
illustrate some of these challenges and evaluate possible strategies for
identifying the effects of social mobility. Figure~\ref{fig-dag} depicts
the causal model guiding identification. In this model, a respondent's
preference \(Y\) is directly affected by her class destination \(D\),
and indirectly influenced by a set of pre-treatment attributes \(C\)
rooted in childhood---family resources, household composition, parental
education, and other ascriptive characteristics. Class origin \(O\) is
itself shaped by these background factors \(C\) and, in turn, affects
destination \(D\). I allow for unobserved determinants \(U\) that
influence background factors \(C\), but I assume that any such
unobserved variation operates only through \(C\). Thus, there are no
direct paths \(U \rightarrow D\) or \(U \rightarrow Y\) beyond those
mediated by \(C\).

Under this structure, conditioning on \((O, C)\) blocks all relevant
backdoor paths from \(D\) to \(Y\)---notably
\(D \leftarrow O \leftarrow C \rightarrow Y\). This renders \((O, C)\) a
minimal sufficient adjustment set for identifying the causal effect of
mobility on preferences, in line with the framework proposed by Breen \&
Ermisch (\citeproc{ref-breen_effects_2024}{2024}). Identification
therefore relies on the assumption that background factors \(C\)
adequately summarize pre-treatment characteristics that jointly shape
both mobility and attitudes.

\begin{figure}[H]

\centering{

\includegraphics[width=0.85\linewidth,height=\textheight,keepaspectratio]{paper_files/figure-pdf/fig-dag-1.pdf}

}

\caption{\label{fig-dag}Causal graph of the effect of intergenerational
social mobility on preferences for commodification. Y = preferences for
pension commodification, D = an individual's class of destination, O =
an individual's class of origin, C = different attributes determined in
childhood or earlier that affects an individual's class of origin and
destination. Finally, U = unobserved factors influencing origin
attributes.}

\end{figure}%

Why are the assumptions plausible here? Conditional independence is
targeted by controlling for a plausible sufficient adjustment set
\((O,C)\) and by using inverse probability weights obtained via entropy
balancing (\citeproc{ref-hainmueller_entropy_2012}{Hainmueller, 2012}),
which enforce covariate balance within each origin stratum and mitigate
selection on observables. Positivity is assessed by inspecting the
distribution of the entropy-balancing weights and trimming observations
with extreme weights, thus restricting inference to regions of common
support. SUTVA holds by treating the destination class as a single
exposure measured before \((Y)\) and assuming no interference. Together,
the DAG and assumptions define conditions for identifying the causal
effect of intergenerational social mobility on preferences for
commodification.

\subsection{Inquiry and estimand}\label{inquiry-and-estimand}

The causal inquiry guiding this study asks: How does intergenerational
social mobility affect individuals' preferences for the commodification
of pensions in Chile? Substantively, I am interested in the effect of
mobility relative to immobility within a given class of origin.

Following the potential outcomes framework proposed by Breen and Ermisch
(\citeproc{ref-breen_effects_2024}{2024, p. 473}), the estimand of
interest is an average treatment effect on the treated (ATT) defined
within each origin class. I focus on the specific case in which the
counterfactual destination corresponds to class immobility \((k' = j)\).
In other words, I ask how the outcome among people from origin \emph{j}
who entered destination \emph{k} would have bee different if those
people had, counterfactually, remained in origin \emph{j} instead
(inmobility).

Formally, for individuals with origin \((O=j)\) who attain destination
\((D=k)\), the estimand is:

\begin{equation}\phantomsection\label{eq-att}{ 
ATT_{j,k,j} = E\big[Y(D = k) \mid O = j, D = k \big] - E\big[Y(D = j) \mid O = j, D = k \big]
}\end{equation}

which represents the mean causal effect of moving from origin class
\((j)\) to destination class \((k)\), compared to the counterfactual
outcome those same individuals would have exhibited had they instead
remained in \((j)\). Thus, this ATT captures the counterfactual contrast
within origin that lies at the heart of mobility effects: how
preferences for market-based pensions would differ if, for the same
upwardly or downwardly mobile individuals, their observed destination
class were replaced by immobility in their origin class.

\subsection{Data and variables}\label{data-and-variables}

\subsubsection{Data}\label{data}

This study draws on data from the Chilean Longitudinal Social Survey
(ELSOC) of the Center for Social Conflict and Cohesion Studies (COES).
This survey is a nationally representative panel study of the urban
adult population in Chile, conducted annually between 2016 and 2023.
Designed to examine individuals' attitudes, emotions, and behaviors
regarding social conflict and cohesion, ELSOC employs a probabilistic,
stratified, clustered, and multistage sampling design covering both
major urban centers and smaller cities. The sampling frame was
proportionally stratified into six categories of urban population size
(e.g., large and small cities), followed by a random selection of
households within 1,067 city blocks. The target population includes men
and women aged 18 to 75 who are habitual residents of private dwellings.

Because respondent occupation\footnote{Parental occupation was collected
  only in 2023 as open-text labels. I coded these texts to the ISCO-08
  two-digit scheme using the National Institute of Statistics of Chile's
  automated coding API and retained cases with high-confidence matches
  (\textgreater95\%). Treating parental occupation as a time-invariant
  origin attribute, I carried these codes back to 2016 and 2018.} is not
measured in every wave, I restrict the analysis to the 2016, 2018, and
2023 waves. After listwise deletion and restricting to key variables
(respondent occupation and the outcome measure), the final analytic
sample comprises \emph{N} = 3,435 observations nested within \emph{N} =
1,787 individuals (2016: 914; 2018: 1,377; 2023: 1,144). Consistent with
the study design, estimation proceeds within trajectory-specific
subsamples (e.g., low→low; low→middle), retaining movers and matched
non-movers from the same origin class. Then, I apply weights to
construct the corresponding counterfactual control sample for each
trajectory. Following Breen and Ermisch
(\citeproc{ref-breen_effects_2024}{2024}), I use the three waves in the
estimations that follow, allowing for correlation between the
observations for each individual in calculating the standard errors
(i.e.~clustering on the personal identifier).

\subsubsection{Outcome variable}\label{outcome-variable}

The outcome variable measures preferences regarding the commodification
of pensions, operationalized with a single item addressing how strongly
individuals justify conditioning access to old-age pension benefits
based on individual income. Respondents were asked: ``Is it fair in
Chile that people with higher incomes have better pensions than people
with lower incomes?'' and answered on a five-point Likert scale ranging
from 1 (``strongly disagree'') to 5 (``strongly agree''). For estimation
and interpretability, I construct a binary indicator in which values
4--5 indicate agreement with market-based access to old-age pensions
(coded 1), while values 1--3 indicate non-agreement (coded 0).
Substantively, the dependent variable is meant to capture
\emph{endorsement} of pension commodification: only responses in the
upper tail of the scale represent an active, explicit justification of
market-based differentiation, whereas neutrality corresponds to an
absence of such endorsement and is therefore treated as closer to
non-support than to support.\footnote{A potential concern is that the
  neutral category may reflect genuine ambivalence rather than
  disagreement. To address this, I conduct robustness checks using
  alternative codings: (i) excluding neutral respondents from the
  analysis; (ii) combining the neutral and agreement categories (3--5
  vs.~1--2); and (iii) estimating models with the original 5-point scale
  using linear specifications. The main mobility effects reported below
  are substantively unchanged across these specifications (see
  \hyperref[anexo]{Supplementary Material}).} This binary coding thus
yields a clear contrast between respondents who \emph{support}
market-based differentiation in pensions and those who do not. I use
this item for two main reasons: first, to enable comparisons with
existing work on market-based justice in social policy
(\citeproc{ref-castillo_perceptions_2025}{Castillo, Laffert, et al.,
2025}; \citeproc{ref-lindh_public_2015}{Lindh, 2015};
\citeproc{ref-otero_power_2024}{Otero \& Mendoza, 2024}); and second,
because it taps two core dimensions of market-based welfare
distribution---(i) the centrality of economic resources as a criterion
for allocating outcomes and (ii) the framing of pensions as tradable
commodities that can be bought and sold according to ability to pay
(\citeproc{ref-lindh_public_2015}{Lindh, 2015}).

\subsubsection{Treatment}\label{treatment}

\paragraph{Intergenerational occupational
mobility}\label{intergenerational-occupational-mobility}

I treat intergenerational occupational mobility as an exposure
indicating whether respondents occupy a different occupational status
than their fathers, closely following Breen and Ermisch's
(\citeproc{ref-breen_effects_2024}{2024}) causal framework. Occupational
assignment proceeds in two steps. First, I derive occupational status
for both origin (father) and destination (respondent) from two-digit
ISCO-08 codes using the International Socio-Economic Index of
Occupational Status (ISEI). Second, I group these ISEI scores into
terciles (low, middle, high), yielding a three-category schema for both
origin and destination strata (see Table~\ref{tbl-matrix}).
Substantively, I use ISEI because occupational status is a core
indicator of stratification and a reliable mobility measure: it locates
jobs on a hierarchical continuum defined by incumbents' typical
education level and earnings
(\citeproc{ref-hauser_intergenerational_2010}{Hauser, 2010};
\citeproc{ref-salgado_uplifting_2025}{Salgado et al., 2025}), and thus
approximates long-run socio-economic position more closely than
volatile, single-year income, which is also prone to recall and
reporting error for both parental and own resources
(\citeproc{ref-barone_rise_2022}{Barone et al., 2022}). This
occupational focus is consistent with a long tradition that interprets
the occupational structure as the backbone of the stratification system
and a key determinant of life chances
(\citeproc{ref-wrightUnderstandingClass2015}{Wright, 2015}). Moreover,
ISEI was explicitly designed to harmonize occupational stratification
across class schemes and SES measures
(\citeproc{ref-ganzeboom_internationally_1996}{Ganzeboom \& Treiman,
1996}), supporting cross-study comparability. In short, using ISEI
aligns the measurement of origin and destination on a common vertical
status continuum that is theoretically meaningful for social mobility
research and empirically feasible given the available information of
ELSOC's data.

\begin{table}

\caption{\label{tbl-matrix}Occupational mobility by occupational
groups.}

\centering{

\centering
\begin{tabular}{cccccc}
\toprule
Father↓ & Offspring→ & Low & Middle & High & Total\\
\midrule
Low &  & 44.7\%   (501) & 33.6\%   (377) & 21.7\%   (244) & 100.0\% (1,122)\\
Middle &  & 35.1\%   (388) & 34.9\%   (385) & 30.0\%   (331) & 100.0\% (1,104)\\
High &  & 24.6\%   (297) & 28.8\%   (348) & 46.7\%   (564) & 100.0\% (1,209)\\
Total &  & 34.5\% (1,186) & 32.3\% (1,110) & 33.2\% (1,139) & 100.0\% (3,435)\\
\bottomrule
\end{tabular}

}

\end{table}%

\paragraph{Pre-treatment control variables for selection into
treatment}\label{pre-treatment-control-variables-for-selection-into-treatment}

Within each origin stratum (\(O = j\)), I estimate average treatment
effects on the treated (ATT) by comparing movers to otherwise similar
non-movers from the same origin. To reduce bias from non-random
selection into mobility based on observed characteristics, I preprocess
the data using entropy balancing, reweighting only the control group
(the immobile) so that the distribution of pre-treatment origin
covariates \(C\) among non-movers matches that of movers on selected
moments. Entropy balancing implements a maximum-entropy reweighting
scheme that chooses unit weights for non-movers to satisfy a set of
balance constraints (e.g., equality of means and, where relevant, higher
moments), while keeping the new weights as close as possible to the
original weights (\citeproc{ref-hainmueller_entropy_2012}{Hainmueller,
2012}). Substantively, this procedure addresses selection on
observables: it reweights immobile respondents so that, within each
origin class, they resemble mobile respondents in their observed
background characteristics, making any remaining differences in the
outcome more plausibly attributable to mobility rather than to
pre-existing measured advantages.

The selection of the adjustment set \(C\) follows two principles. First,
it mirrors the logic in Breen and Ermisch
(\citeproc{ref-breen_effects_2024}{2024}), who condition on attributes
determined in childhood or earlier (e.g., parental resources, household
structure, ascriptive traits). Second, it is constrained to origin-side
characteristics available in ELSOC. Guided by theory and Chilean
evidence on the determinants of relative mobility (mainly
\citeproc{ref-brunori_inequality_2025}{Brunori et al., 2025};
\citeproc{ref-espinoza_movilidad_2014}{Espinoza \& Núñez, 2014};
\citeproc{ref-salgado_uplifting_2025}{Salgado et al., 2025};
\citeproc{ref-torche_unequal_2005}{Torche, 2005}), I include six
covariates covering family resources, household composition, and
ascriptive status: (a) parental education (highest of father/mother), in
10 ordinal categories from no schooling to postgraduate studies; (b)
co-residence with both parents at age 15 (0 = no, 1 = yes); (c)
nationality (0 = non-Chilean, 1 = Chilean); (d) age in years; (e) sex (0
= male, 1 = female); and (f) indigenous ethnicity (0 = no, 1 = yes).
This set of covariates is fixed prior to the destination class and
constitutes the maximum set on the origin side in ELSOC, thus allowing
for the plausible capture of both family and contextual influences that
may affect career trajectories, making the conditional independence
hypothesis more credible within each \(O = j\). Descriptive statistics
for all variables are presented in Supplementary Material Table A1.

After achieving balance, I construct stabilized IPW-ATT weights by
setting treated (mobile) cases to weight 1 and assigning the
entropy-balancing weights to controls, rescaled within each origin
stratum so that the mean weight remains close to one. This aligns the
counterfactual distribution of non-movers with the covariate profile of
movers in each origin class (see \hyperref[anexo]{Supplementary
Material} for balance diagnostics). The resulting weights are then used
in the outcome stage to estimate the causal effect of intergenerational
mobility on preferences for the commodification of pensions.

\subsubsection{Effect heterogeneity}\label{effect-heterogeneity}

To explore whether mobility effects vary across key contextual
dimensions, I examine meritocratic perceptions as a moderator.
Meritocracy is captured with two items
(\citeproc{ref-young_rise_1958}{Young, 1958}), one referring to effort
(``In Chile, people are rewarded for their efforts'') and one to talent
(``In Chile, people are rewarded for their intelligence and skills''),
combined into a single index and dichotomized into low (≤ 3) and high (≥
4) meritocracy. Because meritocracy is measured after mobility, it is
used not as a causal effect modifier but as a post-treatment moderator
to probe potential mechanisms. I estimate interaction models between
mobility and meritocracy and report conditional treatment effects
(simple slopes) at low and high meritocratic levels.

\subsubsection{Controls}\label{controls}

All models include the same pre-treatment covariates \(C\) used in the
IPW construction, not to re-balance groups (entropy balancing already
does so), but to (i) block backdoor paths from \(C\) to the outcome
\(Y\) as implied by the DAG, and (ii) achieve double robustness:
estimates remain consistent if either the weighting model or the outcome
model is correctly specified. Concretely, I adjust for (a) father's
educational level, (b) co-residence with both parents at age 15, (c)
nationality, (d) age, (e) sex, and (f) ethnicity, along with wave fixed
effects to absorb secular trends.

\subsection{Analytical strategy}\label{analytical-strategy}

Because the dependent variable---the preference for pension
commodification---is binary \((Y_{it}\in{0,1})\), I estimate weighted
linear probability models (WLS) using stabilized inverse probability
weights (IPW) obtained via entropy balancing. Although OLS is often
questioned with binary outcomes, it consistently estimates the
conditional mean \(E(Y|X)\) under standard exogeneity
\(E[\varepsilon_i|X_i]=0\), and heteroskedasticity can be handled with
robust or clustered standard errors
(\citeproc{ref-wooldridge_introductory_2009}{Wooldridge, 2009},
ch.~7.5). WLS with IPW further improves efficiency and implements the
ATT estimand by recreating the counterfactual distribution of non-movers
(\citeproc{ref-gelman_regression_2020}{Gelman et al., 2020, pp.
270--272}). Linear models are preferred here because coefficients are
directly interpretable as average differences in predicted
probabilities, whereas non-linear models complicate both weighting and
interpretation (\citeproc{ref-gelman_regression_2020}{Gelman et al.,
2020}, ch.~13).

Formally, the model is:

\begin{equation}\phantomsection\label{eq-final}{
Y_{it}=\alpha+\beta T_i+X_i'\gamma+\lambda_t+\varepsilon_{it}
}\end{equation}

where \(Y_{it}\) indicates whether individual \(i\) in wave \(t\)
supports more market-based pension access; \(\alpha\) is the baseline
probability for the reference categories; \(\beta T_i\) captures the
intergenerational mobility contrast: \(T_i=1\) when the observed
destination is \(D_i=k\) (mobile to \(k\)) and \(T_i=0\) when \(D_i=j\)
(immobile), so that \(\beta\) is the ATT within origin
\(O=j (\widehat\beta=\widehat{ATT}{j,k\mid j})\), i.e., the
percentage-point change in the probability of preferring pension
commodification from reaching \(k\) rather than remaining in \(j\). The
term \(X_i'\gamma\) includes the pre-treatment covariates \((C)\) used
to construct the entropy-balancing weights; \(\lambda_t\) are wave fixed
effects; and \(\varepsilon_{it}\) is idiosyncratic error. The analytic
sample is restricted to individuals sharing the same origin \((O=j)\).
Therefore, estimation uses pooled WLS regression with stabilized
IPW--ATT from entropy balancing and CR2 standard errors clustered by
individual to address heteroskedasticity and within-person dependence.

The specification is doubly robust: entropy-balancing weights are
combined with covariate adjustment so consistency holds if either the
weighting model or the outcome model is correctly specified, while also
improving efficiency
(\citeproc{ref-wooldridge_introductory_2009}{Wooldridge, 2009}). These
results are virtually identical to baseline models without covariates
(see \hyperref[anexo]{Supplementary Material} for complete models), with
no substantive changes in the mobility coefficients. Heterogeneous
effects by meritocratic perceptions are examined through interaction
models described in the previous section.

To evaluate the robustness of the findings, I implement two sets of
sensitivity analyses. First, I conduct standard robustness checks by
re-estimating the models under alternative codings of the outcome: (i)
excluding neutral responses from the analysis, (ii) combining neutral
responses with agreement (3--5 vs.~1--2), and (iii) using the original
5-point item in a linear specification. Second, I assess sensitivity to
unmeasured confounding using the omitted-variable--bias framework of
Cinelli and Hazlett (\citeproc{ref-cinelli_making_2020}{2020}), which
quantifies how strong an unobserved confounder would need to be to
explain away the estimated ATT.

\section{Results}\label{results}

\subsection{Descriptive analysis}\label{descriptive-analysis}

Figure~\ref{fig-alluvial} shows the annual frequencies of preferences
for pension commodification in 2016, 2018, and 2023. Each year presents
stacked percentages frequencies for the level of agreement and
disagreement. Overall, a large majority rejects market-based access to
old-age pension benefits, but this opposition has eased over time:
83.5\% in 2016, 81.2\% in 2018, and 71.8\% in 2023. The 2016--2018
change is modest, whereas 2023 registers a 9.4 point drop in
disagreement relative to 2018, mirrored by rising agreement: 16.5\%
(2016), 18.8\% (2018), 28.2\% (2023). Substantively, while most
respondents continue to oppose the idea that higher-income individuals
should obtain better pensions via the market, a non-trivial and growing
portion endorses this statement, with the sharpest expansion
concentrated in the latest wave (+9.4 points from 2018 to 2023).

\begin{figure}[H]

\centering{

\includegraphics[width=0.85\linewidth,height=\textheight,keepaspectratio]{paper_files/figure-pdf/fig-alluvial-1.pdf}

}

\caption{\label{fig-alluvial}Change in preferences for pension
commodification over time (2016, 2018, and 2023).}

\end{figure}%

Regarding the relationship between preferences for the commodification
of pensions and intergenerational occupational mobility,
Figure~\ref{fig-mean} shows the percentage of agreement with
market-based access to pension benefits according to mobility
trajectories and survey waves. Averaging across waves, agreement is
highest among the High-High immobile (27.8\%), followed by the
Middle-High upwardly mobile (24.6\%) and the High-Middle downwardly
mobile (24.0\%). The lowest levels are found among ascending Low-High
(17.3\%), and immobile Low-Low (17.4\%) groups. The specific profiles of
each wave accentuate these gradients in most trajectories, especially in
2023: agreement reaches 31.9\% for High-High, 29.4\% for upward
Middle-High, and peaks at 36.8\% for downward High-Middle (the highest
of all trajectories for that year). Descriptively, immobility at the top
is associated with greater support for commodification; among those who
move, support is greatest for the upward Middle-High trajectory and the
downward High-Middle trajectory, patterns that intensify in 2023. These
shifts anticipate the heterogeneity documented below.

\begin{figure}[H]

\centering{

\includegraphics[width=0.9\linewidth,height=\textheight,keepaspectratio]{paper_files/figure-pdf/fig-mean-1.pdf}

}

\caption{\label{fig-mean}Percentage agreement on preferences for pension
commodification according to mobility trajectory and time (2016, 2018,
and 2023).}

\end{figure}%

\subsection{Mobility effects models}\label{mobility-effects-models}

Figure~\ref{fig-reg} reports the double-robust estimates of the effect
of intergenerational occupational mobility on preferences for pension
commodification, obtained from weighted models that combine stabilized
IPW with covariate adjustment and fixed wave effects (see
\hyperref[anexo]{Supplementary Material} for complete models). Results
reveal a clear directional asymmetry between upward and downward
mobility trajectories. Among upwardly mobile respondents, only the
Middle→High trajectory exhibits a significant positive effect on support
for pension commodification (\(\beta\) = 0.09, 95\% CI {[}0.02,
0.16{]}). Interpreted as an origin-specific ATT, this estimate
compares---for individuals who actually moved from the middle to the
high socioeconomic status---their observed endorsement with the
counterfactual endorsement they would have expressed had they remained
immobile in the middle socioeconomic status. Holding all other
predictors constant, the point estimate implies a 9 percentage point
higher probability of endorsing market-based access to pension benefits
for Middle→High movers relative to their own immobility counterfactual,
statistically significant at the 95\% confidence level. In contrast, two
downward trajectories: High→Middle (\(\beta\) = −0.07, CI {[}−0.15,
−0.00{]}) and High→Low (\(\beta\) = −0.10, CI {[}−0.18, −0.02{]}), show
significant negative effects, suggesting that individuals who experience
downward mobility from higher-status origins express weaker preferences
for market-based pension benefits. The remaining pathways (Low→Middle,
Low→High, Middle→Low) display no significant differences relative to
their non-mobile counterparts. These patterns indicate that upward
movement within the upper segment (Middle→High) reinforces
pro-commodification attitudes, whereas downward movement from privileged
origins reduces them, providing favourable evidence for the first
hypothesis (\(H_1\)).

\begin{figure}[H]

\centering{

\includegraphics[width=0.9\linewidth,height=\textheight,keepaspectratio]{paper_files/figure-pdf/fig-reg-1.pdf}

}

\caption{\label{fig-reg}Effects of intergenerational occupational
mobility on preferences for pension commodification. Coefficient plot of
origin-specific ATT estimates comparing mobility versus immobility.
Estimates from pooled WLS with doubly robust adjustment, wave fixed
effects, and standard errors clustered by individual; bars show 95\%
confidence intervals.}

\end{figure}%

\subsection{The role of meritocracy}\label{the-role-of-meritocracy}

To probe whether meritocratic perceptions channel the impact of mobility
on preferences for pension commodification, I estimate interaction
models between mobility and meritocratic perceptions and examine the
conditional treatment effects of mobility at low and high levels of
meritocracy (see \hyperref[anexo]{Supplementary Material} for complete
models).

\begin{figure}[H]

\centering{

\includegraphics[width=0.9\linewidth,height=\textheight,keepaspectratio]{paper_files/figure-pdf/fig-interact-1.pdf}

}

\caption{\label{fig-interact}Effects of intergenerational occupational
mobility on preferences for pension commodification, by meritocratic
perception. Coefficient plot of origin-specific ATT estimates comparing
mobility versus immobility, estimated from interaction models (T ×
Merit). Points show the marginal effects (simple slopes) of the mobility
treatment at low and high merit; bars are 95\% confidence intervals.
Estimates from pooled WLS with doubly robust adjustment, wave fixed
effects, and standard errors clustered by individual.}

\end{figure}%

Figure~\ref{fig-interact} displays the origin-specific ATT estimates
from the meritocracy interaction models as marginal effects of the
mobility treatment at low and high meritocratic perceptions. Across
trajectories, mobility effects among low-merit respondents are uniformly
small, imprecise, and centered around zero for both upward and downward
moves. Among high-merit respondents, only upward mobility from middle to
high status shows a clear pro-commodification pattern, with a positive
and statistically significant effect on preferences for pension
commodification (\(\beta = 0.22\), 95\% CI {[}0.05, 0.38{]}). All other
trajectories---Low→Middle, Low→High, Middle→Low, High→Middle, and
High→Low---remain statistically indistinguishable from zero.
Nonetheless, the point estimates broadly align with the theoretical
expectations for upward moves: upwardly mobile individuals with stronger
meritocratic perceptions tend to display higher support for pension
commodification than their low-merit counterparts. By contrast, downward
trajectories exhibit a pattern opposite to the mechanism hypothesis,
with high-merit respondents generally showing more de-commodifying
attitudes than those with weaker meritocratic perceptions. However,
formal interaction tests do not detect robust heterogeneity in mobility
effects by meritocratic perceptions (see \hyperref[anexo]{Supplementary
Material} for full interaction models): even for the Middle→High
trajectory, the simple effect under high meritocracy does not differ
significantly from its low-merit counterpart. Taken together, these
results provide no convincing evidence that meritocratic perceptions
moderate mobility effects on preferences for pension commodification.
Accordingly, hypotheses \(H_{2a}\) and \(H_{2b}\)---which posit stronger
pro-market reactions to upward mobility and attenuated de-commodifying
reactions to downward mobility under high meritocracy---are not
supported.

\subsection{Robustness check and sensitivity
analysis}\label{robustness-check-and-sensitivity-analysis}

As a robustness check, I re-estimate the models under three alternative
codings of the outcome: (i) recoding the neutral category as agreement
(3--5 vs.~1--2), (ii) excluding neutral responses from the analysis, and
(iii) using the original 5-point item in a linear specification. Across
these alternative models (see \hyperref[anexo]{Supplementary Material}
for complete tables), the substantive pattern of the main effects is
preserved: the Middle→High trajectory remains positively associated with
support for pension commodification, and the High→Low trajectory remains
negatively associated with it, with both coefficients generally
increasing in magnitude. By contrast, the negative High→Middle effect
becomes statistically indistinguishable from zero in the robustness
specifications, indicating that this particular estimate is more
sensitive to how the outcome is coded than the other two trajectories.

To assess the extent to which the estimated mobility effects may be
driven by unmeasured confounding, I implement the sensitivity analysis
proposed by Cinelli and Hazlett
(\citeproc{ref-cinelli_making_2020}{2020}), which quantifies how strong
an unobserved confounder would need to be---in terms of its joint
explanatory power for both treatment and outcome---to attenuate or
explain away the ATT, using gender as a benchmark covariate (the
predictor with the largest partial association with the outcome aside
from treatment). For the Middle→High trajectory
(\(\beta \approx 0.09\)), partial \(R^2 \approx 0.013\), an unobserved
confounder would need to explain about 10--11\% of the residual variance
of both treatment and outcome to reduce the effect to zero and roughly
4\% to render it statistically insignificant, substantially more than
gender does. For the High→Low effect (\(\beta \approx -0.10\)), partial
\(R^2 \approx 0.012\), the required strength is of similar magnitude
(around 10--11\% to explain away the effect and about 4\% to remove
significance), again exceeding the explanatory power of the main
observed covariates. By contrast, the High→Middle effect
(\(\beta \approx -0.07\)), partial \(R^2 \approx 0.007\) is somewhat
less robust: a confounder explaining around 8\% of residual variance in
both treatment and outcome could nullify the estimate, and one
explaining about 2\% could make it non-significant. Overall, the
sensitivity analysis suggests that the positive Middle→High and negative
High→Low effects are moderately robust to unmeasured confounding,
whereas the High→Middle effect is more vulnerable to relatively modest
omitted variables (see \hyperref[anexo]{Supplementary Material} for
complete tables).

\section{Discussion and conclusions}\label{discussion-and-conclusions}

Pension privatisation lies at the heart of Latin America's trajectory of
commodification since the 1980s, reshaping how societies organise
old-age security (\citeproc{ref-ferre_welfare_2023}{Ferre, 2023}). By
replacing public pay-as-you-go pillars with individual capitalisation
schemes---privately managed defined-contribution accounts---pensions
became governed by contribution-based entitlements and market
performance (\citeproc{ref-huber_political_2000}{Huber \& Stephens,
2000}; \citeproc{ref-oecd_pensions_2023}{OECD, 2023}), with Chile as a
compelling example. This shift both restructured who bears old-age risks
and redefined the principles deemed legitimate for allocating pension
resources (\citeproc{ref-borzutzky_pension_2012}{Borzutzky, 2012};
\citeproc{ref-madero-cabib_private_2019}{Madero-Cabib et al., 2019};
\citeproc{ref-solimano_rise_2021}{Solimano, 2021}). From a moral-economy
perspective, these agreements generate feedback on public beliefs,
fostering opinions that access to core social services should reflect
income and individual responsibility, extending the principles of market
justice (\citeproc{ref-mau_inequality_2015}{Mau, 2015}). Existing
research shows that socioeconomic position and beliefs about inequality
shape support for market-based access to welfare services
(\citeproc{ref-busemeyer_skills_2014}{Busemeyer, 2014};
\citeproc{ref-castillo_perceptions_2025}{Castillo, Laffert, et al.,
2025}; \citeproc{ref-immergut_it_2020}{Immergut \& Schneider, 2020};
\citeproc{ref-lindh_public_2015}{Lindh, 2015};
\citeproc{ref-svallfors_political_2007}{Svallfors, 2007}). However, we
still know little about how citizens assess the fairness of pension
commodification in such settings, and how social mobility and
meritocratic perceptions shape these judgements. This article addresses
this gap by examining whether intergenerational mobility causally
affects support for pension commodification in Chile and whether this
effect varies by meritocratic perceptions. Drawing on survey data and
under explicit identification assumptions, I apply Breen and Ermisch's
(\citeproc{ref-breen_effects_2024}{2024}) causal framework of mobility
effects, treating destination class as a treatment, conditioning on
origin class, and estimating heterogeneous destination effects.

The findings reveal a tension in Chileans' judgements of market-based
pension welfare. Across 2016, 2018 and 2023, large majorities reject the
claim that it is fair for higher-income people to obtain better pensions
than those with lower incomes. However, a non-trivial---and
growing---minority endorses this principle, with the sharpest increase
between 2018 and 2023. Market-based pension allocation thus remains a
minority view. However, it is becoming more common, in line with
longitudinal evidence of rising support for income-based access to
welfare services in Chile
(\citeproc{ref-castillo_perceptions_2025}{Castillo, Laffert, et al.,
2025}). This coexistence unfolds in a system of mandatory, privately
managed, defined-contribution accounts that tightly link old-age income
to earnings, contribution density and financial returns. This design has
produced low and unequal pensions and state interventions against
old-age poverty (\citeproc{ref-galvez_pensiones_2024}{Gálvez et al.,
2024}; \citeproc{ref-madero-cabib_private_2019}{Madero-Cabib et al.,
2019}), while framing pensions as personal investments rather than
solidarity-based rights and promoting individual responsibility and
contribution-based desert
(\citeproc{ref-borzutzky_pension_2012}{Borzutzky, 2012}). The Chilean
case fits evidence that supports the claim that income-based welfare
access is stronger in liberal, marketised regimes
(\citeproc{ref-lindh_public_2015}{Lindh, 2015}). This coexistence also
echoes recent pension conflict---from the ``No+AFP'' protests to
post-pandemic withdrawals and the ``Con mi plata NO'' mobilisation
(\citeproc{ref-galvez_con_2023}{Gálvez, 2023})---and illustrates mixed
policy feedback: the same regime generates rights-based critiques and
proprietarian defences of pension commodification. Together, these
patterns show that, despite discontent and a legitimacy crisis, support
for pension commodification remains salient, raising the question of
which social trajectories and beliefs underpin it.

Turning to the social mobility hypothesis, the results support a causal
role of intergenerational mobility in shaping preferences for pension
commodification. Rise from middle to high occupational status is
causally related to significantly greater support for pension
commodification, whereas the opposite holds for descent from high to
middle and from high to low status, relative to immobility in the origin
class. These patterns are consistent with a myopic material
self-interest mechanism (\citeproc{ref-helgason_longterm_2023}{Helgason
\& Rehm, 2023}) in Chile's pension system, where benefits mainly depend
on individual contributions: upwardly mobile middle→high individuals
gain income, employment stability and contribution capacity, and can
better bear investment risk, making a contribution-linked, market-based
pension regime more attractive. By contrast, those moving down from high
positions lose resources and security, become more exposed to
labour-market risks and depend more on public or solidarity-based
pillars, which helps explain their lower support for income-based
allocation (\citeproc{ref-busemeyer_welfare_2020}{Busemeyer \& Iversen,
2020}; \citeproc{ref-lindh_public_2015}{Lindh, 2015}). The fact that
significant effects emerge only for trajectories into and out of the
upper occupational status resonates with research on static social class
and pensions, which shows that higher socioeconomic groups and those
obtaining higher financial returns are more supportive of market-based
welfare and private pension arrangements
(\citeproc{ref-busemeyer_welfare_2020}{Busemeyer \& Iversen, 2020};
\citeproc{ref-castillo_socialization_2024}{Castillo et al., 2024};
\citeproc{ref-kerner_pension_2020}{Kerner, 2020};
\citeproc{ref-otero_power_2024}{Otero \& Mendoza, 2024}). It also aligns
with evidence on social mobility and redistribution, where upward
mobility is linked to less redistributive attitudes and lower support
for public spending on pensions, while downward mobility has the
opposite effect (\citeproc{ref-alesina_intergenerational_2018}{Alesina
et al., 2018}; \citeproc{ref-ares_changing_2020}{Ares, 2020};
\citeproc{ref-gugushvili_subjective_2017}{Gugushvili, 2017};
\citeproc{ref-rodriguez_intergenerational_2024}{Rodriguez \&
Matilla-Garcia, 2024}; \citeproc{ref-schmidt_experience_2011}{Schmidt,
2011}). Still, material self-interest may not be the only channel
through which mobility affects these judgements, raising the question of
whether meritocratic perceptions condition these effects by providing a
normative lens to interpret mobility experiences.

The second set of hypotheses examined whether meritocratic perceptions
condition the effects of intergenerational mobility. Building on
acculturation accounts and self-serving bias arguments
(\citeproc{ref-gugushvili_subjective_2017}{Gugushvili, 2017};
\citeproc{ref-jaime-castillo_social_2019}{Jaime-Castillo \&
Marqués-Perales, 2019}), I expected mobility to affect preferences for
pension commodification not only through material self-interest but also
through shifts in normative frames of distributive justice. Prior
research shows that upwardly mobile individuals tend to interpret their
trajectory through a meritocratic lens, reading their success as
evidence that effort and ability are rewarded and reinforcing
individualistic attributions of poverty and wealth. In contrast,
downward or blocked mobility is more often associated with structural
explanations of inequality and more ambivalent meritocratic beliefs,
sometimes including the internalisation of status loss as personal
failure (\citeproc{ref-atria_economic_2020}{Atria et al., 2020};
\citeproc{ref-bucca_merit_2016}{Bucca, 2016};
\citeproc{ref-davidai_why_2018}{Davidai, 2018};
\citeproc{ref-day_movin_2017}{Day \& Fiske, 2017};
\citeproc{ref-deng_its_2025}{Deng \& Wang, 2025};
\citeproc{ref-gugushvili_intergenerational_2016}{Gugushvili, 2016b};
\citeproc{ref-mijs_how_2022}{Mijs \& Hoy, 2022}). Following Mau's
(\citeproc{ref-mau_inequality_2015}{2015}) account of the ``majority
class'', I therefore expected upward mobility under strong meritocratic
convictions to be especially conducive to endorsing commodified
pensions, and strong meritocracy to dampen the usual de-commodifying
effect of downward mobility. Empirically, however, the interaction
models offer little support for this mechanism. Meritocratic perceptions
do not systematically moderate the mobility effects identified in the
primary analysis, and the few trajectory-specific patterns that
tentatively point in the expected direction are small and imprecisely
estimated. This result contrasts with survey and experimental studies
showing that upward mobility experiences or exposure to self-made
narratives reduce support for redistribution and public pension spending
partly by reinforcing meritocratic convictions
(\citeproc{ref-deng_its_2025}{Deng \& Wang, 2025};
\citeproc{ref-gugushvili_subjective_2017}{Gugushvili, 2017};
\citeproc{ref-matamoros-lima_social_2025}{Matamoros-Lima et al., 2025};
\citeproc{ref-molina_its_2019}{Molina et al., 2019};
\citeproc{ref-schmidt_experience_2011}{Schmidt, 2011};
\citeproc{ref-shariff_income_2016}{Shariff et al., 2016}). One
possibility is that, in this context, intergenerational occupational
mobility---as operationalised here---has limited impact on meritocratic
perceptions, which appear more weakly tied to mobility trajectories than
previous research suggests and more strongly anchored in broader
socialisation processes and class-specific cultures.

This study makes several contributions. Substantively, it offers one of
the first systematic analyses of preferences for pension commodification
in Chile's highly unequal and marketised pension regime, addressing
Mau's (\citeproc{ref-mau_inequality_2015}{2015}) question of why core
elements of neoliberal welfare arrangements gain public acceptance. By
showing that upward and downward intergenerational mobility into and out
of upper-class positions causally shape support for market-based pension
allocation, the findings highlight social mobility as a key source of
legitimation for commodified pensions. Second, the article advances
research on market justice and market-based welfare preferences by
moving beyond static class locations and showing how intergenerational
occupational mobility structures support for pension commodification in
ways that partly echo, but also refine, existing evidence on class,
income and financial returns. Third, the null results for the moderating
role of meritocratic perceptions nuance accounts that tightly link
mobility, meritocracy and the embrace of neoliberal welfare: in this
context, mobility effects on pension commodification preferences appear
to operate independently of meritocratic beliefs. Finally, the study
contributes methodologically by applying Breen and Ermisch's
(\citeproc{ref-breen_effects_2024}{2024}) causal framework of mobility
effects to survey data, formalising mobility as origin-specific
treatment effects and estimating these using linear probability models
with inverse probability weighting.

Despite these contributions, the study has limitations that should be
acknowledged and read as avenues for future improvement. First, the
causal estimates hinge on the construction of inverse probability
weighting within Breen and Ermisch's
(\citeproc{ref-breen_effects_2024}{2024}) framework. Identification
relies on conditional independence given a relatively limited set of
origin-side covariates, together with positivity and SUTVA. Entropy
balancing helps to enforce covariate balance and mitigate selection on
observables, but unmeasured confounding cannot be ruled out. Sensitivity
analyses suggest moderate robustness to omitted variables; therefore,
the estimates should be interpreted with caution. Second, while the
analysis identifies mobility effects on preferences, it does not explain
why or how they arise. The results are analysed through material
self-interest and acculturation mechanisms. Still, the study does not
implement formal causal mediation analysis, nor can it determine whether
meritocratic perceptions function as mediators rather than mere
correlates. Third, the findings are context-specific: they are derived
from Chile, a highly unequal and strongly marketised pension regime,
which enhances internal validity but limits external validity. The
patterns documented here, therefore, cannot be straightforwardly
generalised to countries with different welfare institutions, mobility
regimes or levels of inequality, and should be seen as a theoretically
informed case study rather than a cross-national pattern.

Future research could extend these findings in several directions.
First, there is a need for richer measures of preferences for pension
commodification. This study relies on a single item that conflates
income-based allocation and fairness judgments; better items that
separate justice principles and dimensions of commodification would
allow a finer assessment of how people evaluate pensions as
income-conditioned commodities. Second, methodological limitations could
be addressed by including more detailed origin-side covariates and
life-course data, repeated measures of meritocratic perceptions, and
causal mediation analyses that unpack mechanisms such as material
self-interest, risk exposure, and meritocratic frames, alongside
alternative measures of mobility based on class schemas, income
trajectories, or wealth. Third, comparative designs would clarify how
institutional context shapes the attitudinal consequences of
intergenerational mobility. Applying similar causal strategies in
countries with different welfare regimes, levels of inequality, mobility
patterns, and degrees of pension marketisation would show whether the
Chilean pattern is context-specific or part of a broader configuration
linking mobility, merit and support for market-based welfare.
Ultimately, clarifying when and for whom market allocation of pensions
is judged fair is crucial to understanding the legitimate reach of
welfare commodification in unequal, marketised societies.

\section{References}\label{references}

\phantomsection\label{refs}
\begin{CSLReferences}{1}{0}
\bibitem[\citeproctext]{ref-alesina_intergenerational_2018}
Alesina, A., Stantcheva, S., \& Teso, E. (2018). Intergenerational
{Mobility} and {Preferences} for {Redistribution}. \emph{American
Economic Review}, \emph{108}(2), 521--554.
\url{https://doi.org/10.1257/aer.20162015}

\bibitem[\citeproctext]{ref-araujo_desafios_2012}
Araujo, K., \& Martuccelli, D. (2012). \emph{Desafíos comunes: Retrato
de la sociedad chilena y sus individuos} (1a. ed). Santiago: LOM
Ediciones.

\bibitem[\citeproctext]{ref-arenas_sistemas_2019}
Arenas, A. (2019). \emph{{Los sistemas de pensiones en la encrucijada:
desafíos para la sostenibilidad en América Latina}}. Santiago: Naciones
Unidas, CEPAL.

\bibitem[\citeproctext]{ref-ares_changing_2020}
Ares, M. (2020). Changing classes, changing preferences: How social
class mobility affects economic preferences. \emph{West European
Politics}, \emph{43}(6), 1211--1237.
\url{https://doi.org/10.1080/01402382.2019.1644575}

\bibitem[\citeproctext]{ref-arza_pension_2008}
Arza, C. (2008). Pension {Reform} in {Latin America}: {Distributional
Principles}, {Inequalities} and {Alternative Policy Options}.
\emph{Journal of Latin American Studies}, \emph{40}(1), 1--28.
\url{https://doi.org/10.1017/S0022216X07003616}

\bibitem[\citeproctext]{ref-atria_economic_2020}
Atria, J., Castillo, J., Maldonado, L., \& Ramirez, S. (2020). Economic
{Elites}' {Attitudes Toward Meritocracy} in {Chile}: {A Moral Economy
Perspective}. \emph{American Behavioral Scientist}, \emph{64}(9),
1219--1241. \url{https://doi.org/10.1177/0002764220941214}

\bibitem[\citeproctext]{ref-barone_rise_2022}
Barone, C., Hertel, F. R., \& Smallenbroek, O. (2022). The rise of
income and the demise of class and social status? {A} systematic review
of measures of socio-economic position in stratification research.
\emph{Research in Social Stratification and Mobility}, \emph{78},
100678. \url{https://doi.org/10.1016/j.rssm.2022.100678}

\bibitem[\citeproctext]{ref-barriga_pensiones_2024}
Barriga, F., \& Kremerman, M. (2024). {Pensiones sin Seguridad Social:
\textquestiondown Cómo se calcula el monto de las pensiones en Chile?}
\emph{Fundación Sol}.

\bibitem[\citeproctext]{ref-batruch_belief_2023}
Batruch, A., Jetten, J., Van De Werfhorst, H., Darnon, C., \& Butera, F.
(2023). Belief in {School Meritocracy} and the {Legitimization} of
{Social} and {Income Inequality}. \emph{Social Psychological and
Personality Science}, \emph{14}(5), 621--635.
\url{https://doi.org/10.1177/19485506221111017}

\bibitem[\citeproctext]{ref-benabou_social_2001}
Benabou, R., \& Ok, E. A. (2001). Social {Mobility} and the {Demand} for
{Redistribution}: {The Poum Hypothesis}. \emph{The Quarterly Journal of
Economics}, \emph{116}(2), 447--487.
\url{https://doi.org/10.1162/00335530151144078}

\bibitem[\citeproctext]{ref-blair_research_2023}
Blair, G., Coppock, A., \& Humphreys, M. (2023). \emph{Research design
in the social sciences: Declaration, diagnosis, and redesign}. Princeton
(N.J.) Oxford: Princeton University press.

\bibitem[\citeproctext]{ref-boccardo_30_2020}
Boccardo, G. (2020). \emph{30 años de privatizaciones en {Chile}: Lo que
la pandemia reveló} (Nodo XXI). Santiago.

\bibitem[\citeproctext]{ref-borzutzky_pension_2012}
Borzutzky, S. (2012). Pension market failure in {Chile}: Foundations,
analysis and policy reforms. \emph{Journal of Comparative Social
Welfare}, \emph{28}(2), 103--112.
\url{https://doi.org/10.1080/17486831.2012.655980}

\bibitem[\citeproctext]{ref-borzutzky_chiles_2016}
Borzutzky, S., \& Hyde, M. (2016). Chile's private pension system at 35:
Impact and lessons. \emph{Journal of International and Comparative
Social Policy}, \emph{32}(1), 57--73.
\url{https://doi.org/10.1080/21699763.2016.1148623}

\bibitem[\citeproctext]{ref-breen_effects_2024}
Breen, R., \& Ermisch, J. (2024). The {Effects} of {Social Mobility}.
\emph{Sociological Science}, \emph{11}, 467--488.
\url{https://doi.org/10.15195/v11.a17}

\bibitem[\citeproctext]{ref-brunori_inequality_2025}
Brunori, P., Ferreira, F. H. G., \& Neidhöfer, G. (2025). Inequality of
opportunity and intergenerational persistence in {Latin America}.
\emph{Oxford Open Economics}, \emph{4}(Supplement\_1), i167--i199.
\url{https://doi.org/10.1093/ooec/odae021}

\bibitem[\citeproctext]{ref-bucca_merit_2016}
Bucca, M. (2016). Merit and blame in unequal societies: {Explaining
Latin Americans}' beliefs about wealth and poverty. \emph{Research in
Social Stratification and Mobility}, \emph{44}, 98--112.
\url{https://doi.org/10.1016/j.rssm.2016.02.005}

\bibitem[\citeproctext]{ref-busemeyer_skills_2014}
Busemeyer, M. (2014). \emph{Skills and {Inequality}: {Partisan Politics}
and the {Political Economy} of {Education Reforms} in {Western Welfare
States}}. Cambridge University Press.

\bibitem[\citeproctext]{ref-busemeyer_positive_2021}
Busemeyer, M., Abrassart, A., \& Nezi, R. (2021). Beyond {Positive} and
{Negative}: {New Perspectives} on {Feedback Effects} in {Public Opinion}
on the {Welfare State}. \emph{British Journal of Political Science},
\emph{51}(1), 137--162. \url{https://doi.org/10.1017/S0007123418000534}

\bibitem[\citeproctext]{ref-busemeyer_welfare_2020}
Busemeyer, M., \& Iversen, T. (2020). The {Welfare State} with {Private
Alternatives}: {The Transformation} of {Popular Support} for {Social
Insurance}. \emph{The Journal of Politics}, \emph{82}(2), 671--686.
\url{https://doi.org/10.1086/706980}

\bibitem[\citeproctext]{ref-campbell_institutional_2020}
Campbell, J. L. (2020). \emph{Institutional {Change} and
{Globalization}}. Princeton University Press.
\url{https://doi.org/10.2307/j.ctv131bw68}

\bibitem[\citeproctext]{ref-canalesceron_sujeto_2021}
Canales Cerón, M., Orellana Calderón, V. S., \& Guajardo Mañán, F.
(2021). Sujeto y cotidiano en la era neoliberal: El caso de la educación
chilena. \emph{Revista Mexicana de Ciencias Políticas y Sociales},
\emph{67}(244).
\url{https://doi.org/10.22201/fcpys.2448492xe.2022.244.70386}

\bibitem[\citeproctext]{ref-carranza_what_2024}
Carranza, R., Contreras, D., \& Otero, G. (2024). What makes elites more
or less egalitarian? {Variations} in attitudes towards inequality within
the economic, political and cultural elites in {Chile}.
\emph{Socio-Economic Review}, \emph{22}(3), 1141--1167.
\url{https://doi.org/10.1093/ser/mwae008}

\bibitem[\citeproctext]{ref-castillo_legitimacy_2011}
Castillo, J. C. (2011). Legitimacy of {Inequality} in a {Highly Unequal
Context}: {Evidence} from the {Chilean Case}. \emph{Social Justice
Research}, \emph{24}(4), 314--340.
\url{https://doi.org/10.1007/s11211-011-0144-5}

\bibitem[\citeproctext]{ref-castillo_changes_2025}
Castillo, J. C., Iturra, J., \& Carrasco, K. (2025). Changes in the
{Justification} of {Educational Inequalities}: {The Role} of
{Perceptions} of {Inequality} and {Meritocracy During} the {COVID
Pandemic}. \emph{Social Justice Research}.
\url{https://doi.org/10.1007/s11211-025-00458-0}

\bibitem[\citeproctext]{ref-castillo_multidimensional_2023}
Castillo, J. C., Iturra, J., Maldonado, L., Atria, J., \& Meneses, F.
(2023). A {Multidimensional Approach} for {Measuring Meritocratic
Beliefs}: {Advantages}, {Limitations} and {Alternatives} to the {ISSP
Social Inequality Survey}. \emph{International Journal of Sociology},
1--25. \url{https://doi.org/10.1080/00207659.2023.2274712}

\bibitem[\citeproctext]{ref-castillo_perceptions_2025}
Castillo, J. C., Laffert, A., Carrasco, K., \& Iturra-Sanhueza, J.
(2025). Perceptions of inequality and meritocracy: Their interplay in
shaping preferences for market justice in {Chile} (2016--2023).
\emph{Frontiers in Sociology}, \emph{10}, 1634219.
\url{https://doi.org/10.3389/fsoc.2025.1634219}

\bibitem[\citeproctext]{ref-castillo_deserving_2019}
Castillo, J. C., Olivos, F., \& Azar, A. (2019). Deserving a {Just
Pension}: {A Factorial Survey Approach}*. \emph{Social Science
Quarterly}, \emph{100}(1), 359--378.
\url{https://doi.org/10.1111/ssqu.12539}

\bibitem[\citeproctext]{ref-castillo_socialization_2024}
Castillo, J. C., Salgado, M., Carrasco, K., \& Laffert, A. (2024). The
{Socialization} of {Meritocracy} and {Market Justice Preferences} at
{School}. \emph{Societies}, \emph{14}(11), 214.
\url{https://doi.org/10.3390/soc14110214}

\bibitem[\citeproctext]{ref-castillo_meritocracia_2019}
Castillo, J. C., Torres, A., Atria, J., \& Maldonado, L. (2019).
Meritocracia y desigualdad económica: {Percepciones}, preferencias e
implicancias. \emph{Revista Internacional de Sociología}, \emph{77}(1),
117. \url{https://doi.org/10.3989/ris.2019.77.1.17.114}

\bibitem[\citeproctext]{ref-cavaille_fair_2025}
Cavaillé, C. (2025). \emph{Fair {Enough}?: {Support} for
{Redistribution} in the {Age} of {Inequality}} (1st ed.). Cambridge
University Press. \url{https://doi.org/10.1017/9781009366038}

\bibitem[\citeproctext]{ref-chancel_world_2022}
Chancel, L., Piketty, T., Saez, E., \& Zucman, G. (2022). World
inequality report 2022.
https://bibliotecadigital.ccb.org.co/handle/11520/27510.

\bibitem[\citeproctext]{ref-cinelli_making_2020}
Cinelli, C., \& Hazlett, C. (2020). Making {Sense} of {Sensitivity}:
{Extending Omitted Variable Bias}. \emph{Journal of the Royal
Statistical Society Series B: Statistical Methodology}, \emph{82}(1),
39--67. \url{https://doi.org/10.1111/rssb.12348}

\bibitem[\citeproctext]{ref-darnon_where_2018}
Darnon, C., Wiederkehr, V., Dompnier, B., \& Martinot, D. (2018).
{``{Where} there is a will, there is a way''}: {Belief} in school
meritocracy and the social-class achievement gap. \emph{British Journal
of Social Psychology}, \emph{57}(1), 250--262.
\url{https://doi.org/10.1111/bjso.12214}

\bibitem[\citeproctext]{ref-davidai_why_2018}
Davidai, S. (2018). Why do {Americans} believe in economic mobility?
{Economic} inequality, external attributions of wealth and poverty, and
the belief in economic mobility. \emph{Journal of Experimental Social
Psychology}, \emph{79}, 138--148.
\url{https://doi.org/10.1016/j.jesp.2018.07.012}

\bibitem[\citeproctext]{ref-day_movin_2017}
Day, M. V., \& Fiske, S. T. (2017). Movin' on {Up}? {How Perceptions} of
{Social Mobility Affect Our Willingness} to {Defend} the {System}.
\emph{Social Psychological and Personality Science}, \emph{8}(3),
267--274. \url{https://doi.org/10.1177/1948550616678454}

\bibitem[\citeproctext]{ref-deng_its_2025}
Deng, Y., \& Wang, F. (2025). It's my fault, {I} should try harder!
{The} narratives of self-made upward mobility sustain belief in
meritocracy in low social mobility context. \emph{British Journal of
Psychology}, bjop.70015. \url{https://doi.org/10.1111/bjop.70015}

\bibitem[\citeproctext]{ref-dubet_repensar_2011}
Dubet, F. (2011). \emph{{Repensar la justicia social}} (Sexta Edición).
Siglo XXI.

\bibitem[\citeproctext]{ref-duru-bellat_who_2012}
Duru-Bellat, M., \& Tenret, E. (2012). Who's for {Meritocracy}?
{Individual} and {Contextual Variations} in the {Faith}.
\emph{Comparative Education Review}, \emph{56}(2), 223--247.
\url{https://doi.org/10.1086/661290}

\bibitem[\citeproctext]{ref-edlundDemocraticClassStruggle2015a}
Edlund, J., \& Lindh, A. (2015). The {Democratic Class Struggle
Revisited}: {The Welfare State}, {Social Cohesion} and {Political
Conflict}. \emph{Acta Sociologica}, \emph{58}(4), 311--328.
\url{https://doi.org/10.1177/0001699315610176}

\bibitem[\citeproctext]{ref-espinoza_movilidad_2014}
Espinoza, V., \& Núñez, J. (2014). Movilidad ocupacional en {Chile}
2001-2009. \textquestiondown{{Desigualdad}} de ingresos con igualdad de
oportunidades? \emph{Revista Internacional de Sociología}, \emph{72}(1),
57--82. \url{https://doi.org/10.3989/ris.2011.11.08}

\bibitem[\citeproctext]{ref-eyles_social_2022}
Eyles, A., Major, L. E., \& Machin, S. (2022). \emph{Social {Mobility} -
{Past}, {Present} and {Future}: {The State} of {Play} in {Social
Mobility}, on the 25th {Anniversary} of the {Sutton Trust}}. Sutton
Trust.

\bibitem[\citeproctext]{ref-ferre_welfare_2023}
Ferre, J. C. (2023). Welfare regimes in twenty-first-century {Latin
America}. \emph{Journal of International and Comparative Social Policy},
\emph{39}(2), 101--127. \url{https://doi.org/10.1017/ics.2023.16}

\bibitem[\citeproctext]{ref-flores_top_2020}
Flores, I., Sanhueza, C., Atria, J., \& Mayer, R. (2020). Top {Incomes}
in {Chile}: {A Historical Perspective} on {Income Inequality},
1964--2017. \emph{Review of Income and Wealth}, \emph{66}(4), 850--874.
\url{https://doi.org/10.1111/roiw.12441}

\bibitem[\citeproctext]{ref-fourcade_moral_2007}
Fourcade, M., \& Healy, K. (2007). Moral {Views} of {Market Society}.
\emph{Annual Review of Sociology}, \emph{33}(1), 285--311.
\url{https://doi.org/10.1146/annurev.soc.33.040406.131642}

\bibitem[\citeproctext]{ref-galvez_con_2023}
Gálvez, R. (2023). Con mi plata no\ldots{} alcanza. \emph{CNN Chile}.

\bibitem[\citeproctext]{ref-galvez_pensiones_2024}
Gálvez, R., Kremerman, M., \& Palacios, V. R. (2024). {Pensiones bajo el
mínimo: Los montos de las pensiones que paga el sistema de
capitalización individual en Chile}. \emph{Fundación Sol}.

\bibitem[\citeproctext]{ref-ganzeboom_internationally_1996}
Ganzeboom, H. B. G., \& Treiman, D. J. (1996). Internationally
{Comparable Measures} of {Occupational Status} for the 1988
{International Standard Classification} of {Occupations}. \emph{Social
Science Research}, \emph{25}(3), 201--239.
\url{https://doi.org/10.1006/ssre.1996.0010}

\bibitem[\citeproctext]{ref-garcia-sanchez_perceptions_2018}
García-Sánchez, E., Willis, G. B., Rodríguez-Bailón, R., Palacio Sañudo,
J., David Polo, J., \& Rentería Pérez, E. (2018). Perceptions of
{Economic Inequality} and {Support} for {Redistribution}: {The} role of
{Existential} and {Utopian Standards}. \emph{Social Justice Research},
\emph{31}(4), 335--354. \url{https://doi.org/10.1007/s11211-018-0317-6}

\bibitem[\citeproctext]{ref-gelman_regression_2020}
Gelman, A., Hill, J., \& Vehtari, A. (2020). \emph{Regression and {Other
Stories}} (1st ed.). Cambridge University Press.
\url{https://doi.org/10.1017/9781139161879}

\bibitem[\citeproctext]{ref-gingrich_making_2011}
Gingrich, J. R. (2011). \emph{Making {Markets} in the {Welfare State}:
{The Politics} of {Varying Market Reforms}} (1st ed.). Cambridge
University Press. \url{https://doi.org/10.1017/CBO9780511791529}

\bibitem[\citeproctext]{ref-gugushvili_intergenerational_2016c}
Gugushvili, A. (2016a). Intergenerational objective and subjective
mobility and attitudes towards income differences: Evidence from
transition societies. \emph{Journal of International and Comparative
Social Policy}, \emph{32}(3), 199--219.
\url{https://doi.org/10.1080/21699763.2016.1206482}

\bibitem[\citeproctext]{ref-gugushvili_intergenerational_2016}
Gugushvili, A. (2016b). Intergenerational {Social Mobility} and {Popular
Explanations} of {Poverty}: {A Comparative Perspective}. \emph{Social
Justice Research}, \emph{29}(4), 402--428.
\url{https://doi.org/10.1007/s11211-016-0275-9}

\bibitem[\citeproctext]{ref-gugushvili_subjective_2017}
Gugushvili, A. (2017). Subjective {Intergenerational Mobility} and
{Support} for {Welfare State Programmes}.

\bibitem[\citeproctext]{ref-hainmueller_entropy_2012}
Hainmueller, J. (2012). Entropy {Balancing} for {Causal Effects}: {A
Multivariate Reweighting Method} to {Produce Balanced Samples} in
{Observational Studies}. \emph{Political Analysis}, \emph{20}(1),
25--46. \url{https://doi.org/10.1093/pan/mpr025}

\bibitem[\citeproctext]{ref-hauser_intergenerational_2010}
Hauser, R. M. (2010). Intergenerational {Economic Mobility} in the
{United States} ---{Measures}, {Differentials}, and {Trends}.

\bibitem[\citeproctext]{ref-helgason_longterm_2023}
Helgason, A. F., \& Rehm, P. (2023). Long-term income trajectories and
the evolution of political attitudes. \emph{European Journal of
Political Research}, \emph{62}(1), 264--284.
\url{https://doi.org/10.1111/1475-6765.12506}

\bibitem[\citeproctext]{ref-helgason_class_2025}
Helgason, A. F., \& Rehm, P. (2025). Class experiences and the long-term
evolution of economic values. \emph{Social Forces}, \emph{103}(3),
1125--1143. \url{https://doi.org/10.1093/sf/soae135}

\bibitem[\citeproctext]{ref-hoyt_mindsets_2023}
Hoyt, C. L., Burnette, J. L., Billingsley, J., Becker, W., \& Babij, A.
D. (2023). Mindsets of poverty: {Implications} for redistributive policy
support. \emph{Analyses of Social Issues and Public Policy},
\emph{23}(3), 668--693. \url{https://doi.org/10.1111/asap.12367}

\bibitem[\citeproctext]{ref-huber_political_2000}
Huber, E., \& Stephens, J. D. (2000). \emph{The political economy of
pension reform: {Latin America} in comparative perspective}. Geneva:
UNRISD.

\bibitem[\citeproctext]{ref-hyde_chiles_2015}
Hyde, M., \& Borzutzky, S. (2015). Chile's {``{Neoliberal}''}
{Retirement System}? {Concentration}, {Competition}, and {Economic
Predation} in {``{Private}''} {Pensions}: {Chile}'s {``{Neoliberal}''}
{Retirement System}? \emph{Poverty \& Public Policy}, \emph{7}(2),
123--157. \url{https://doi.org/10.1002/pop4.98}

\bibitem[\citeproctext]{ref-immergut_it_2020}
Immergut, E. M., \& Schneider, S. M. (2020). Is it unfair for the
affluent to be able to purchase {``better''} healthcare? {Existential}
standards and institutional norms in healthcare attitudes across 28
countries. \emph{Social Science \& Medicine}, \emph{267}, 113146.
\url{https://doi.org/10.1016/j.socscimed.2020.113146}

\bibitem[\citeproctext]{ref-jaime-castillo_public_2013}
Jaime-Castillo, A. M. (2013). Public opinion and the reform of the
pension systems in {Europe}: The influence of solidarity principles.
\emph{Journal of European Social Policy}, \emph{23}(4), 390--405.
\url{https://doi.org/10.1177/0958928713507468}

\bibitem[\citeproctext]{ref-jaime-castillo_social_2019}
Jaime-Castillo, A. M., \& Marqués-Perales, I. (2019). Social mobility
and demand for redistribution in {Europe}: A comparative analysis.
\emph{The British Journal of Sociology}, \emph{70}(1), 138--165.
\url{https://doi.org/10.1111/1468-4446.12363}

\bibitem[\citeproctext]{ref-janmaat_subjective_2013}
Janmaat, J. G. (2013). Subjective {Inequality}: A {Review} of
{International Comparative Studies} on {People}'s {Views} about
{Inequality}. \emph{European Journal of Sociology}, \emph{54}(3),
357--389. \url{https://doi.org/10.1017/S0003975613000209}

\bibitem[\citeproctext]{ref-jasso_gender_1999}
Jasso, G., \& Wegener, B. (1999). Gender and {Country Differences} in
the {Sense} of {Justice}: {Justice Evaluation}, {Gender Earnings Gap},
and {Earnings Functions} in {Thirteen Countries}. \emph{International
Journal of Comparative Sociology}, \emph{40}(1), 94--115.
\url{https://doi.org/10.1177/002071529904000106}

\bibitem[\citeproctext]{ref-kerner_pension_2020}
Kerner, A. (2020). Pension {Returns} and {Popular Support} for
{Neoliberalism} in {Post-Pension Reform Latin America}. \emph{British
Journal of Political Science}, \emph{50}(2), 585--620.
\url{https://doi.org/10.1017/S0007123417000710}

\bibitem[\citeproctext]{ref-kluegel_legitimation_1999}
Kluegel, J. R., Mason, D. S., \& Wegener, B. (1999). The {Legitimation}
of {Capitalism} in the {Postcommunist Transition}: {Public Opinion}
about {Market Justice}, 1991-1996. \emph{European Sociological Review},
\emph{15}(3), 251--283. Retrieved from
\url{https://www.jstor.org/stable/522731}

\bibitem[\citeproctext]{ref-kluegel_beliefs_1981}
Kluegel, J. R., \& Smith, E. R. (1981). Beliefs {About Stratification}.
\emph{Annual Review of Sociology}, 29--56.

\bibitem[\citeproctext]{ref-koos_moral_2019}
Koos, S., \& Sachweh, P. (2019). The moral economies of market
societies: Popular attitudes towards market competition, redistribution
and reciprocity in comparative perspective. \emph{Socio-Economic
Review}, \emph{17}(4), 793--821.
\url{https://doi.org/10.1093/ser/mwx045}

\bibitem[\citeproctext]{ref-kulinClassValuesAttitudes2013}
Kulin, J., \& Svallfors, S. (2013). Class, {Values}, and {Attitudes
Towards Redistribution}: {A European Comparison}. \emph{European
Sociological Review}, \emph{29}(2), 155--167.
\url{https://doi.org/10.1093/esr/jcr046}

\bibitem[\citeproctext]{ref-lane_market_1986}
Lane, R. E. (1986). Market {Justice}, {Political Justice}.
\emph{American Political Science Review}, \emph{80}(2), 383--402.
\url{https://doi.org/10.2307/1958264}

\bibitem[\citeproctext]{ref-langsaether_explaining_2022}
Langsæther, P. E., Evans, G., \& O'Grady, T. (2022). Explaining the
{Relationship Between Class Position} and {Political Preferences}: {A
Long-Term Panel Analysis} of {Intra-Generational Class Mobility}.
\emph{British Journal of Political Science}, \emph{52}(2), 958--967.
\url{https://doi.org/10.1017/S0007123420000599}

\bibitem[\citeproctext]{ref-lee_fairness_2023}
Lee, J.-S., \& Stacey, M. (2023). Fairness perceptions of educational
inequality: The effects of self-interest and neoliberal orientations.
\emph{The Australian Educational Researcher}.
\url{https://doi.org/10.1007/s13384-023-00636-6}

\bibitem[\citeproctext]{ref-lindh_public_2015}
Lindh, A. (2015). Public {Opinion} against {Markets}? {Attitudes}
towards {Market Distribution} of {Social Services} -- {A Comparison} of
17 {Countries}. \emph{Social Policy \& Administration}, \emph{49}(7),
887--910. \url{https://doi.org/10.1111/spol.12105}

\bibitem[\citeproctext]{ref-lindh_bringing_2023}
Lindh, A., \& McCall, L. (2023). Bringing the market in: An expanded
framework for understanding popular responses to economic inequality.
\emph{Socio-Economic Review}, \emph{21}(2), 1035--1055.
\url{https://doi.org/10.1093/ser/mwac018}

\bibitem[\citeproctext]{ref-lindner_does_2024}
Lindner, T., Mijs, J. J. B., De Koster, W., \& Van Der Waal, J. (2024).
Does informing citizens about the non-meritocratic nature of inequality
bolster support for a universal basic income? {Evidence} from a
population-based survey experiment. \emph{European Societies},
\emph{26}(3), 575--601.
\url{https://doi.org/10.1080/14616696.2023.2272263}

\bibitem[\citeproctext]{ref-lopez-roldan_comparative_2021}
López-Roldán, P., \& Fachelli, S. (Eds.). (2021). \emph{Towards a
{Comparative Analysis} of {Social Inequalities} between {Europe} and
{Latin America}}. Cham: Springer International Publishing.
\url{https://doi.org/10.1007/978-3-030-48442-2}

\bibitem[\citeproctext]{ref-mac-clure_justicia_2024}
Mac-Clure, O., Barozet, E., \& Franetovic, G. (2024). {Justicia
distributiva y posición social subjetiva: \textquestiondown la
meritocracia justifica la desigualdad de ingresos?} \emph{Convergencia
Revista de Ciencias Sociales}, \emph{31}, 1.
\url{https://doi.org/10.29101/crcs.v31i0.22258}

\bibitem[\citeproctext]{ref-madariaga_three_2020}
Madariaga, A. (2020). The three pillars of neoliberalism: {Chile}'s
economic policy trajectory in comparative perspective.
\emph{Contemporary Politics}, \emph{26}(3), 308--329.
\url{https://doi.org/10.1080/13569775.2020.1735021}

\bibitem[\citeproctext]{ref-madero-cabib_private_2019}
Madero-Cabib, I., Biehl, A., Sehnbruch, K., Calvo, E., \& Bertranou, F.
(2019). Private {Pension Systems Built} on {Precarious Foundations}: {A
Cohort Study} of {Labor-Force Trajectories} in {Chile}. \emph{Research
on Aging}, \emph{41}(10), 961--987.
\url{https://doi.org/10.1177/0164027519874687}

\bibitem[\citeproctext]{ref-matamoros-lima_social_2025}
Matamoros-Lima, J., Galdi, S., Moya, M., \& Willis, G. B. (2025). Social
mobility beliefs and attitudes toward redistribution: {Potential}
explanatory mechanisms. \emph{Political Psychology}, \emph{46}(4),
884--902. \url{https://doi.org/10.1111/pops.13042}

\bibitem[\citeproctext]{ref-mau_inequality_2015}
Mau, S. (2015). \emph{Inequality, {Marketization} and the {Majority
Class}: {Why Did} the {European Middle Classes Accept Neo-Liberalism}?}
Houndmills: Palgrave Macmillan.

\bibitem[\citeproctext]{ref-mijs_unfulfillable_2016}
Mijs, J. (2016). The {Unfulfillable Promise} of {Meritocracy}: {Three
Lessons} and {Their Implications} for {Justice} in {Education}.
\emph{Social Justice Research}, \emph{29}(1), 14--34.
\url{https://doi.org/10.1007/s11211-014-0228-0}

\bibitem[\citeproctext]{ref-mijs_paradox_2019}
Mijs, J. (2019). The paradox of inequality: Income inequality and belief
in meritocracy go hand in hand. \emph{Socio-Economic Review},
\emph{19}(1), 7--35. \url{https://doi.org/10.1093/ser/mwy051}

\bibitem[\citeproctext]{ref-mijs_belief_2022}
Mijs, J., Daenekindt, S., de Koster, W., \& van der Waal, J. (2022).
Belief in {Meritocracy Reexamined}: {Scrutinizing} the {Role} of
{Subjective Social Mobility}. \emph{Social Psychology Quarterly},
\emph{85}(2), 131--141. \url{https://doi.org/10.1177/01902725211063818}

\bibitem[\citeproctext]{ref-mijs_how_2022}
Mijs, J., \& Hoy, C. (2022). How {Information} about {Inequality Impacts
Belief} in {Meritocracy}: {Evidence} from a {Randomized Survey
Experiment} in {Australia}, {Indonesia} and {Mexico}. \emph{Social
Problems}, \emph{69}(1), 91--122.
\url{https://doi.org/10.1093/socpro/spaa059}

\bibitem[\citeproctext]{ref-miller_selfserving_1975}
Miller, D. T., \& Ross, M. (1975). Self-serving biases in the
attribution of causality: {Fact} or fiction? \emph{Psychological
Bulletin}, \emph{82}(2), 213--225.
\url{https://doi.org/10.1037/h0076486}

\bibitem[\citeproctext]{ref-molina_its_2019}
Molina, M. D., Bucca, M., \& Macy, M. W. (2019). It's not just how the
game is played, it's whether you win or lose. \emph{SCIENCE ADVANCES}.

\bibitem[\citeproctext]{ref-newman_economic_2023}
Newman, B. J. (2023). Economic {Inequality}, the {Working Poor}, and
{Belief} in the {American Dream}. \emph{Public Opinion Quarterly},
\emph{86}(4), 944--954. \url{https://doi.org/10.1093/poq/nfac043}

\bibitem[\citeproctext]{ref-nudesoc_informe_2020}
NUDESOC, N. de S. C. (2020). \emph{Informe de resultados oficial
{Encuesta Zona Cero}}. Santiago de Chile.

\bibitem[\citeproctext]{ref-oecd_pensions_2023}
OECD. (2023). \emph{Pensions at a {Glance} 2023: {OECD} and {G20
Indicators}}. OECD Publishing. \url{https://doi.org/10.1787/678055dd-en}

\bibitem[\citeproctext]{ref-otero_power_2024}
Otero, G., \& Mendoza, M. (2024). The {Power} of {Diversity}: {Class},
{Networks} and {Attitudes Towards Inequality}. \emph{Sociology},
\emph{58}(4), 851--876. \url{https://doi.org/10.1177/00380385231217625}

\bibitem[\citeproctext]{ref-panes_criticas_2020}
Panes, D. (2020). \emph{Críticas y experiencias obreras en torno al
sistema de {AFP}'s en {Chile} (1981-2020)} (PhD thesis). Universidad de
Chile, Santiago.

\bibitem[\citeproctext]{ref-pnud_desiguales_2017}
PNUD (Ed.). (2017). \emph{{Desiguales: orígenes, cambios y desafíos de
la brecha social en Chile}}. Santiago, Chile: PNUD : Uqbar Editores.

\bibitem[\citeproctext]{ref-quadagno_has_2012}
Quadagno, J., \& Pederson, J. (2012). Has support for {Social Security}
declined? {Attitudes} toward the public pension scheme in the {USA},
2000 and 2010. \emph{International Journal of Social Welfare},
\emph{21}(s1). \url{https://doi.org/10.1111/j.1468-2397.2012.00877.x}

\bibitem[\citeproctext]{ref-rodriguez_intergenerational_2024}
Rodriguez, I., \& Matilla-Garcia, M. (2024). Intergenerational
{Mobility} and {Preferences} for {Redistribution}: {Empirical Evidence}
from {Spain}. SSRN. \url{https://doi.org/10.2139/ssrn.5045536}

\bibitem[\citeproctext]{ref-ruiz_formacion_2020}
Ruiz, F. (2020). {Formación de clases y conflicto social en el sistema
previsional chileno: mecanismos de acumulación capitalista y desafíos
para la democracia}. \emph{REVISTA ENCUENTROS}, \emph{18}(3), 88--99.
\url{https://doi.org/10.15665/encuent.v18i3.2134}

\bibitem[\citeproctext]{ref-salgado_uplifting_2025}
Salgado, M., Díaz, M., Gamarra, C., \& Núñez, J. (2025). Uplifting
mechanism or equalizing force? {The} expansion of higher education and
its role in intergenerational mobility in {Chile}. \emph{Higher
Education}. \url{https://doi.org/10.1007/s10734-025-01560-7}

\bibitem[\citeproctext]{ref-sandel_tyranny_2020}
Sandel, M. J. (2020). \emph{The tyranny of merit: {What}'s become of the
common good?} (First edition). New York: {Farrar, Straus and Giroux}.

\bibitem[\citeproctext]{ref-satz_por_2019}
Satz, D. (2019). \emph{{Por Qué Algunas Cosas No Deberían Estar en
Venta: Los límites Morales Del Mercado}}. Ciudad Autónoma de Buenos
Aires: Siglo XXI Editores.

\bibitem[\citeproctext]{ref-schmidt_experience_2011}
Schmidt, A. W. (2011). The experience of social mobility and the
formation of attitudes toward redistribution. In.

\bibitem[\citeproctext]{ref-sen_merit_2000}
Sen, A. (2000). Merit and {Justice}. In K. Arrow, S. Bowles, \& S. N.
Durlauf (Eds.), \emph{Meritocracy and {Economic Inequality}} (pp.
5--16). Princeton University Press.
\url{https://doi.org/10.1515/9780691190334-003}

\bibitem[\citeproctext]{ref-shariff_income_2016}
Shariff, A. F., Wiwad, D., \& Aknin, L. B. (2016). Income {Mobility
Breeds Tolerance} for {Income Inequality}: {Cross-National} and
{Experimental Evidence}. \emph{Perspectives on Psychological Science},
\emph{11}(3), 373--380. \url{https://doi.org/10.1177/1745691616635596}

\bibitem[\citeproctext]{ref-solimano_rise_2021}
Solimano, A. (2021). \emph{The {Rise} and {Fall} of the {Privatized
Pension System} in {Chile}: {An International Perspective}} (1st ed).
London: Anthem Press.

\bibitem[\citeproctext]{ref-somma_no_2021}
Somma, N. M., Bargsted, M., Disi Pavlic, R., \& Medel, R. M. (2021). No
water in the oasis: The {Chilean Spring} of 2019--2020. \emph{Social
Movement Studies}, \emph{20}(4), 495--502.
\url{https://doi.org/10.1080/14742837.2020.1727737}

\bibitem[\citeproctext]{ref-song_there_2025}
Song, X., \& Zhou, X. (2025). Is {There} a {Mobility Effect}? {On
Methodological Issues} in the {Mobility Contrast Model}.
\emph{Sociological Methods \& Research}, \emph{54}(4), 1576--1593.
\url{https://doi.org/10.1177/00491241251347983}

\bibitem[\citeproctext]{ref-streeck_how_2016}
Streeck, W. (2016). \emph{How will capitalism end? Essays on a failing
system}. London: Verso.

\bibitem[\citeproctext]{ref-svallforsMoralEconomyClass2006a}
Svallfors, S. (2006). \emph{The {Moral Economy} of {Class}: {Class} and
{Attitudes} in {Comparative Perspective}}. Stanford University Press.

\bibitem[\citeproctext]{ref-svallfors_political_2007}
Svallfors, S. (Ed.). (2007). \emph{The {Political Sociology} of the
{Welfare State}: {Institutions}, {Social Cleavages}, and {Orientations}}
(1st ed.). Stanford University Press.
\url{https://doi.org/10.2307/j.ctvr0qv0q}

\bibitem[\citeproctext]{ref-tejero-peregrina_perceived_2025}
Tejero-Peregrina, L., Willis, G., Sánchez-Rodríguez, Á., \&
Rodríguez-Bailón, R. (2025). From {Perceived Economic Inequality} to
{Support} for {Redistribution}: {The Role} of {Meritocracy Perception}.
\emph{International Review of Social Psychology}, \emph{38}(1), 4.
\url{https://doi.org/10.5334/irsp.1013}

\bibitem[\citeproctext]{ref-torche_unequal_2005}
Torche, F. (2005). Unequal {But Fluid}: {Social Mobility} in {Chile} in
{Comparative Perspective}. \emph{American Sociological Review},
\emph{70}(3), 422--450. \url{https://doi.org/10.1177/000312240507000304}

\bibitem[\citeproctext]{ref-torche_intergenerational_2014}
Torche, F. (2014). Intergenerational {Mobility} and {Inequality}: {The
Latin American Case}. \emph{Annual Review of Sociology}, \emph{40}(1),
619--642. \url{https://doi.org/10.1146/annurev-soc-071811-145521}

\bibitem[\citeproctext]{ref-vandewerfhorst_meritocracy_2024}
Van De Werfhorst, H. G. (2024). Is {Meritocracy Not So Bad After All}?
{Educational Expansion} and {Intergenerational Mobility} in 40
{Countries}. \emph{American Sociological Review}, \emph{89}(6),
1181--1213. \url{https://doi.org/10.1177/00031224241292352}

\bibitem[\citeproctext]{ref-verbic_political_2019}
Verbič, M., \& Spruk, R. (2019). Political economy of pension reforms:
An empirical investigation. \emph{European Journal of Law and
Economics}, \emph{47}(2), 171--232.
\url{https://doi.org/10.1007/s10657-018-9606-7}

\bibitem[\citeproctext]{ref-vondemknesebeck_are_2016}
Von Dem Knesebeck, O., Vonneilich, N., \& Kim, T. J. (2016). Are health
care inequalities unfair? {A} study on public attitudes in 23 countries.
\emph{International Journal for Equity in Health}, \emph{15}(1), 61.
\url{https://doi.org/10.1186/s12939-016-0350-8}

\bibitem[\citeproctext]{ref-wiederkehr_belief_2015}
Wiederkehr, V., Bonnot, V., Krauth-Gruber, S., \& Darnon, C. (2015).
Belief in school meritocracy as a system-justifying tool for low status
students. \emph{Frontiers in Psychology}, \emph{6}.

\bibitem[\citeproctext]{ref-wilson_role_2003}
Wilson, C. (2003). The {Role} of a {Merit Principle} in {Distributive
Justice}. \emph{The Journal of Ethics}, \emph{7}(3), 277--314.
\url{https://doi.org/10.1023/A:1024667228488}

\bibitem[\citeproctext]{ref-wooldridge_introductory_2009}
Wooldridge, J. M. (2009). \emph{{Introductory econometrics: a modern
approach}} (4th ed). Mason, OH: South Western, Cengage Learning.

\bibitem[\citeproctext]{ref-wrightClassCountsComparative1997a}
Wright, E. O. (1997). \emph{Class {Counts}: {Comparative Studies} in
{Class Analysis}}. Cambridge University Press.

\bibitem[\citeproctext]{ref-wrightUnderstandingClass2015}
Wright, E. O. (2015). \emph{Understanding {Class}} (Verso).

\bibitem[\citeproctext]{ref-young_rise_1958}
Young, M. (1958). \emph{The rise of the meritocracy}. New Brunswick,
N.J., U.S.A: Transaction Publishers.

\bibitem[\citeproctext]{ref-zhu_meritocratic_2025}
Zhu, Li. (2025). Meritocratic beliefs in the {United States}, {Finland},
and {China}: {A} multidimensional approach using latent class analysis.
\emph{The British Journal of Sociology}, \emph{76}(1), 153--172.
\url{https://doi.org/10.1111/1468-4446.13152}

\bibitem[\citeproctext]{ref-zhu_policy_2015}
Zhu, Ling, \& Lipsmeyer, C. S. (2015). Policy feedback and economic
risk: The influence of privatization on social policy preferences.
\emph{Journal of European Public Policy}, \emph{22}(10), 1489--1511.
\url{https://doi.org/10.1080/13501763.2015.1031159}

\end{CSLReferences}

\section{Supplementary material}\label{anexo}

This section presents the supplementary material for this study.

\subsection{Descriptive statistics}\label{descriptive-statistics}

\begin{longtable}[]{@{}
  >{\raggedright\arraybackslash}p{(\linewidth - 6\tabcolsep) * \real{0.4216}}
  >{\raggedright\arraybackslash}p{(\linewidth - 6\tabcolsep) * \real{0.2647}}
  >{\raggedright\arraybackslash}p{(\linewidth - 6\tabcolsep) * \real{0.2059}}
  >{\raggedright\arraybackslash}p{(\linewidth - 6\tabcolsep) * \real{0.1078}}@{}}
\caption{Descriptive statistics for all
variables.}\label{tbl-summary1}\tabularnewline
\toprule\noalign{}
\begin{minipage}[b]{\linewidth}\raggedright
Label
\end{minipage} & \begin{minipage}[b]{\linewidth}\raggedright
Stats / Values
\end{minipage} & \begin{minipage}[b]{\linewidth}\raggedright
Freqs (\% of Valid)
\end{minipage} & \begin{minipage}[b]{\linewidth}\raggedright
Valid
\end{minipage} \\
\midrule\noalign{}
\endfirsthead
\toprule\noalign{}
\begin{minipage}[b]{\linewidth}\raggedright
Label
\end{minipage} & \begin{minipage}[b]{\linewidth}\raggedright
Stats / Values
\end{minipage} & \begin{minipage}[b]{\linewidth}\raggedright
Freqs (\% of Valid)
\end{minipage} & \begin{minipage}[b]{\linewidth}\raggedright
Valid
\end{minipage} \\
\midrule\noalign{}
\endhead
\bottomrule\noalign{}
\endlastfoot
Preference for pension commodification &
\begin{minipage}[t]{\linewidth}\raggedright
1. Disagree\\
2. Agree\strut
\end{minipage} & \begin{minipage}[t]{\linewidth}\raggedright
2702 (78.7\%)\\
733 (21.3\%)\strut
\end{minipage} & \begin{minipage}[t]{\linewidth}\raggedright
3435\\
(100.0\%)\strut
\end{minipage} \\
Father stratum & \begin{minipage}[t]{\linewidth}\raggedright
1. Low\\
2. Middle\\
3. High\strut
\end{minipage} & \begin{minipage}[t]{\linewidth}\raggedright
1122 (32.7\%)\\
1104 (32.1\%)\\
1209 (35.2\%)\strut
\end{minipage} & \begin{minipage}[t]{\linewidth}\raggedright
3435\\
(100.0\%)\strut
\end{minipage} \\
Offspring stratum & \begin{minipage}[t]{\linewidth}\raggedright
1. Low\\
2. Middle\\
3. High\strut
\end{minipage} & \begin{minipage}[t]{\linewidth}\raggedright
1186 (34.5\%)\\
1110 (32.3\%)\\
1139 (33.2\%)\strut
\end{minipage} & \begin{minipage}[t]{\linewidth}\raggedright
3435\\
(100.0\%)\strut
\end{minipage} \\
Parental education & \begin{minipage}[t]{\linewidth}\raggedright
Mean (sd) : 4 (2.2)\\
min \textless{} med \textless{} max:\\
1 \textless{} 4 \textless{} 10\\
IQR (CV) : 3 (0.6)\strut
\end{minipage} & \begin{minipage}[t]{\linewidth}\raggedright
1 : 222 ( 6.5\%)\\
2 : 905 (26.3\%)\\
3 : 576 (16.8\%)\\
4 : 321 ( 9.3\%)\\
5 : 826 (24.0\%)\\
6 : 30 ( 0.9\%)\\
7 : 214 ( 6.2\%)\\
8 : 75 ( 2.2\%)\\
9 : 245 ( 7.1\%)\\
10 : 21 ( 0.6\%)\strut
\end{minipage} & \begin{minipage}[t]{\linewidth}\raggedright
3435\\
(100.0\%)\strut
\end{minipage} \\
Co-residence with both parents at age 15 &
\begin{minipage}[t]{\linewidth}\raggedright
1. No co-residence\\
2. Co-residence\strut
\end{minipage} & \begin{minipage}[t]{\linewidth}\raggedright
996 (29.0\%)\\
2439 (71.0\%)\strut
\end{minipage} & \begin{minipage}[t]{\linewidth}\raggedright
3435\\
(100.0\%)\strut
\end{minipage} \\
Nacionality & \begin{minipage}[t]{\linewidth}\raggedright
1. Non-Chilean\\
2. Chilean\strut
\end{minipage} & \begin{minipage}[t]{\linewidth}\raggedright
62 ( 1.8\%)\\
3373 (98.2\%)\strut
\end{minipage} & \begin{minipage}[t]{\linewidth}\raggedright
3435\\
(100.0\%)\strut
\end{minipage} \\
Sex & \begin{minipage}[t]{\linewidth}\raggedright
1. Male\\
2. Female\strut
\end{minipage} & \begin{minipage}[t]{\linewidth}\raggedright
1507 (43.9\%)\\
1928 (56.1\%)\strut
\end{minipage} & \begin{minipage}[t]{\linewidth}\raggedright
3435\\
(100.0\%)\strut
\end{minipage} \\
Age (in years) & \begin{minipage}[t]{\linewidth}\raggedright
Mean (sd) : 42.6 (12.5)\\
min \textless{} med \textless{} max:\\
18 \textless{} 43 \textless{} 75\\
IQR (CV) : 21 (0.3)\strut
\end{minipage} & 58 distinct values &
\begin{minipage}[t]{\linewidth}\raggedright
3435\\
(100.0\%)\strut
\end{minipage} \\
Indigenous ethnicity & \begin{minipage}[t]{\linewidth}\raggedright
1. Non-indigenous\\
2. Indigenous\strut
\end{minipage} & \begin{minipage}[t]{\linewidth}\raggedright
3035 (88.4\%)\\
400 (11.6\%)\strut
\end{minipage} & \begin{minipage}[t]{\linewidth}\raggedright
3435\\
(100.0\%)\strut
\end{minipage} \\
Meritocracy perception & \begin{minipage}[t]{\linewidth}\raggedright
1. Low\\
2. High\strut
\end{minipage} & \begin{minipage}[t]{\linewidth}\raggedright
2741 (79.8\%)\\
694 (20.2\%)\strut
\end{minipage} & \begin{minipage}[t]{\linewidth}\raggedright
3435\\
(100.0\%)\strut
\end{minipage} \\
Wave & \begin{minipage}[t]{\linewidth}\raggedright
1. 2016\\
2. 2018\\
3. 2023\strut
\end{minipage} & \begin{minipage}[t]{\linewidth}\raggedright
914 (26.6\%)\\
1377 (40.1\%)\\
1144 (33.3\%)\strut
\end{minipage} & \begin{minipage}[t]{\linewidth}\raggedright
3435\\
(100.0\%)\strut
\end{minipage} \\
\end{longtable}

\subsection{Balance evaluation}\label{balance-evaluation}

\begin{figure}[H]

\centering{

\includegraphics[width=0.9\linewidth,height=\textheight,keepaspectratio]{paper_files/figure-pdf/fig-balance-1.pdf}

}

\caption{\label{fig-balance}Balance SMD --- Mobility treatment (ATT)}

\end{figure}%

\subsection{Mobility effects models}\label{mobility-effects-models-1}

\begin{table}

\caption{\label{tbl-complete1}Effects of intergenerational occupational
mobility on preferences for pension commodification, with covariates and
wave fixed effects.}

\centering{

\begin{center}
\scalebox{0.7}{
\begin{tabular}{l c c c c c c}
\hline
 & Low-Middle & Low-High & Middle-High & Middle-Low & High-Middle & High-Low \\
\hline
Intercept                                               & $0.18$            & $-0.02$           & $0.30^{*}$        & $0.23$            & $0.13$            & $0.14$            \\
                                                        & $ [-0.36;  0.71]$ & $ [-0.35;  0.31]$ & $ [ 0.01;  0.58]$ & $ [-0.08;  0.54]$ & $ [-0.19;  0.46]$ & $ [-0.28;  0.56]$ \\
Mobility treatment                                      & $-0.02$           & $-0.02$           & $0.09^{*}$        & $0.00$            & $-0.07^{*}$       & $-0.10^{*}$       \\
                                                        & $ [-0.08;  0.05]$ & $ [-0.09;  0.05]$ & $ [ 0.02;  0.16]$ & $ [-0.06;  0.06]$ & $ [-0.15; -0.00]$ & $ [-0.18; -0.02]$ \\
Age (in years)                                          & $0.00$            & $0.00$            & $-0.00$           & $-0.00$           & $0.00$            & $0.00$            \\
                                                        & $ [-0.00;  0.00]$ & $ [-0.00;  0.00]$ & $ [-0.00;  0.00]$ & $ [-0.00;  0.00]$ & $ [-0.00;  0.01]$ & $ [-0.00;  0.01]$ \\
Female (Ref. = Male)                                    & $-0.14^{*}$       & $-0.14^{*}$       & $-0.13^{*}$       & $-0.07^{*}$       & $-0.11^{*}$       & $-0.12^{*}$       \\
                                                        & $ [-0.20; -0.08]$ & $ [-0.22; -0.06]$ & $ [-0.20; -0.06]$ & $ [-0.14; -0.01]$ & $ [-0.19; -0.03]$ & $ [-0.21; -0.03]$ \\
Chilean nacionality (Ref. = Non-Chilean)                & $-0.03$           & $0.20$            & $0.08$            & $0.04$            & $-0.00$           & $0.06$            \\
                                                        & $ [-0.63;  0.57]$ & $ [-0.05;  0.46]$ & $ [-0.19;  0.36]$ & $ [-0.26;  0.34]$ & $ [-0.27;  0.27]$ & $ [-0.30;  0.42]$ \\
Indigenous ethnicity (Ref. = Non-indigenous)            & $0.03$            & $-0.05$           & $0.05$            & $-0.01$           & $0.07$            & $0.01$            \\
                                                        & $ [-0.06;  0.12]$ & $ [-0.15;  0.05]$ & $ [-0.08;  0.18]$ & $ [-0.10;  0.08]$ & $ [-0.08;  0.22]$ & $ [-0.13;  0.14]$ \\
Co-residence with both parents (Ref. = No co-residence) & $-0.00$           & $0.04$            & $-0.06$           & $-0.01$           & $0.10^{*}$        & $0.10^{*}$        \\
                                                        & $ [-0.08;  0.07]$ & $ [-0.04;  0.12]$ & $ [-0.14;  0.01]$ & $ [-0.09;  0.06]$ & $ [ 0.01;  0.19]$ & $ [ 0.01;  0.19]$ \\
Parental education                                      & $0.02$            & $0.01$            & $-0.02^{*}$       & $-0.01$           & $-0.00$           & $-0.01$           \\
                                                        & $ [-0.00;  0.05]$ & $ [-0.01;  0.04]$ & $ [-0.03; -0.01]$ & $ [-0.02;  0.01]$ & $ [-0.02;  0.02]$ & $ [-0.03;  0.02]$ \\
Wave (Ref.= 2016)                                       &                   &                   &                   &                   &                   &                   \\
                                                        &                   &                   &                   &                   &                   &                   \\
\quad Wave 2018                                         & $0.02$            & $-0.01$           & $0.08^{*}$        & $0.04$            & $-0.00$           & $-0.00$           \\
                                                        & $ [-0.05;  0.09]$ & $ [-0.09;  0.07]$ & $ [ 0.02;  0.14]$ & $ [-0.02;  0.10]$ & $ [-0.08;  0.08]$ & $ [-0.09;  0.09]$ \\
\quad Wave 2023                                         & $0.06$            & $0.04$            & $0.15^{*}$        & $0.14^{*}$        & $0.12^{*}$        & $0.10$            \\
                                                        & $ [-0.02;  0.13]$ & $ [-0.06;  0.13]$ & $ [ 0.08;  0.22]$ & $ [ 0.07;  0.21]$ & $ [ 0.02;  0.21]$ & $ [-0.00;  0.21]$ \\
\hline
R$^2$                                                   & $0.04$            & $0.04$            & $0.07$            & $0.03$            & $0.05$            & $0.06$            \\
Adj. R$^2$                                              & $0.03$            & $0.03$            & $0.06$            & $0.02$            & $0.04$            & $0.05$            \\
Num. obs.                                               & $878$             & $745$             & $716$             & $773$             & $912$             & $861$             \\
RMSE                                                    & $0.40$            & $0.37$            & $0.39$            & $0.39$            & $0.44$            & $0.42$            \\
N Clusters                                              & $497$             & $449$             & $443$             & $470$             & $536$             & $525$             \\
\hline
\multicolumn{7}{l}{\scriptsize{Note: Cells contain regression coefficients with confidence intervals in parentheses. $^*$ Null hypothesis value outside the confidence interval..}}
\end{tabular}
}
\label{table:coefficients}
\end{center}

}

\end{table}%

\begin{table}

\caption{\label{tbl-complete2}Effects of intergenerational occupational
mobility on preferences for pension commodification.}

\centering{

\begin{center}
\scalebox{0.8}{
\begin{tabular}{l c c c c c c}
\hline
 & Low-Middle & Low-High & Middle-High & Middle-Low & High-Middle & High-Low \\
\hline
Intercept          & $0.21^{*}$       & $0.18^{*}$       & $0.17^{*}$      & $0.19^{*}$       & $0.32^{*}$       & $0.28^{*}$        \\
                   & $ [ 0.16; 0.26]$ & $ [ 0.13; 0.23]$ & $ [0.12; 0.21]$ & $ [ 0.15; 0.24]$ & $ [ 0.26; 0.38]$ & $ [ 0.21;  0.35]$ \\
Mobility treatment & $-0.01$          & $-0.02$          & $0.08^{*}$      & $-0.00$          & $-0.07$          & $-0.09^{*}$       \\
                   & $ [-0.08; 0.05]$ & $ [-0.08; 0.05]$ & $ [0.01; 0.15]$ & $ [-0.06; 0.06]$ & $ [-0.15; 0.00]$ & $ [-0.17; -0.02]$ \\
\hline
R$^2$              & $0.00$           & $0.00$           & $0.01$          & $0.00$           & $0.01$           & $0.01$            \\
Adj. R$^2$         & $-0.00$          & $-0.00$          & $0.01$          & $-0.00$          & $0.01$           & $0.01$            \\
Num. obs.          & $878$            & $745$            & $716$           & $773$            & $912$            & $861$             \\
RMSE               & $0.40$           & $0.38$           & $0.40$          & $0.39$           & $0.45$           & $0.43$            \\
N Clusters         & $497$            & $449$            & $443$           & $470$            & $536$            & $525$             \\
\hline
\multicolumn{7}{l}{\scriptsize{Note: Cells contain regression coefficients with confidence intervals in parentheses. $^*$ Null hypothesis value outside the confidence interval..}}
\end{tabular}
}
\label{table:coefficients}
\end{center}

}

\end{table}%

\begin{table}

\caption{\label{tbl-complete3}Interactions effects between
intergenerational occupational mobility and perceived meritocracy on
preferences for pension commodification, with covariates and wave fixed
effects.}

\centering{

\begin{center}
\scalebox{0.7}{
\begin{tabular}{l c c c c c c}
\hline
 & Low-Middle & Low-High & Middle-High & Middle-Low & High-Middle & High-Low \\
\hline
Intercept                                                    & $0.15$           & $-0.04$          & $0.30^{*}$       & $0.21$           & $0.08$           & $0.05$           \\
                                                             & $ [-0.36; 0.67]$ & $ [-0.40; 0.31]$ & $ [ 0.04; 0.56]$ & $ [-0.09; 0.51]$ & $ [-0.25; 0.40]$ & $ [-0.30; 0.41]$ \\
Mobility treatment                                           & $-0.02$          & $-0.03$          & $0.07$           & $-0.00$          & $-0.07$          & $-0.08$          \\
                                                             & $ [-0.09; 0.04]$ & $ [-0.10; 0.05]$ & $ [-0.01; 0.14]$ & $ [-0.07; 0.06]$ & $ [-0.15; 0.00]$ & $ [-0.15; 0.00]$ \\
High meritocracy perception (Ref.= Low)                      & $0.05$           & $0.06$           & $-0.01$          & $0.03$           & $0.17^{*}$       & $0.26^{*}$       \\
                                                             & $ [-0.06; 0.16]$ & $ [-0.07; 0.19]$ & $ [-0.09; 0.08]$ & $ [-0.07; 0.13]$ & $ [ 0.00; 0.33]$ & $ [ 0.04; 0.48]$ \\
Mobility treatment x High meritocracy perception (Ref.= Low) & $0.04$           & $0.05$           & $0.15$           & $0.04$           & $-0.03$          & $-0.14$          \\
                                                             & $ [-0.12; 0.20]$ & $ [-0.14; 0.24]$ & $ [-0.03; 0.33]$ & $ [-0.11; 0.19]$ & $ [-0.23; 0.17]$ & $ [-0.38; 0.10]$ \\
\hline
Controls                                                     & Yes              & Yes              & Yes              & Yes              & Yes              & Yes              \\
R$^2$                                                        & $0.04$           & $0.05$           & $0.08$           & $0.04$           & $0.07$           & $0.10$           \\
Adj. R$^2$                                                   & $0.03$           & $0.04$           & $0.07$           & $0.02$           & $0.06$           & $0.08$           \\
Num. obs.                                                    & $878$            & $745$            & $716$            & $773$            & $912$            & $861$            \\
RMSE                                                         & $0.40$           & $0.37$           & $0.39$           & $0.39$           & $0.44$           & $0.41$           \\
N Clusters                                                   & $497$            & $449$            & $443$            & $470$            & $536$            & $525$            \\
\hline
\multicolumn{7}{l}{\scriptsize{Note: Cells contain regression coefficients with confidence intervals in parentheses. $^*$ Null hypothesis value outside the confidence interval..}}
\end{tabular}
}
\label{table:coefficients}
\end{center}

}

\end{table}%

\begin{table}

\caption{\label{tbl-complete4}Marginal effects of mobility by
meritocratic perception (Low vs High Meritocracy).}

\centering{

\centering
\begin{tabular}{l|c|c|c|c|c}
\hline
Trajectory & Merit & Estimate & Std. Error & CI Low & CI High\\
\hline
Low-Middle & Low Meritocracy & -0.024 & 0.034 & -0.091 & 0.042\\
\hline
Low-Middle & High Meritocracy & 0.015 & 0.072 & -0.127 & 0.157\\
\hline
Low-High & Low Meritocracy & -0.029 & 0.038 & -0.104 & 0.046\\
\hline
Low-High & High Meritocracy & 0.024 & 0.088 & -0.148 & 0.195\\
\hline
Middle-High & Low Meritocracy & 0.066 & 0.038 & -0.009 & 0.140\\
\hline
Middle-High & High Meritocracy & 0.215 & 0.084 & 0.051 & 0.379\\
\hline
Middle-Low & Low Meritocracy & -0.004 & 0.034 & -0.071 & 0.063\\
\hline
Middle-Low & High Meritocracy & 0.038 & 0.068 & -0.095 & 0.171\\
\hline
High-Middle & Low Meritocracy & -0.074 & 0.039 & -0.150 & 0.003\\
\hline
High-Middle & High Meritocracy & -0.100 & 0.094 & -0.285 & 0.085\\
\hline
High-Low & Low Meritocracy & -0.077 & 0.039 & -0.153 & 0.000\\
\hline
High-Low & High Meritocracy & -0.216 & 0.112 & -0.436 & 0.005\\
\hline
\end{tabular}

}

\end{table}%

\newpage{}

\subsection{Robustness check and sensitivity
analysis}\label{robustness-check-and-sensitivity-analysis-1}

\begin{table}

\caption{\label{tbl-robus1}Effects of intergenerational occupational
mobility on preferences for pension commodification coded as 3-5
vs.~1-2, with covariates and wave fixed effects.}

\centering{

\begin{center}
\scalebox{0.7}{
\begin{tabular}{l c c c c c c}
\hline
 & Low-Middle & Low-High & Middle-High & Middle-Low & High-Middle & High-Low \\
\hline
Intercept                                               & $0.16$            & $0.01$            & $0.59^{*}$        & $0.57^{*}$        & $0.48^{*}$        & $0.29$            \\
                                                        & $ [-0.38;  0.70]$ & $ [-0.37;  0.40]$ & $ [ 0.17;  1.01]$ & $ [ 0.23;  0.92]$ & $ [ 0.04;  0.92]$ & $ [-0.32;  0.90]$ \\
Mobility treatment                                      & $0.01$            & $0.05$            & $0.09^{*}$        & $-0.02$           & $-0.08$           & $-0.13^{*}$       \\
                                                        & $ [-0.07;  0.08]$ & $ [-0.02;  0.13]$ & $ [ 0.01;  0.16]$ & $ [-0.09;  0.05]$ & $ [-0.15;  0.00]$ & $ [-0.21; -0.04]$ \\
Age (in years)                                          & $0.00$            & $-0.00$           & $-0.00^{*}$       & $-0.00$           & $0.00$            & $0.00$            \\
                                                        & $ [-0.00;  0.00]$ & $ [-0.01;  0.00]$ & $ [-0.01; -0.00]$ & $ [-0.00;  0.00]$ & $ [-0.00;  0.00]$ & $ [-0.00;  0.01]$ \\
Female (Ref. = Male)                                    & $-0.15^{*}$       & $-0.16^{*}$       & $-0.17^{*}$       & $-0.09^{*}$       & $-0.11^{*}$       & $-0.10^{*}$       \\
                                                        & $ [-0.22; -0.08]$ & $ [-0.24; -0.07]$ & $ [-0.25; -0.09]$ & $ [-0.16; -0.02]$ & $ [-0.19; -0.03]$ & $ [-0.20; -0.00]$ \\
Chilean nacionality (Ref. = Non-Chilean)                & $0.03$            & $0.28$            & $-0.10$           & $-0.24$           & $-0.21$           & $-0.02$           \\
                                                        & $ [-0.57;  0.63]$ & $ [-0.01;  0.57]$ & $ [-0.54;  0.33]$ & $ [-0.56;  0.08]$ & $ [-0.71;  0.29]$ & $ [-0.79;  0.76]$ \\
Indigenous ethnicity (Ref. = Non-indigenous)            & $-0.03$           & $-0.10$           & $0.02$            & $-0.02$           & $0.02$            & $-0.07$           \\
                                                        & $ [-0.12;  0.07]$ & $ [-0.22;  0.01]$ & $ [-0.11;  0.15]$ & $ [-0.12;  0.08]$ & $ [-0.13;  0.16]$ & $ [-0.21;  0.07]$ \\
Co-residence with both parents (Ref. = No co-residence) & $0.01$            & $0.06$            & $0.01$            & $0.00$            & $0.15^{*}$        & $0.12^{*}$        \\
                                                        & $ [-0.08;  0.09]$ & $ [-0.03;  0.14]$ & $ [-0.07;  0.10]$ & $ [-0.08;  0.08]$ & $ [ 0.06;  0.24]$ & $ [ 0.02;  0.22]$ \\
Parental education                                      & $0.02$            & $0.02$            & $-0.02$           & $0.00$            & $0.00$            & $0.00$            \\
                                                        & $ [-0.01;  0.05]$ & $ [-0.01;  0.05]$ & $ [-0.03;  0.00]$ & $ [-0.02;  0.02]$ & $ [-0.01;  0.02]$ & $ [-0.02;  0.02]$ \\
Wave (Ref.= 2016)                                       &                   &                   &                   &                   &                   &                   \\
                                                        &                   &                   &                   &                   &                   &                   \\
\quad Wave 2018                                         & $0.05$            & $0.00$            & $0.07$            & $0.03$            & $0.02$            & $-0.00$           \\
                                                        & $ [-0.02;  0.12]$ & $ [-0.08;  0.08]$ & $ [-0.01;  0.15]$ & $ [-0.04;  0.11]$ & $ [-0.06;  0.10]$ & $ [-0.09;  0.09]$ \\
\quad Wave 2023                                         & $0.13^{*}$        & $0.11^{*}$        & $0.23^{*}$        & $0.20^{*}$        & $0.15^{*}$        & $0.15^{*}$        \\
                                                        & $ [ 0.04;  0.22]$ & $ [ 0.00;  0.21]$ & $ [ 0.14;  0.32]$ & $ [ 0.12;  0.28]$ & $ [ 0.05;  0.25]$ & $ [ 0.05;  0.26]$ \\
\hline
R$^2$                                                   & $0.05$            & $0.07$            & $0.09$            & $0.05$            & $0.06$            & $0.07$            \\
Adj. R$^2$                                              & $0.04$            & $0.05$            & $0.08$            & $0.04$            & $0.05$            & $0.06$            \\
Num. obs.                                               & $878$             & $745$             & $716$             & $773$             & $912$             & $861$             \\
RMSE                                                    & $0.44$            & $0.43$            & $0.45$            & $0.44$            & $0.48$            & $0.45$            \\
N Clusters                                              & $497$             & $449$             & $443$             & $470$             & $536$             & $525$             \\
\hline
\multicolumn{7}{l}{\scriptsize{Note: Cells contain regression coefficients with confidence intervals in parentheses. $^*$ Null hypothesis value outside the confidence interval..}}
\end{tabular}
}
\label{table:coefficients}
\end{center}

}

\end{table}%

\begin{table}

\caption{\label{tbl-robus2}Effects of intergenerational occupational
mobility on preferences for pension commodification excluding
intermediate category, with covariates and wave fixed effects.}

\centering{

\begin{center}
\scalebox{0.7}{
\begin{tabular}{l c c c c c c}
\hline
 & Low-Middle & Low-High & Middle-High & Middle-Low & High-Middle & High-Low \\
\hline
Intercept                                               & $1.84^{*}$        & $1.76^{*}$        & $2.39^{*}$        & $2.40^{*}$        & $2.14^{*}$        & $2.35^{*}$        \\
                                                        & $ [ 1.08;  2.60]$ & $ [ 1.02;  2.49]$ & $ [ 1.80;  2.97]$ & $ [ 1.88;  2.92]$ & $ [ 1.45;  2.82]$ & $ [ 1.53;  3.18]$ \\
Mobility treatment                                      & $-0.03$           & $-0.03$           & $0.24^{*}$        & $-0.01$           & $-0.12$           & $-0.22^{*}$       \\
                                                        & $ [-0.16;  0.09]$ & $ [-0.16;  0.11]$ & $ [ 0.10;  0.38]$ & $ [-0.13;  0.11]$ & $ [-0.26;  0.01]$ & $ [-0.37; -0.07]$ \\
Age (in years)                                          & $0.00$            & $-0.00$           & $-0.00$           & $-0.00$           & $0.00$            & $-0.00$           \\
                                                        & $ [-0.01;  0.01]$ & $ [-0.01;  0.01]$ & $ [-0.01;  0.00]$ & $ [-0.01;  0.00]$ & $ [-0.00;  0.01]$ & $ [-0.01;  0.01]$ \\
Female (Ref. = Male)                                    & $-0.24^{*}$       & $-0.23^{*}$       & $-0.27^{*}$       & $-0.16^{*}$       & $-0.16^{*}$       & $-0.22^{*}$       \\
                                                        & $ [-0.35; -0.12]$ & $ [-0.39; -0.07]$ & $ [-0.42; -0.12]$ & $ [-0.29; -0.03]$ & $ [-0.30; -0.02]$ & $ [-0.39; -0.06]$ \\
Chilean nacionality (Ref. = Non-Chilean)                & $0.07$            & $0.28$            & $-0.16$           & $-0.34$           & $-0.20$           & $-0.07$           \\
                                                        & $ [-0.73;  0.87]$ & $ [-0.33;  0.89]$ & $ [-0.69;  0.37]$ & $ [-0.76;  0.07]$ & $ [-0.82;  0.43]$ & $ [-0.75;  0.61]$ \\
Indigenous ethnicity (Ref. = Non-indigenous)            & $0.03$            & $-0.08$           & $-0.02$           & $0.04$            & $0.05$            & $-0.21$           \\
                                                        & $ [-0.14;  0.20]$ & $ [-0.28;  0.12]$ & $ [-0.28;  0.25]$ & $ [-0.15;  0.23]$ & $ [-0.20;  0.31]$ & $ [-0.49;  0.07]$ \\
Co-residence with both parents (Ref. = No co-residence) & $0.00$            & $0.05$            & $-0.09$           & $-0.01$           & $0.15$            & $0.15$            \\
                                                        & $ [-0.14;  0.15]$ & $ [-0.10;  0.21]$ & $ [-0.23;  0.06]$ & $ [-0.16;  0.13]$ & $ [-0.01;  0.32]$ & $ [-0.03;  0.33]$ \\
Parental education                                      & $0.01$            & $0.02$            & $-0.04^{*}$       & $-0.01$           & $0.00$            & $-0.01$           \\
                                                        & $ [-0.04;  0.07]$ & $ [-0.04;  0.07]$ & $ [-0.08; -0.01]$ & $ [-0.05;  0.02]$ & $ [-0.03;  0.04]$ & $ [-0.05;  0.03]$ \\
Wave (Ref.= 2016)                                       &                   &                   &                   &                   &                   &                   \\
                                                        &                   &                   &                   &                   &                   &                   \\
\quad Wave 2018                                         & $0.02$            & $-0.13$           & $0.10$            & $-0.01$           & $-0.08$           & $-0.18$           \\
                                                        & $ [-0.11;  0.16]$ & $ [-0.28;  0.03]$ & $ [-0.05;  0.26]$ & $ [-0.14;  0.13]$ & $ [-0.24;  0.07]$ & $ [-0.37;  0.01]$ \\
\quad Wave 2023                                         & $0.21^{*}$        & $0.03$            & $0.39^{*}$        & $0.37^{*}$        & $0.30^{*}$        & $0.16$            \\
                                                        & $ [ 0.06;  0.36]$ & $ [-0.15;  0.21]$ & $ [ 0.23;  0.54]$ & $ [ 0.23;  0.51]$ & $ [ 0.13;  0.46]$ & $ [-0.04;  0.36]$ \\
\hline
R$^2$                                                   & $0.04$            & $0.04$            & $0.10$            & $0.06$            & $0.06$            & $0.08$            \\
Adj. R$^2$                                              & $0.03$            & $0.02$            & $0.09$            & $0.05$            & $0.05$            & $0.07$            \\
Num. obs.                                               & $812$             & $677$             & $629$             & $693$             & $808$             & $770$             \\
RMSE                                                    & $0.77$            & $0.73$            & $0.75$            & $0.75$            & $0.82$            & $0.79$            \\
N Clusters                                              & $480$             & $431$             & $415$             & $445$             & $499$             & $494$             \\
\hline
\multicolumn{7}{l}{\scriptsize{Note: Cells contain regression coefficients with confidence intervals in parentheses. $^*$ Null hypothesis value outside the confidence interval..}}
\end{tabular}
}
\label{table:coefficients}
\end{center}

}

\end{table}%

\begin{table}

\caption{\label{tbl-robus3}Effects of intergenerational occupational
mobility on preferences for pension commodification using the original
5-point likert scale, with covariates and wave fixed effects.}

\centering{

\begin{center}
\scalebox{0.7}{
\begin{tabular}{l c c c c c c}
\hline
 & Low-Middle & Low-High & Middle-High & Middle-Low & High-Middle & High-Low \\
\hline
Intercept                                               & $1.99^{*}$        & $1.76^{*}$        & $2.91^{*}$        & $2.86^{*}$        & $2.53^{*}$        & $2.59^{*}$        \\
                                                        & $ [ 0.81;  3.17]$ & $ [ 0.82;  2.70]$ & $ [ 2.19;  3.64]$ & $ [ 2.24;  3.47]$ & $ [ 1.59;  3.47]$ & $ [ 1.34;  3.83]$ \\
Mobility treatment                                      & $-0.03$           & $0.03$            & $0.30^{*}$        & $-0.02$           & $-0.19$           & $-0.33^{*}$       \\
                                                        & $ [-0.21;  0.15]$ & $ [-0.16;  0.22]$ & $ [ 0.11;  0.48]$ & $ [-0.18;  0.14]$ & $ [-0.38;  0.00]$ & $ [-0.54; -0.12]$ \\
Age (in years)                                          & $0.00$            & $-0.00$           & $-0.00$           & $-0.00$           & $0.00$            & $0.00$            \\
                                                        & $ [-0.01;  0.01]$ & $ [-0.01;  0.01]$ & $ [-0.01;  0.00]$ & $ [-0.01;  0.00]$ & $ [-0.01;  0.01]$ & $ [-0.01;  0.01]$ \\
Female (Ref. = Male)                                    & $-0.37^{*}$       & $-0.37^{*}$       & $-0.40^{*}$       & $-0.23^{*}$       & $-0.26^{*}$       & $-0.32^{*}$       \\
                                                        & $ [-0.54; -0.20]$ & $ [-0.58; -0.15]$ & $ [-0.60; -0.21]$ & $ [-0.41; -0.06]$ & $ [-0.46; -0.06]$ & $ [-0.55; -0.08]$ \\
Chilean nacionality (Ref. = Non-Chilean)                & $0.10$            & $0.56$            & $-0.25$           & $-0.49^{*}$       & $-0.32$           & $-0.05$           \\
                                                        & $ [-1.17;  1.38]$ & $ [-0.08;  1.20]$ & $ [-0.89;  0.39]$ & $ [-0.93; -0.05]$ & $ [-1.23;  0.59]$ & $ [-1.24;  1.15]$ \\
Indigenous ethnicity (Ref. = Non-indigenous)            & $-0.00$           & $-0.18$           & $-0.00$           & $0.02$            & $0.08$            & $-0.26$           \\
                                                        & $ [-0.24;  0.24]$ & $ [-0.46;  0.10]$ & $ [-0.35;  0.35]$ & $ [-0.24;  0.28]$ & $ [-0.29;  0.44]$ & $ [-0.66;  0.13]$ \\
Co-residence with both parents (Ref. = No co-residence) & $0.01$            & $0.11$            & $-0.06$           & $-0.01$           & $0.28^{*}$        & $0.26^{*}$        \\
                                                        & $ [-0.20;  0.22]$ & $ [-0.11;  0.33]$ & $ [-0.27;  0.15]$ & $ [-0.21;  0.19]$ & $ [ 0.05;  0.50]$ & $ [ 0.01;  0.50]$ \\
Parental education                                      & $0.04$            & $0.03$            & $-0.06^{*}$       & $-0.01$           & $0.01$            & $-0.01$           \\
                                                        & $ [-0.04;  0.11]$ & $ [-0.04;  0.11]$ & $ [-0.10; -0.01]$ & $ [-0.05;  0.04]$ & $ [-0.04;  0.05]$ & $ [-0.07;  0.05]$ \\
Wave (Ref.= 2016)                                       &                   &                   &                   &                   &                   &                   \\
                                                        &                   &                   &                   &                   &                   &                   \\
\quad Wave 2018                                         & $0.08$            & $-0.12$           & $0.16$            & $0.03$            & $-0.05$           & $-0.17$           \\
                                                        & $ [-0.11;  0.26]$ & $ [-0.32;  0.09]$ & $ [-0.04;  0.36]$ & $ [-0.15;  0.21]$ & $ [-0.26;  0.16]$ & $ [-0.42;  0.09]$ \\
\quad Wave 2023                                         & $0.33^{*}$        & $0.14$            & $0.55^{*}$        & $0.53^{*}$        & $0.42^{*}$        & $0.30^{*}$        \\
                                                        & $ [ 0.11;  0.54]$ & $ [-0.11;  0.39]$ & $ [ 0.35;  0.76]$ & $ [ 0.35;  0.72]$ & $ [ 0.18;  0.65]$ & $ [ 0.02;  0.57]$ \\
\hline
R$^2$                                                   & $0.04$            & $0.05$            & $0.10$            & $0.06$            & $0.06$            & $0.08$            \\
Adj. R$^2$                                              & $0.03$            & $0.04$            & $0.09$            & $0.05$            & $0.05$            & $0.07$            \\
Num. obs.                                               & $878$             & $745$             & $716$             & $773$             & $912$             & $861$             \\
RMSE                                                    & $1.10$            & $1.04$            & $1.06$            & $1.07$            & $1.17$            & $1.12$            \\
N Clusters                                              & $497$             & $449$             & $443$             & $470$             & $536$             & $525$             \\
\hline
\multicolumn{7}{l}{\scriptsize{Note: Cells contain regression coefficients with confidence intervals in parentheses. $^*$ Null hypothesis value outside the confidence interval..}}
\end{tabular}
}
\label{table:coefficients}
\end{center}

}

\end{table}%

\begin{table}

\caption{\label{tbl-evalues1}E-values for ATT estimates for Middle-High
trajectory (Cinelli \& Hazlett, 2020 approximation)}

\centering{

\centering
\begin{tabular}{lrrrrrr}
\multicolumn{7}{c}{Outcome: \textit{y}} \\
\hline \hline 
Treatment: & Est. & S.E. & t-value & $R^2_{Y \sim D |{\bf X}}$ & $RV_{q = 1}$ & $RV_{q = 1, \alpha = 0.05}$  \\ 
\hline 
\textit{t} & 0.088 & 0.029 & 2.995 & 1.3\% & 10.7\% & 3.8\% \\ 
\hline 
df = 706 & & \multicolumn{5}{r}{ \small \textit{Bound (1x sexo)}: $R^2_{Y\sim Z| {\bf X}, D}$ = 2.7\%, $R^2_{D\sim Z| {\bf X} }$ = 3.6\%} \\
\end{tabular}

}

\end{table}%

\begin{table}

\caption{\label{tbl-evalues2}E-values for ATT estimates for High-Middle
trajectory (Cinelli \& Hazlett, 2020 approximation)}

\centering{

\centering
\begin{tabular}{lrrrrrr}
\multicolumn{7}{c}{Outcome: \textit{y}} \\
\hline \hline 
Treatment: & Est. & S.E. & t-value & $R^2_{Y \sim D |{\bf X}}$ & $RV_{q = 1}$ & $RV_{q = 1, \alpha = 0.05}$  \\ 
\hline 
\textit{t} & -0.074 & 0.03 & -2.447 & 0.7\% & 7.8\% & 1.6\% \\ 
\hline 
df = 902 & & \multicolumn{5}{r}{ \small \textit{Bound (1x sexo)}: $R^2_{Y\sim Z| {\bf X}, D}$ = 1.5\%, $R^2_{D\sim Z| {\bf X} }$ = 3.9\%} \\
\end{tabular}

}

\end{table}%

\begin{table}

\caption{\label{tbl-evalues3}E-values for ATT estimates for High-Low
trajectory (Cinelli \& Hazlett, 2020 approximation)}

\centering{

\centering
\begin{tabular}{lrrrrrr}
\multicolumn{7}{c}{Outcome: \textit{y}} \\
\hline \hline 
Treatment: & Est. & S.E. & t-value & $R^2_{Y \sim D |{\bf X}}$ & $RV_{q = 1}$ & $RV_{q = 1, \alpha = 0.05}$  \\ 
\hline 
\textit{t} & -0.098 & 0.03 & -3.245 & 1.2\% & 10.5\% & 4.3\% \\ 
\hline 
df = 851 & & \multicolumn{5}{r}{ \small \textit{Bound (1x sexo)}: $R^2_{Y\sim Z| {\bf X}, D}$ = 1.7\%, $R^2_{D\sim Z| {\bf X} }$ = 0.8\%} \\
\end{tabular}

}

\end{table}%




\end{document}
